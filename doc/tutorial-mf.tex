\documentclass[11pt]{article}
%\overfullrule=5pt
\usepackage{amsmath}
\usepackage{amsfonts}
\usepackage{amssymb}

\DeclareMathOperator{\Tr}{Tr}
\DeclareMathOperator{\PSL}{PSL}
\DeclareMathOperator{\SL}{SL}
\newcommand{\Q}{{\mathbb Q}}
\newcommand{\Z}{{\mathbb Z}}
\newcommand{\R}{{\mathbb R}}
\newcommand{\C}{{\mathbb C}}
\newcommand{\z}{\zeta}
\newcommand{\al}{\alpha}
\newcommand{\be}{\beta}
\newcommand{\ga}{\gamma}
\renewcommand{\z}{\zeta}
\renewcommand{\th}{\theta}
\newcommand{\G}{\Gamma}
\newcommand{\new}{\text{new}}

\def\kbd#1{{\tt #1}}

\begin{document}
\pagestyle{plain}

\title{Tutorial for Modular Forms in Pari/GP}
\author{Henri Cohen}

\maketitle

\smallskip

\section{Introduction}

Three packages are available to work with modular forms and related functions
in \kbd{Pari/GP}. The first one is the $L$-function package, which has been
available since 2.9.0 (2015), and computes with general motivic
$L$-functions, and in particular with $L$-functions attached to Dirichlet
characters, Hecke characters, Artin representations, and modular forms. The
name of most functions in this package begins with \kbd{lfun}, such as
\kbd{lfuninit}.

The second is the modular symbol package, whose primary aim
is not so much to compute modular form spaces and modular forms, but to
compute $p$-adic $L$-functions attached to modular forms. The name of most
functions in this package begins with \kbd{ms}, such as \kbd{msinit}.

The third package is the modular forms package, whose aim is to compute in
the standard spaces $M_k(\G_0(N),\chi)$ with $k$ integral or half-integral,
both with modular form \emph{spaces} and individual modular \emph{forms}. The
name of most functions in this package begins with \kbd{mf}, such as
\kbd{mfinit}. The goal of the present manual is to describe this package in
view of a new user's guide, so will be more a tutorial than an actual
reference guide, although we do include such a guide at the end.

\medskip

We can work on five subspaces of $M_k(\G_0(N),\chi)$, through a
corresponding \emph{space flag} in the commands: the cuspidal \emph{new
space} $S_k^{\new}(\G_0(N),\chi)$ (flag = 0), the full cuspidal space
$S_k(\G_0(N),\chi)$ (flag = 1), the old space $S_k^{\text{old}}(\G_0(N),\chi)$
(flag = 2, probably of little use), the space generated by all Eisenstein
series ${\cal E}_k(\G_0(N),\chi)$ (flag = 3), and finally the full space
including the Eisenstein part $M_k(\G_0(N),\chi)$ (flag = 4, which can be
omitted since it is the default). Note that although it can be defined, we have
not included the space $M_k^{\new}$, nor the ``certain space'' of
Zagier--Skoruppa. Also, in the half-integral weight case, we have included
only the full cuspidal space and the full space (flags $1$ and $4$), although
in the future we may also include Kohnen's $+$-space and the corresponding
newspace when $N$ is squarefree.

Note in particular that the package includes the computation of modular
forms of weight $k=1$ and of half-integral weight.

\medskip

The modular forms themselves are represented in a special internal format
which the user need not worry about and which basically is a recipe to
compute successive Fourier coefficients at infinity: if $F$ is a GP modular
form, \kbd{mfcoefs}$(F, 10)$ will give you the Fourier coefficients at infinity
from $a(0)$ to $a(10)$ of the modular form corresponding to $F$ as a row
\emph{vector} (if you want a power series expansion, use the GP function
\kbd{Ser}, see below). Many operations are available on such objects, but the
most important thing the user needs to know is that the number of Fourier
coefficients need not be specified in advance: the command
\kbd{mfcoefs}$(F,n)$ is valid for any integer $n$. We will of course explain
the details of this below.

\medskip

Finally, note that we may roughly divide the complexity of available functions
into three levels:

\begin{enumerate}
\item The first level includes all the basic modular form and modular spaces
creation and operations. The most time-consuming functions are on the one hand
those dealing with forms and spaces involving Dirichlet characters of large
order, and on the other hand modular spaces of weight $1$ when there exist
``exotic'' forms. Reasonable levels (for low weight) can go up to a few
thousands.
\item The second level needs technical information about spaces generated by
products of two Eisenstein series, and is quite expensive, but allows to do
a number of computations which would be almost impossible otherwise, such as
Fourier expansions of $f|_k\gamma$ (hence at any cusp), numerical
evaluation of modular forms at any point in the upper half-plane, even close
to the real axis, or $L$-functions attached to an arbitrary form.
Reasonable levels (for low weight) can go up to a few hundred.
\item The third level (which uses the second level functions) allows more
numerical computations such as period polynomials, modular symbols,
Petersson products, special value polynomials, etc\dots
\end{enumerate}

\section{Creation of Modular Forms}

In \kbd{Pari/GP} modular forms can be created in three different ways:

\begin{itemize}\item As \emph{basic modular forms}, i.e., forms attached (or
  not) to different mathematical objects, and which are of so frequent use
  that we have implemented them so that the user has them at his disposal.
  Examples: \kbd{mfDelta} (Ramanujan's delta), \kbd{mfEk}
  (Eisenstein series of weight $k$ on the full modular group; of course
  we also have more general Eisenstein series), \kbd{mffrometaquo}
  (eta quotients), \kbd{mffromell} (modular form attached to an elliptic
  curve over $\Q$), \kbd{mffromqf} (modular form attached
  to a lattice).
\item From existing forms by applying \emph{operations}.
  Examples: multiplication/division, linear combination,
  derivation and integration, Serre derivative, RC-brackets, Hecke and
  Atkin--Lehner operations, expansion and diamond operators, etc\dots
\item Through the creation of the modular form \emph{spaces}:
  typically, if only \kbd{mf=mfinit} is applied, then a
  basis of forms is obtained by the command \kbd{mfbasis(mf)}. On the other
  hand the command \kbd{mfeigenbasis(mf)} produces the canonical basis of
  eigenforms (in some order).\end{itemize}

\section{A First Session: working with Leaves}

This is now a tutorial session. We will see sample commands as we go along.

\begin{verbatim}
? D = mfDelta(); V = mfcoefs(D, 8)
% = [0, 1, -24, 252, -1472, 4830, -6048, -16744, 84480]
\end{verbatim}

The command \kbd{mfcoefs(D,n)} gives the vector of Fourier coefficients (at
infinity) $[a(0),a(1),\dots,a(n)]$ (note that there are $n+1$ coefficients).
This is a compact representation, but if you prefer power series you
can use \kbd{Ser(V,q)} (convert a vector into a power series).

\begin{verbatim}
% = q - 24*q^2 + 252*q^3 - 1472*q^4 + 4830*q^5 - 6048*q^6\
      - 16744*q^7 + 84480*q^8 + O(q^9)
\end{verbatim}
(This simple-minded recipe only works when the form has rational
coefficients. Make sure to use \kbd{q = varhigher("q")} first if the form has
non-rational algebraic coefficients to avoid problems with variable priorities.)
Similarly

\begin{verbatim}
? E4 = mfEk(4); E6 = mfEk(6); apply(f->mfcoefs(f,3),[E4,E6])
% = [[1, 240, 2160, 6720], [1, -504, -16632, -122976]]
? E43 = mfpow(E4, 3); E62 = mfpow(E6, 2);
? DP = mflinear([E43, E62], [1, -1]/1728);
? mfcoefs(DP, 6)
% = [0, 1, -24, 252, -1472, 4830, -6048]
? mfisequal(D, DP)
% = 1
\end{verbatim}

Self-explanatory. Note that there is a command \kbd{mfcoef(F, n)} (without
the final ``s'') which simply outputs the coefficient $a(n)$.
A final example of the same type:

\begin{verbatim}
? F = mffrometaquo([1,2; 11,2]); mfcoefs(F,10)
% = [0, 1, -2, -1, 2, 1, 2, -2, 0, -2, -2]
? G = mffromell(ellinit("11a1"))[2];
? mfisequal(F, G)
% = 1
\end{verbatim}

  Here, \kbd{mffrometaquo} takes as argument a matrix representing an
  \emph{eta quotient}, here $\eta(1\times\tau)^2\eta(11\times\tau)^2$.

  The second component of the \kbd{mffromell} output is
  the modular form associated to the elliptic curve by modularity.

\section{A Second Session: Modular Form Spaces}

In the first session, we have seen a few preinstalled modular forms (that
we can call \emph{leaves}), and a number of operations on them. All reasonable
operations have been implemented (if some are missing, please tell us). We are
now going to work with \emph{spaces} of modular forms.

\begin{verbatim}
? mf = mfinit([1,12]); L = mfbasis(mf); #L
% = 2
? mfdim(mf)
% = 2
\end{verbatim}

This creates the full space of modular forms of level $1$ and weight $12$.
This space is created thanks to an almost random basis that one can obtain
using \kbd{mfbasis}, and we see either by asking for the number of elements
of \kbd{L} or by using the command \kbd{mfdim}, that it has dimension $2$,
not surprising. We can see it better by writing:

\begin{verbatim}
? mfcoefs(L[1],6)
% = [691/65520, 1, 2049, 177148, 4196353, 48828126, 362976252]
? mfcoefs(L[2],6)
% = [0, 1, -24, 252, -1472, 4830, -6048]
\end{verbatim}
or simply
\begin{verbatim}
? mfcoefs(mf,6)    \\ apply mfcoefs to mfbasis elements
% =
[691/65520     0]
[        1     1]
[     2049   -24]
[   177148   252]
[  4196353 -1472]
[ 48828126  4830]
[362976252 -6048]
\end{verbatim}

Note two things: first, the Eisenstein series are given before the cusp forms
(this may change, but for now this is the case), and second, the Eisenstein
series is normalized so that it is the coefficient $a(1)$ which is equal to
$1$, and not $a(0)$. In particular, here at least, it is a normalized
Hecke eigenform.

If we want to work only in the cuspidal space $S_{12}(\G)$, we simply use
the flag $1$, such as:

\begin{verbatim}
? mf = mfinit([1,12], 1); L = mfbasis(mf); #L
% = 1
? mfcoefs(L[1],6)
% = [0, 1, -24, 252, -1472, 4830, -6048]
\end{verbatim}

Let us now look at higher dimensional cases. In the following example, we
consider the \emph{new space} (flag = $0$), although in the present case
this is the same as the cuspidal space:

\begin{verbatim}
? mf = mfinit([35,2], 0); L = mfbasis(mf); #L
% = 3
? for (i = 1, 3, print(mfcoefs(L[i], 10)))
[0, 3, -1, 0, 3, 1, -8, -1, -9, 1, -1]
[0, -1, 9, -8, -11, -1, 4, 1, 13, 7, 9]
[0, 0, -8, 10, 4, -2, 4, 2, -4, -12, -8]
\end{verbatim}

These are essentially random cusp forms. Usually, you want the eigenforms:
this is obtained by the function \kbd{mfeigenbasis} (note in passing that
\kbd{B=mfeigenbasis(mf)} adds components to \kbd{mf}, so that the next
call is instantaneous). You can ask for the defining number fields with the
command \kbd{mffields}. Note that these commands act only on the new space,
but the package also accepts the cuspidal space (but not the others),
although the result is only about the new space.

\begin{verbatim}
? mffields(mf)
% = [y, y^2 - y - 4]
? L = mfeigenbasis(mf); #L
% = 2
? mfcoefs(L[1],10)
% = [0, 1, 0, 1, -2, -1, 0, 1, 0, -2, 0]
? mfcoefs(L[2],4)
% = [Mod(0, y^2 - y - 4), Mod(1, y^2 - y - 4),\
     Mod(-y, y^2 - y - 4),Mod(y - 1, y^2 - y - 4),\
     Mod(y + 2, y^2 - y - 4)]
? lift(mfcoefs(L[2],10))
% = [0, 1, -y, y - 1, y + 2, 1, -4, -1, -y - 4, -y + 2, -y]
\end{verbatim}

The command \kbd{mffields} gives the polynomials in the variable $y$ defining
the number field extensions on which
the eigenforms are defined. Here, one of the fields is $\Q$, the other is
$\Q(\sqrt{17})$. To obtain the eigenforms, we use \kbd{mfeigenbasis}, and
there are only two and not three, since the one defined on $\Q(\sqrt{17})$
goes together with its conjugate. Asking directly \kbd{mfcoefs(L[2],4)} gives
the coefficients as \kbd{polmods}, not easy to read, so it is usually
preferable to \emph{lift} them, giving the last command, where in the output
we must of course remember that $y$ stands for \emph{one} of the two roots
of $y^2-y-4=0$, i.e., $(1\pm\sqrt{17})/2$.

In fact, for some numerical computations, we really need the
coefficients of the eigenform embedded in $\C$, and not just as abstract
algebraic numbers (in our case of trivial character, they will be in $\R$).
This is why a few functions (most notably \kbd{mfeval} and \kbd{lfunmf})
will return a \emph{vector} of results and not a scalar when called on such
a form.

The first eigenform found above is \emph{rational}, hence by the modularity
theorem there exists up to isogeny a unique elliptic curve to which it
corresponds. We check this by writing

\begin{verbatim}
? [mf,F] = mffromell(ellinit("35a1")); mfcoefs(F, 10)
% = [0, 1, 0, 1, -2, -1, 0, 1, 0, -2, 0]
? mfisequal(F, L[1])
% = 1
\end{verbatim}

For a more typical example (still with no character):

\begin{verbatim}
? [ mfdim([96, 2], flag) | flag <- [0..4] ]
% = [2, 9, 7, 15, 24]
\end{verbatim}

This gives us the dimensions of the new space, the cuspidal space,
the old space, the space of Eisenstein series, and the whole space of
modular forms.

Just for fun, we write (recall that the default is the full space):

\begin{verbatim}
? mf = mfinit([96,2]); L = mfbasis(mf);
? for (i = 12, 15, print(mfcoefs(L[i], 15)))
[23/24, 1, 3, 4, 7, 6, 12, 8, 15, 13, 18, 12, 28, 14, 24, 24]
[31/24, 1, 3, 4, 7, 6, 12, 8, 15, 13, 18, 12, 28, 14, 24, 24]
[47/24, 1, 3, 4, 7, 6, 12, 8, 15, 13, 18, 12, 28, 14, 24, 24]
[95/24, 1, 3, 4, 7, 6, 12, 8, 15, 13, 18, 12, 28, 14, 24, 24]
\end{verbatim}

Apparently, these four Eisenstein series differ only by their constant
term, which is of course not possible. Indeed:

\begin{verbatim}
? F = mflinear([L[14],L[12]],[1,-1]); mfcoefs(F, 50)
% = [1, 0, 0, 0, 0, 0, 0, 0, 0, 0, 0, 0, 0, 0, 0, 0, 0, 0,\
     0, 0, 0, 0, 0, 0, 24, 0, 0, 0, 0, 0, 0, 0, 0, 0, 0, 0,\
     0, 0, 0, 0, 0, 0, 0, 0, 0, 0, 0, 0, 24, 0, 0]
? G = mfhecke(F, 24); mfcoefs(G, 12)
% = [1, 24, 24, 96, 24, 144, 96, 192, 24, 312, 144, 288, 96]
? mftobasis(mf, G)
% = [0, 0, 0, 0, 24, 0, 0, 0, 0, 0, 0, 0, 0, 0, 0, 0, 0, 0,\
     0, 0, 0, 0, 0, 0]~
? 24*mfcoefs(L[5], 12)
% = [1, 24, 24, 96, 24, 144, 96, 192, 24, 312, 144, 288, 96]
\end{verbatim}

The first command shows that the Eisenstein series differ on their $n$-th
Fourier coefficient for $n=0$, $24$, and $48$, and the second command applies
the Hecke operator $T_{24}$ (sometimes denoted $U_{24}$) to the difference,
whose effect is to replace $a(n)$ by $a(24n)$, giving the much more
compact output of $G$. The last commands show that $G$ is equal to
$24$ times the fifth Eisenstein series \kbd{L[5]}.

\begin{verbatim}
? mf=mfinit([96,2],0); mffields(mf)
% = [y, y]
? L = mfeigenbasis(mf); for(i=1, 2, print(mfcoefs(L[i], 16)))
[0, 1, 0, 1, 0, 2, 0, -4, 0, 1, 0, 4, 0, -2, 0, 2, 0]
[0, 1, 0, -1, 0, 2, 0, 4, 0, 1, 0, -4, 0, -2, 0, -2, 0]
? Fa = mffromell(ellinit("96a1"))[2]; mfcoefs(Fa, 16)
% = [0, 1, 0, 1, 0, 2, 0, -4, 0, 1, 0, 4, 0, -2, 0, 2, 0]
? Fb = mffromell(ellinit("96b1"))[2]; mfcoefs(Fb, 16)
% = [0, 1, 0, -1, 0, 2, 0, 4, 0, 1, 0, -4, 0, -2, 0, -2, 0]
\end{verbatim}

The \kbd{mffromell} function returns a triple \kbd{[mf,F,C]},
where \kbd{mf} is the modular form cuspidal space to which \kbd{F} belongs,
\kbd{F} is the rational eigenform corresponding to the elliptic curve by
modularity, and \kbd{C} is the vector of coefficients of \kbd{F} on the
basis in \kbd{mf}, which we recall is usually not a basis of eigenforms
(otherwise \kbd{F} would belong to this basis).

Note also that \kbd{Fa} and \kbd{Fb} are twists of one another:

\begin{verbatim}
? mfisequal(mftwist(Fa, -4), Fb)
% = 1
\end{verbatim}

\section{Interlude: Dirichlet characters}

There are many ways to represent multiplicative characters on $(\Z/N\Z)^*$ in
\kbd{Pari/Gp}, we will list them by increasing order of sophistication,
restricting to characters with complex values:

\begin{itemize}

\item A quadratic character $(D/.)$ (Kronecker symbol) is described by
the integer $D$. For instance $1$ is the trivial character.

\item There is a (non-canonical but fixed) bijection between $(\Z/N\Z)^\times$
and its character group, via \emph{Conrey labels}. So \kbd{Mod}$(a,N)$
represents a character whenever $a$ is coprime to $N$. This makes it easy
to loop on all characters without worrying too much about which is which.
In this labeling, \kbd{Mod(1,N)} is the trivial character, and characters
are multiplied/divided by performing the corresponding operation on their
Conrey labels.

\item The finite abelian group $G = (\Z/N\Z)^*$ is written
$$G = \bigoplus_{i\leq n}\; (\Z/d_i\Z) \cdot g_i,$$
with $d_n \mid \dots \mid d_2 \mid d_1$ (SNF condition), all $d_i > 0$, and
$\prod_i d_i = \phi(N)$. The SNF condition makes the $d_i$ unique, but the
generators $g_i$, of respective order $d_i$, are definitely not unique. The
$\oplus$ notation means that all elements of $G$ can be written uniquely as
$\prod_i g_i^{n_i}$ where $n_i \in \Z/d_i\Z$. The $g_i$ are the so-called
\emph{SNF generators} of $G$. The command \kbd{znstar}$(N)$ outputs the SNF
structure (group order, $d_i$ and  $g_i$), but $G = \kbd{znstar}(N, 1)$ is
needed to initialize a group we can work with: most importantly we can now
solve discrete logarithm problems and decompose elements on the $g_i$.

A character on the abelian group $\oplus (\Z/d_j\Z) g_j$ is given by a row
vector $\chi = [a_1,\ldots,a_n]$ of integers $0\leq a_i  < d_i$ such that
$\chi(g_j) = e(a_j / d_j)$ for all $j$, with the standard notation $e(x) :=
\exp(2i\pi x)$. In other words, $\chi(\prod g_j^{n_j}) = e(\sum a_j n_j /
d_j)$. In this encoding $[0,\dots,0]$ is the trivial character. Of course
a character $\chi$ must always be given as a \emph{pair} $[G,\chi]$,
since $\chi$ is meaningless without knowledge of the $(g_i)$ or the $(d_i)$.
\end{itemize}

The command \kbd{znchar}$(S)$ converts a datum describing a character to the
third form $[G,\chi]$. The command \kbd{znchartokronecker} converts a
character of order $\leq 2$ to the first form $(D/.)$, and functions such
as \kbd{zncharconductor}, \kbd{znchartoprimitive}, and \kbd{zncharinduce}
allow to restrict or extend characters between different $(\Z/M\Z)^*$.

Note the important fact that it is necessary to give the two arguments $G$ and
$\chi$ separately to these functions, for instance \kbd{zncharconductor(G,chi)}
(and not \kbd{zncharconductor([G,chi])}).

Functions such as \kbd{charmul}, \kbd{chardiv}, \kbd{charpow},
\kbd{charorder} or \kbd{chareval} apply to more general abelian characters
than characters on $(\Z/N\Z)^\times$, whence the prefix \kbd{char} instead of
\kbd{znchar}.

\section{A Third Session: Nontrivial Characters}

Recall that a nontrivial character can be represented either by a discriminant
$D$ (not necessarily fundamental), the character being the Legendre--Kronecker
symbol $(D/n)$, or by its \emph{Conrey label} in $(\Z/N\Z)^\times$, for
instance \kbd{Mod(161,633)} (which has order 42, as \kbd{znorder} tells us).

Defining modular form spaces with character is as simple as without:
we replace the parameters $[N,k]$ by $[N,k,\chi]$.
Instead of, say, \kbd{mf=mfinit([35,2])}, one can write
\kbd{mf=mfinit([35,2,5], 0)}, where 5 is the quadratic character $(5/.)$. Thus:

\begin{verbatim}
? mf = mfinit([35,2,5],0); mffields(mf)
% = [y^2 + 1]
? F = mfeigenbasis(mf)[1]; lift(mfcoefs(F, 10))
% = [0, 1, 2*y, -y, -2, -y - 2, 2, -y, 0, 2, -4*y + 2]
\end{verbatim}
where in the last output $y$ is equal to one of the two roots of $y^2+1=0$,
i.e., $\pm i$.

Working with nontrivial characters allows us in particular to work with odd
weights, and in particular in weight $1$:

\begin{verbatim}
? mf = mfinit([23,1,-23], 0); mfdim(mf)
% = 1
? F = mfbasis(mf)[1]; mfcoefs(F, 16)
% = [0, 1, -1, -1, 0, 0, 1, 0, 1, 0, 0, 0, 0, -1, 0, 0, -1]
? mfgaloistype(mf,F)
% = 6
\end{verbatim}

The last output means that the image in $\PSL_2(\C)$ of the projective
representation associated to $F$ is of type $D_6$. Note that an ''exotic''
representation is given by a negative number, opposite of the cardinality
of the projective image.

Since this form is of dihedral type, it can be obtained via theta functions.
Indeed:

\begin{verbatim}
? F1 = mffromqf([2,1;1,12])[2]; V1 = mfcoefs(F1, 16)
% = [1, 2, 0, 0, 2, 0, 4, 0, 4, 2, 0, 0, 4, 0, 0, 0, 2]
? F2 = mffromqf([4,1;1,6])[2]; V2 = mfcoefs(F2, 16)
% = [1, 0, 2, 2, 2, 0, 2, 0, 2, 2, 0, 0, 4, 2, 0, 0, 4]
? (V1 - V2)/2
% = [0, 1, -1, -1, 0, 0, 1, 0, 1, 0, 0, 0, 0, -1, 0, 0, -1]
? mfisequal(F, mflinear([F1, F2], [1, -1]/2))
% = 1
\end{verbatim}

Here we were lucky in that we ``knew'' that the correct character was
$(-23/n)$. But what if we did not know this ? The first observation is
that modular form spaces corresponding to Galois conjugate characters
are isomorphic ($\chi$ is Galois conjugate to $\chi'$ if $\chi'=\chi^m$
for some $m$ coprime to the order of $\chi$). Thus, it is sufficient
to find a representative of each equivalence class, and this is given by
the \kbd{GP} commands \kbd{G=znstar(N,1); chargalois(G)}, where $N$ is the
level of the desired character (note that $N$ will not necessarily be
the conductor of the characters). This exactly outputs a list of representative
of each equivalence class (do not for now try to understand the details of
this command, nor the fact that \kbd{chargalois} and \kbd{znstar} have
optional parameters). However, this is not quite yet what we want.
Although only for efficiency, we want characters with the same parity
as the weight, otherwise the corresponding modular form spaces will be $0$.
This is achieved by the \kbd{GP} command \kbd{zncharisodd(G,chi)} which
does what you think it does. Let us first do this for $N=23$: we write
\begin{verbatim}
? G = znstar(23, 1);
? L = [chi | chi<-chargalois(G), zncharisodd(G,chi)]; #L
% = 2
? [mfdim([23,1,[G,chi]], 0) | chi <- L ]
% = [0, 1]
? [charorder(G,chi) | chi <- L]
% = [22, 2]
\end{verbatim}

This tells us that (up to Galois conjugation) there are two possible odd
characters, one, of order $22$, giving a $0$-dimensional space, the other
being the quadratic character given above. Note that \kbd{chargalois}
returns (orbits of) characters attached to an arbitrary abelian finite group
$G$ while \kbd{mfinit} expects a \emph{pair} \kbd{[G,chi]} for some
\kbd{znstar} $G$, as written above.

When doing long explorations with all characters of a certain level, it
is preferable to use \emph{wildcards}. For instance, instead of the above
one can write:

\begin{verbatim}
? mfall = mfinit([23,1,0], 0); #mfall
% = 1
? mf = mfall[1]; mfdim(mf)
% = 1
? mfparams(mf)
% = [23, 1, -23, 0]
\end{verbatim}

This does not exactly give us the same information: the third parameter $0$
in the first command asks for \emph{all} nonempty spaces of level $23$ and
weight $1$, and the program tells us that there is only one, of dimension $1$.
The last command \kbd{mfparams} outputs \kbd{[N,k,CHI,space]}, so here tells
us that the corresponding character is the Kronecker--Legendre symbol $(-23/n)$.

Using wildcards, let us explore levels in certain ranges: we write
\begin{verbatim}
wt1exp(lim1,lim2)=
{ my(mfall,mf,chi,v);
  for (N = lim1, lim2,
    mfall = mfinit([N,1,0], 0); /* use wildcard, more efficient */
    for (i=1, #mfall,
      mf = mfall[i];
      chi = mfparams(mf)[3]; /* nice format: D or Mod(a,N) */
      [ print([N,chi,-t]) | t<-mfgaloistype(mf), t < 0 ]
    )
  );
}
\end{verbatim}

For instance, \kbd{wt1exp(1,230)} outputs in 4 seconds

\begin{verbatim}
[124, Mod(87, 124), 12]
[133, Mod(83, 133), 12]
[148, Mod(105, 148), 24]
[171, Mod(94, 171), 12]
[201, Mod(104, 201), 12]
[209, Mod(197, 209), 12]
[219, Mod(8, 219), 12]
[224, Mod(95, 224), 12]
[229, Mod(122, 229), 24]
[229, Mod(122, 229), 24]
\end{verbatim}

Thus, the smallest exotic $A_4$ form is in level $124$ and the smallest $S_4$
form is in level $148$. Note that in level $229$, we have two (non Galois
conjugate) eigenforms of type $S_4$.

If we type \kbd{wt1exp(633,633)}, in 6 seconds we obtain \kbd{[633, Mod(107,
633), 60]}, and this level is indeed the lowest level for which there exists
a type $A_5$ form. The character orders are obtained either as
\kbd{znorder(chi)} (since all the \kbd{chi} are \kbd{intmods}), or
using the general construction
\begin{verbatim}
  [G,v] = znstar(chi);
  ord = charorder(G,v)
\end{verbatim}
where we first convert \kbd{chi} to a general abelian character in
$[G,\chi]$ format.

\section{Leaf Functions}

Although we have already seen most of these functions in the first session,
we repeat some of examples here.

\subsection{Functions Created from Scratch}

We now start a slightly more systematic exploration of the available functions.
We begin by \emph{leaf functions}, i.e., functions created from scratch or
from a given mathematical object.

\begin{verbatim}
? D = mfDelta(); mfcoefs(D, 5)
% = [0, 1, -24, 252, -1472, 4830]
? E4 = mfEk(4); mfcoefs(E4, 5)
% = [1, 240, 2160, 6720, 17520, 30240]
? E6 = mfEk(6);
? D2 = mflinear([mfpow(E4, 3), mfpow(E6, 2)], [1, -1]/1728);
? mfisequal(D, D2)
% = 1
\end{verbatim}

  Self-explanatory. More complicated Eisenstein series:

\begin{verbatim}
? E3 = mfeisenstein(1, 1, -3]); mfcoefs(E3, 10)
% = [1/6, 1, 0, 1, 1, 0, 0, 2, 0, 1, 0]
? E4 = mfeisenstein(5, -4, 1); mfcoefs(E4, 10)
% = [5/4, 1, 1, -80, 1, 626, -80, -2400, 1, 6481, 626]
? H2 = mfEH(5/2); mfcoefs(H2,10)
% = [1/120, -1/12, 0, 0, -7/12, -2/5, 0, 0, -1, -25/12, 0]
\end{verbatim}

The \kbd{mfeisenstein(k,c1,c2)} command generates the Eisenstein series of weight
$k$ and characters \kbd{c1} and \kbd{c2}. The \kbd{mfEH(k)} command is specific
to half-integral weight $k$ and generates the Cohen--Eisenstein series of
weight $k$.

\begin{verbatim}
? T = mfTheta(); mfcoefs(T,16)
% = [1, 2, 0, 0, 2, 0, 0, 0, 0, 2, 0, 0, 0, 0, 0, 0, 2]
? mf = mfinit([4, 5, -4]); mftobasis(mf, mfpow(T, 10))
% = [64/5, 4/5, 32/5]
? B = mfbasis(mf); apply(mfdescribe, B)
% = ["F_5(1, -4)", "F_5(-4, 1)", "TR^new([4, 5, -4, y])"]
? mfisCM(B[3])
% = -4
\end{verbatim}

  Here, we compute the coefficients of $\th^{10}$ on the basis of \kbd{mf}
  (we know of course the level, weight, and character). We then apply
  the \kbd{mfdescribe} function, which tells us that the first two forms in
  the basis are Eisenstein series, and the third one is some trace form
  on the cuspidal new space. However, the last command says that this third
  basis element is a \emph{CM form}, so that its coefficients can be computed
  just as fast as those of Eisenstein series, so that there does exist
  an explicit formula for the number of representations as a sum of ten
  squares.

  Keeping the above sessions, we can also write:

\begin{verbatim}
? mftobasis(mf, mfpow(H2, 2))
% = [1/18000, 1/18000, -3/2000]~
\end{verbatim}

\subsection{Functions Created from Mathematical Objects}

\begin{verbatim}
? [mf,F,co] = mffromell(ellinit("26b1")); co
% = [1/2, 1/2]~
? mfcoefs(F,10)
% = [0, 1, 1, -3, 1, -1, -3, 1, 1, 6, -1]
\end{verbatim}

This creates the modular form attached by modularity to the second isogeny
class of elliptic curves over $\Q$ for conductor $26$. The result is a
3-component vector: \kbd{mf} is the modular form space, $F$ the modular form,
and \kbd{co} are the coefficients of $F$ on the basis of \kbd{mf}.

Similarly, there are functions \kbd{mffromqf} (from quadratic forms),
\kbd{mffromlfun} (from $L$-functions attached to eigenforms), and
\kbd{mffrometaquo}:

\begin{verbatim}
? F = mffrometaquo([1, 2; 11, 2]); mfcoefs(F, 10)
% = [0, 1, -2, -1, 2, 1, 2, -2, 0, -2, -2]
? F = mffrometaquo([1, 2; 2, -1]); mfparams(F)
% = [16, 1/2, 1, y]
? mfcoefs(F, 10)
% = [1, -2, 0, 0, 2, 0, 0, 0, 0, -2, 0]
\end{verbatim}

The \kbd{mfparams} command tells us that $F\in M_{1/2}(\G_0(16))$.

\section{Atkin, Hecke and Expanding Operators}

\begin{verbatim}
? mf = mfinit([96,4], 0); mfdim(mf)
% = 6
? M = mfheckemat(mf, 7)
% =
[0    0   0    372    696   0]

[0    0  36      0      0 -96]

[0 27/5   0 -276/5 -276/5   0]

[1    0 -12      0      0  62]

[0    0   1      0      0 -16]

[0 -3/5   0   14/5  -16/5   0]
? P = charpoly(M)
% = x^6 - 1456*x^4 + 209664*x^2 - 2985984
? factor(P)
% =
[x - 36 1]

[x - 12 1]

[ x - 4 1]

[ x + 4 1]

[x + 12 1]

[x + 36 1]
\end{verbatim}

Note a few things: first, the matrix of the Hecke operator $T(7)$ does not
have integral coefficients. Indeed, recall that the basis of modular forms
in \kbd{mf} is mostly random, so there is no reason for the matrix to be
integral. On the other hand, since the eigenvalues of Hecke operators are
algebraic integers, the characteristic polynomial of $T(7)$ must be monic
with integer coefficients. As it happens, it factors completely into
linear factors to the power $1$, so all the eigenvalues of $T(7)$ are in
fact in $\Z$: this immediately shows that the splitting will be entirely
rational and the eigenforms with integer coefficients. Let's check:

\begin{verbatim}
? mffields(mf)
% = [y, y, y, y, y, y]
? L = mfeigenbasis(mf); for(i=1,6,print(mfcoefs(L[i],16)))
[0, 1, 0, 3, 0, 10, 0, 4, 0, 9, 0, -20, 0, 70, 0, 30, 0]
[0, 1, 0, 3, 0, 2, 0, 12, 0, 9, 0, 60, 0, -42, 0, 6, 0]
[0, 1, 0, 3, 0, -14, 0, -36, 0, 9, 0, -36, 0, 54, 0, -42, 0]
[0, 1, 0, -3, 0, 10, 0, -4, 0, 9, 0, 20, 0, 70, 0, -30, 0]
[0, 1, 0, -3, 0, 2, 0, -12, 0, 9, 0, -60, 0, -42, 0, -6, 0]
[0, 1, 0, -3, 0, -14, 0, 36, 0, 9, 0, 36, 0, 54, 0, 42, 0]
\end{verbatim}

We see that of the six eigenforms, the last three are twists of the first
three.

There also exists the command \kbd{G=mfhecke(mf,F,n)}, which given a modular
form $F$ in \kbd{mf}, outputs the modular form $T(n)F$.

\begin{verbatim}
? mf=mfinit([96,6],0); mffields(mf)
% = [y, y, y, y, y, y, y^2 - 31, y^2 - 31]
? mfatk = mfatkininit(mf,3);
% factor(charpoly(mfatk[2]/mfatk[3]))
% =
[x - 1 5]

[x + 1 5]
\end{verbatim}

This requires a little explanation: the command \kbd{mfatkininit(mf,3)}
computes a number of quantities necessary to work with the Atkin--Lehner
operator $W_3$ in the space \kbd{mf}. The main part of the result is
the second component, which is essentially the matrix of $W_3$ on the
basis of \kbd{mf}, and which is guaranteed to have exact coefficients
(here rational). However in the general case, the matrix of $W_3$
is equal to \kbd{mfatk[2]/mfatk[3]}, where \kbd{mfatk[3]} may be an
inexact complex number. For now you need not worry about the first component.

Thus, the eigenvalues (or possibly the pseudo-eigenvalues) must be of modulus
$1$, and in the case of a quadratic character defined modulo $N/Q$, they
are equal to $\pm1$ in even weight, to $\pm i$ in odd weight. Here,
$1$ and $-1$ both occur $5$ times. However, this does not tell us which
eigenvalues correspond to each eigenspace. For this, we do the following:

\begin{verbatim}
? mfatkineigenvalues(mf,3)
% = [[-1], [-1], [-1], [1], [1], [1], [-1, -1], [1, 1]]
? mf=minit([96,3,-3],0); mffields(mf)
% = [y^4 + 8*y^2 + 9, y^4 + 4*y^2 + 1]
? mfatkineigenvalues(mf,32)
% = [[I, -I, -I, I], [-I, I, I, -I]]
? mfatkineigenvalues(mf,3)
% = [[a, -conj(a), -a, conj(a)], [b, -conj(b), conj(b), -b]]
\end{verbatim}

  The first command tells us that in the six rational eigenspaces, the first
  three have eigenvalue $-1$, the other three $+1$, and in the eigenspaces
  of dimension $2$, the first eigenspace has both eigenvalues $-1$, the
  second both $+1$. As is seen from the next lines, it is of course not
  necessary for the eigenvalues of $W_Q$ in the same eigenspace to be equal.

  In the next two commands, we are now in a case where the character is
  non trivial and the weight odd. The eigenvalues are now $\pm i$, and not
  equal in the same eigenspace.

  Finally, the last command is a case where the character is not defined modulo
  $N/Q=96/3=32$, so we only have pseudoeigenvalues, which are simply of
  absolute value $1$ by Atkin--Lehner theory. Here, $a$ and $b$ are
  complicated complex numbers and \kbd{conj} denotes the complex conjugate
  (using the \kbd{algdep} command, one can check that $a$ is a root of
  $9x^4+10x^2+9=0$ and $b$ is a root of $3x^4-2x^2+3=0$.

Note that when the character is (trivial or) quadratic and defined modulo
$N/Q$ the output is always rounded, but otherwise, the eigenvalues are given
as approximate complex numbers.

As for the Hecke operators, there exists an \kbd{mfatkin} command, whose
syntax is \kbd{mfatkin(mfatk, F)}, where \kbd{mfatk} is the output of
an \kbd{mfatkininit} command and $F$ is in the space \kbd{mfatk}, and which
outputs the modular form $F|_kW_Q$, where $Q$ is implicit in \kbd{mfatk}.

Finally note the \kbd{mfbd} expanding command which computes $B(d)F$:

\begin{verbatim}
? E4 = mfEk(4); mfcoefs(E4,6)
% = [1, 240, 2160, 6720, 17520, 30240, 60480]
? F = mfbd(E4,2); mfcoefs(F,6)
% = [1, 0, 240, 0, 2160, 0, 6720]
\end{verbatim}

\section{Algebraic Functions on Modular Forms}

Here we give examples of functions on modular forms which do not involve
any approximate numerical computation. We have already mentioned the most
important ones: \kbd{mfhecke}, \kbd{mfatkin}, and \kbd{mfbd}.

\begin{verbatim}
? E4 = mfEk(4); F = mfderivE2(E4); mfcoefs(F,5)
% = [-1/3, 168, 5544, 40992, 177576, 525168]
? E6 = mfEk(6); mfisequal(F, mflinear([E6], [-1/3]))
% = 1
? G = mfbracket(E4, E6, 1); mfcoefs(G,5)
% = [0, -3456, 82944, -870912, 5087232, -16692480]
? mfisequal(G, mflinear([mfDelta()], [-3456]))
% = 1
\end{verbatim}

\medskip

In the first commands, we compute the Serre derivative of $E_4$, and
check that it is equal to $-E_6/3$. The name \kbd{mfderivE2} of course
comes from the fact that the Serre derivative involves the quasi-modular
Eisenstein series $E_2$. Note that there exists the function \kbd{mfderiv}
(including to negative order, corresponding to integration), which is
provided for the user's convenience for certain computations, but whose
output is outside the range of modular forms.

The second computation checks that the first Rankin--Cohen bracket of
$E_4$ and $E_6$ is a multiple of $\Delta$.

You may complain that it is heavy to write an \kbd{mflinear} command as above
simply to compute a scalar multiple of a form. But nothing prevents you from
defining in a script that you read at the beginning of your session:

\begin{verbatim}
mfscalmul(F,s)=mflinear([F],[s]);
mfadd(F,G)=mflinear([F,G],[1,1]);
mfsub(F,G)=mflinear([F,G],[1,-1]);
\end{verbatim}

There also exist the natural operations on modular forms \kbd{mfmul},
\kbd{mfdiv} (which may result in modular functions, i.e., with poles),
and \kbd{mfpow}. There is also a function \kbd{mfshift} (multiply or divide
by a power of $q$), but which again takes us outside the range of modular
forms.

\begin{verbatim}
? E4 = mfEk(4); F = mftwist(E4, -3); mfcoefs(F, 6)
% = [0, 240, -2160, 0, 17520, -30240, 0, 82560]
? mfparams(F)
% = [9, 4, 1, y]
? mf = mfinit([4, 5, -4], 1); F = mfbasis(mf)[1]; mfcoefs(F, 10)
% = [0, 1, -4, 0, 16, -14, 0, 0, -64, 81, 56]
? mfisCM(F)
% = -4
? G = mftwist(F, -4); mfcoefs(G, 10)
% = [0, 1, 0, 0, 0, -14, 0, 0, 0, 81, 0]
? mfparams(G)
% = [16, 5, -4, y]
? mfconductor(mfinit(G, 1), G)
% = 8
\end{verbatim}

  This session illustrates a number of important issues concerning
  \emph{twisting}. In the first commands, we twist $E_4$ by the quadratic
  character $-3$ (in the present implementation, only twisting by quadratic
  characters is allowed), and we see that the resulting form has level
  $9=(-3)^2$. Fine. In the next command, we compute the unique form
  in $S_4(\G_0(5),\chi_{-4})$, and see that it has CM by $\Q(\sqrt{-4})$.

  However, note that the form is not equal to the form twisted by the
  character $\chi_{-4}$ (only the coefficients of $q^n$ with $n$ prime to $4$
  are equal, the others vanish). The \kbd{mfparams} command tells us that
  the twisted form has level $16=(-4)^2$. However, the final command tells
  us that in fact it has level $8$: \kbd{mfconductor} gives the smallest
  level on which the form is defined.

\begin{verbatim}
? mf = mfinit([96,2], 1); L = mfbasis(mf);
? apply(x->mfconductor(mf,x), L)
% = [24, 48, 96, 32, 96, 48, 96, 96, 96]
? apply(x->mftonew(mf,x)[1][1..2], L)
% = [[24, 1], [24, 2], [24, 4], [32, 1], [32, 3],\
 [48, 1], [48, 2], [96, 1], [96, 1]]
\end{verbatim}

Here we compute the full cuspidal space $S_2(\G_0(96))$, of dimension $9$,
and we ask which is the lowest level on which each form in the basis
is defined. This list shows that there is one form $F_1$ in level $24$
which, by applying $B(d)$ with $d=2$ and $d=4$ gives a form of level $48$
and one of level $96$. Then a form $F_2$ in level $32$, by applying $B(3)$
gives a form of level $96$, a form $F_3$ in level $48$, by applying $B(2)$
gives a form of level $96$, and finally two genuine forms of level $96$
(so that the dimension of the newspace is equal to $2$, which we can check
by typing \kbd{mfdim([96,2],0)}).

The last command \kbd{mftonew} checks all this; look at the precise description
of the command.

\section{Cusps and Cosets}

Recall that in the present version of the package, the only congruence
subgroup that is considered is $\G_0(N)$, so when we consider cusps in
the geometrical sense, they are cusps of $\G_0(N)$, and cosets are
right cosets of $\G_0(N)$ in $\G$, so that $\G=\bigsqcup_j\G_0(N)\ga_j$.

The function \kbd{mfcusps(N)} gives the list of all (equivalence classes of)
cusps of $\G_0(N)$, \kbd{mfcuspwidth(N,cusp)} gives the width of the cusp;
these are linked to the \emph{geometry}. On the other hand, the notion
of \emph{regularity} of a cusp is linked to the specific modular form space,
and the function \kbd{mfcuspisregular([N,k,CHI],cusp)} determines if the cusp
is regular or not:

\begin{verbatim}
? C = mfcusps(108)
% = [0, 1/2, 1/3, 2/3, 1/4, 1/6, 5/6, 1/9, 2/9, 1/12,\
       5/12, 1/18, 5/18, 1/27, 1/36, 5/36, 1/54, 1/108]
? apply(x->mfcuspwidth(108,x), C)
% = [108, 27, 12, 12, 27, 3, 3, 4, 4, 3, 3, 1, 1, 4, 1, 1, 1, 1]
? NK = [108,3,-4];
? apply(x->mfcuspisregular(NK,x), C)
% = [1, 0, 1, 1, 1, 0, 0, 1, 1, 1, 1, 0, 0, 1, 1, 1, 0, 1]
? [c | c<-C, !mfcuspisregular(NK,c)]
% = [1/2, 1/6, 5/6, 1/18, 5/18, 1/54]
\end{verbatim}

The first command list the $18$ cusps of $\G_0(108)$ (\kbd{mfnumcusps(108)}
gives this directly, useful if there are thousands of cusps and you do not
want them explicitly), the second command prints their widths, and the last
commands show that the cusps $1/2$, $1/6$, $5/6$, $1/18$, $5/18$, and $1/54$
are irregular in the space $M_3(\G_0(108),\chi_{-4})$, and the others are
regular.

There is another command \kbd{mfcuspval} having to do with cusps, but this
will be mentioned later.

\medskip

\begin{verbatim}
? C = mfcosets(4)
% = [[0, -1; 1, 0], [1, 0; 1, 1], [0, -1; 1, 2], [0, -1; 1, 3],\
     [1, 0; 2, 1], [1, 0; 4, 1]]
? mftocoset(4, [1, 1; 2, 3], C)
% = [[-1, 1; -4, 3], 5]
\end{verbatim}

   The \kbd{mfcosets(N)} command lists all right cosets of $\G_0(N)$ in $\G$.
   Note that in the present implementation the trivial coset is always the
   last one, and is represented by the matrix $[1,0;N,1]$, but since this
   may change one must be careful.

   The \kbd{mftocoset(N,M,C)} command gives a two-component
   vector $[\ga,i]$, where $\ga\in\G_0(N)$ is such that $M=\ga\cdot C[i]$.

\section{The mfslashexpansion command}

We now give examples of the use of advanced features of the package,
which use inexact complex arithmetic. However in many cases the results
are known algebraic numbers, and the program tries to give them exactly
if possible.

\begin{verbatim}
? mf=mfinit([4,6]);B=mfbasis(mf);
? for(i=1,#B,R=mfslashexpansion(mf,B[i],[1,0;2,1],6,,&A);print([A,R]))
[[0, 1], [-1/504, 1, 33, 244, 1057, 3126, 8052]]
[[0, 1], [-1/504, 0, 1, 0, 33, 0, 244]]
[[0, 1], [-1/32256, -1/64, 33/64, -61/16, 1057/64, -1563/32, 2013/16]]
[[0, 1], [0, -1, 0, 12, 0, -54, 0]]
? R = mfslashexpansion(mf,B[1],[0,-1;4,0],6,1,&A); [A,R]
% = [[0, 1], [-8/63, 0, 0, 0, 64, 0, 0]]
? R = mfslashexpansion(mf,B[1],[0,-1;1,0],6,1,&A); [A,R]
% = [[0, 4], [-1/504, 0, 0, 0, 1, 0, 0, 0]]
? mf=mfinit([4,7,-4]); B=mfbasis(mf);
? for(i=1,#B,R=mfslashexpansion(mf,B[i],[1,0;2,1],6,1,&A);print([A,R]))
[[1/2, 1], [1/64, 91/8, 7813/32, 7353/4, 530713/64, 221445/8, 2413405/32]]
[[1/2, 1], [1, -728, 15626, -117648, 530713, -1771560, 4826810]]
[[1/2, 1], [1/16, -15/2, 5/8, 75, -231/16, -465/2, 733/8]]
[[1/2, 1], [2, 0, 20, 0, -462, 0, 2932]]
? mfslashexpansion(mf,B[1],[0,-1;4,0],6)
  ***   at top-level: mfslashexpansion(mf,B[1],[0,-1;4,0],10)
  ***                 ^---------------------------------------
  *** mfslashexpansion: cannot rationalize coeff in bestapprnf.
? R=mfslashexpansion(mf,B[1],[0,-1;4,0],6,0,&A)
% = [0.23828125000000000000000000000000000000*I,\
    -0.015625000000000000000000000000000000000*I,\
    -0.015625000000000000000000000000000000000*I,\
    11.375000000000000000000000000000000000*I,\
    -0.015625000000000000000000000000000000000*I,\
  -244.15625000000000000000000000000000000*I,\
    11.375000000000000000000000000000000000*I]
? A
% = [0, 1]
? bestappr(R)
% = [61/256*I, -1/64*I, -1/64*I, 91/8*I, -1/64*I, -7813/32*I, 91/8*I]
\end{verbatim}

  Here are some detailed explanations. The first space is $M_6(\G_0(4))$,
  of dimension $4$. We ask for $1+6$ terms of the Fourier expansion of
  $F|_6\ga$ for all $F$ in the given basis, and $\ga=[1,0;2,1]$, which is one
  of the possible Fourier expansions at the cusp $1/2$. The last two parameters
  ($1$ and $A$) are important, but for now let us ignore them.
  We obtain the $4$ desired expansions. In the next commands, we do the same
  for the first basis element and $\ga=[0,-1;4,0]$, which is the
  \emph{Fricke involution}, and corresponds to the cusp $0$. The next
  command, which does essentially the same computation, uses $\ga=[0,-1;1,0]$,
  and now $A=[0,4]$ which tells us that the expansion is in powers of
  $q^{1/4}$.

  The next example is the space $M_7(\G_0(4),\chi_{-4})$, also of
  dimension $4$, and we ask the same thing. Here $A=[1/2,1]$ is essential:
  in the general case, $A=[\al,w]$ means that the expansion given
  by \kbd{mfslashexpansion} is in powers of $q^{1/w}=e^{2\pi i\tau/w}$,
  and must be multiplied by $q^{\al}$. Thus, in the present case, we have
  for instance
  $$B[1]|_7\ga=(1/64)q^{1/2}+(91/8)q^{3/2}+\cdots$$

  In the next command, we try to expand $B[1]|_7\ga$ with $\ga$ equal to
  the Fricke involution, and the command fails because the program tries
  to ``rationalize'' the output in a very simple-minded way, and does not
  succeed (this will hopefully change in the future). This is where the
  one-to-last parameter enters: in the next command we set it to $0$ to
  obtain the expansion as raw complex numbers. We recognize them immediately
  using the \kbd{bestappr} command.

  Note that in the special case (like here) where $\ga$ is a Fricke (or more
  generally an Atkin--Lehner) involution, we can proceed otherwise to obtain
  the expansion, and it is then guaranteed that the output will be
  ``rationalized'':

\begin{verbatim}
? mfatk = mfatkininit(mf,4); C = mfatk[3]
% = -1.000000000000000000000000000*I
? F = mfatkin(mfatk, B[1]); mfcoefs(F, 6)
% = [61/256, -1/64, -1/64, 91/8, -1/64, -7813/32, 91/8]
\end{verbatim}

  This tells us that the true expansion of $F|_7W_4$ is the expansion that
  is output divided by the constant $C$, so we recover the previous
  expansion.

\section{Analytic Commands}

The existence of the \kbd{mfslashexpansion} command allows us to do many
useful things. In fact, already the \kbd{mfatkininit} and \kbd{mfatkin}
commands would not be possible without it. Immediate applications are the
\kbd{mfcuspval} command which computes the valuation at cusps, and the
\kbd{mfeval} command, which in addition to computing values in the upper-half
plane (see below), also computes values at the cusps:

\begin{verbatim}
? T = mfTheta(); mf=mfinit(T);C=mfcusps(4)
% = [0, 1/2, 1/4]
? apply(x->mfcuspval(mf,T,x),C)
% = [0, 1/4, 0]
? mfeval(mf,T,C) // Or apply(x->mfeval(mf,T,x),C)
% = [1/2 - 1/2*I, 0, 1]
\end{verbatim}

  More sophisticated is the computation of numerical periods, and more
  generally of \emph{symbols}
  $$\int_{s_1}^{s_2}(X-\tau)^{k-2}F|_k\ga(\tau)\,d\tau\;,$$
  where $s_1$ and $s_2$ are two cusps (e.g., $s_1=0$, $s_2=\infty$):

\begin{verbatim}
? mf = mfinit([96,4],0); B = mfbasis(mf); F1 = B[1];
? FS1 = mfsymbol(mf,F1);
time = 2,272 ms
? mfsymboleval(FS1,[0,oo])
% = 2.0968669678226579060336519703627002478*I*x^2\
  + 0.36368580656317635568444277442842940073*x\
  - 0.049315736834713109138297211986510643780*I
? mfsymboleval(FS1,[1,5/2])
% =  4.1937339356453158120673039407254004956*I*x^2\
  + (0.72737161312635271136888554885685880147\
  - 14.678068774758605342235563792538901735*I)*x\
 + (-1.2729003229711172448955497104995029026\
  + 15.103654043044843600467382361156555509*I)
? mfsymboleval(FS1,[1,2],[0,-1;1,0])
% = (0.54552870984476453352666416164264410111\
   + 2.5224522361088961642654705389803540222*I)*x^2\
 + (-0.72737161312635271136888554885685880148\
   - 6.2906009034679737181009559110881007434*I)*x\
   + 4.1937339356453158120673039407254004956*I
\end{verbatim}

The general strategy for computing these quantities is first to do a
precomputation which only involves \kbd{mf} and the form $F$ using
\kbd{mfsymbol}, which can take a few seconds, but afterwards all the
computations are instantaneous.

Note that if you only want the period polynomial from $0$ to
$\infty$ use \kbd{mfperiodpol(mf,F1)} which gives the same answer as before
but in only $20$ ms.

You may also use \kbd{mfsymboleval} in two other ways, but note that in
this case the precomputation is not used so the computation may be slow:

\begin{verbatim}
? mf=mfinit([96,6],0);F=mfbasis(mf)[1];
? FS=mfsymbol(mf,F);
time = 9,761 ms.
? mfsymboleval(FS,[I,oo])
% = 0.00297212...*I*x^4 + 0.0137806...*x^3 + ... + 0.00610099...
? mfsymboleval(FS,[I,2*I])
% = 0.00296657...*I*x^4 + 0.0137326...*x^3 + ... + 0.00597601...
? mfsymboleval(FS,[I/10000,I])
time = 20,512 ms.
% = 46.3637302...*I*x^4 + 3.8815894...*x^3 + ... + 0.01838690...
? -x^4*subst(mfsymboleval(FS,[I,10000*I],[0,-1;1,0]),x,-1/x)
time = 741 ms.
% = 46.3637302...*I*x^4 + 3.8815894...*x^3 + ... + 0.01838690...
? mfsymboleval([mf,F],[I,oo])
% = 0.00297212...*I*x^4 + 0.0137806...*x^3 + ... + 0.00610099...
? mfsymboleval([mf,F],[I,2*I])
% = 0.00296657...*I*x^4 + 0.0137326...*x^3 + ... + 0.00597601...
\end{verbatim}

These examples illustrate four points:
\begin{enumerate}\item Computing an \kbd{mfsymbol}
may be rather long ($9.8$ seconds in this example), although as already
mentioned, subsequent computations of symbols \emph{between cusps} will then
be instantaneous.
\item As the next three commands show, \kbd{mfsymboleval}
also accepts paths with endpoints in the upper half-plane, but in this case
the computation may be very slow if an endpoint is very close to the real line.
\item The next command shows the use of the extra parameter $\ga$ which
asks to integrate $F|_k\ga$ instead of $F$, here with \kbd{ga=[0,-1;1,0]}.
This allows to perform the same computation in $0.74$ seconds instead of
$20.5$. This is what is done \emph{automatically} by \kbd{mfsymboleval} when
the endpoints are cusps, but here we must do it by hand.
\item The last two commands show a special format which avoids doing the
longish \kbd{mfsymbol} computation: the results are obtained almost
instantaneously \emph{without} using symbols. The price to pay in using this
``cheaper'' format is that the endpoints of the path cannot be cusps other than
oo.
\end{enumerate}

\begin{verbatim}
? mfpetersson(FS1, FS1)
% = 0.00061471684149817788924091516302517391826
? F2 = B[2]; FS2 = mfsymbol(mf, F2);
? mfpetersson(FS2, FS2)
% = 0.0055324515734836010031682364672265652647
? mfpetersson(FS1, FS2)
% = 1.5879887877319313665 E-40 + 7.652958013165934297 E-42*I
\end{verbatim}

  Same remark: once the \kbd{mfsymbol} \kbd{FS2} initialized, all the
  Petersson product computations (as well as others) are essentially immediate.

  Note that since neither \kbd{F1} nor \kbd{F2} are eigenforms, there is
  no reason for their Petersson product to vanish. To prove this rigorously
  we do as follows:

\begin{verbatim}
? BE = mfeigenbasis(mf);
? M = Mat([mftobasis(mf,f) | f<-BE]); M^(-1)
% =
[1  3  10   4 -20  70]

[1  3   2  12  60 -42]

[1  3 -14 -36 -36  54]

[1 -3  10  -4  20  70]

[1 -3   2 -12 -60 -42]

[1 -3 -14  36  36  54]
\end{verbatim}

On the other hand, it is immediate to see that \kbd{BE[i+3]} is a twist
of \kbd{BE[i]} and that as a consequence their Petersson square are equal.

It follows from the shape of the above matrix that the Petersson scalar
product of $B[i]$ with $B[j]$ will vanish when the corresponding scalar
product of the corresponding columns vanish, hence for $(i,j)=(1,2)$,
$(1,4)$, $(1,5)$, $(2,3)$, $(2,6)$, $(3,4)$, $(3,5)$, $(4,6)$, and $(5,6)$.

\smallskip

Note that \kbd{mfpetersson} can also be used for two noncuspidal forms, as
long as the Petersson product converges. Consider the following example:

\begin{verbatim}
? E1=mfeisenstein(5,1,-3);E2=mfeisenstein(5,-3,1);
? mf=mfinit([12,5,-3]); cusps=mfcusps(12);
? apply(x->mfcuspval(mf,E1,x),cusps)
% = [0, 0, 1, 0, 1, 1]
? apply(x->mfcuspval(mf,E2,x),cusps)
% = [1/3, 1/3, 0, 1/3, 0, 0]
? E1S=mfsymbol(mf,E1);E2S=mfsymbol(mf,E2);
? mfpetersson(E1S,E2S)
% = -1.8848216716468969562647734582232071466 E-5\
     - 1.9057659114817512165 E-43*I
? mf3=mfinit([3,5,-3]);E1S=mfsymbol(mf3,E1);E2S=mfsymbol(mf3,E2);
? mfpetersson(E1S,E2S);
time = 16 ms.
? mf96=mfinit([96,5,-3]);E1S=mfsymbol(mf96,E1);E2S=mfsymbol(mf96,E2);
? mfpetersson(E1S,E2S);
time = 3,521 ms.
\end{verbatim}

The first commands create the two Eisenstein series of weight $5$
$E_5(1,\chi_{-3})$ and $E_5(\chi_{-3},1)$, which belong to
$M_5(\G_0(3),\chi_{-3})$. In the next commands, we look at the larger space of
level $12$ and compute the valuations of $E_1$ and $E_2$ at the six cusps of
$\G_0(12)$. We see that at these six cusps one of the two Eisenstein series
vanishes, so the Petersson product will converge, and is computed in the
next command. In the last commands we compute the same product but in level
$3$ and level $96$; because of the normalization, we obtain the same result
(not given), but of course the times are very different: $0.016$ seconds in
level $3$ and $3.5$ seconds in level $96$.

\medskip

There are two more important numerical functions: evaluating a modular form
at a point in the upper half plane, and evaluating the corresponding
$L$-function. We begin by a trivial example:

\begin{verbatim}
? E4 = mfEk(4); mf = mfinit(E4); mfeval(mf,E4,I)
% = 1.4557628922687093224624220035988692874
? 3*gamma(1/4)^8/(2*Pi)^6
% = 1.4557628922687093224624220035988692874
\end{verbatim}

  This is of course a trivial computation, simply sum the $q$-expansion.
  The fact that the value of a modular form with rational coefficients such
  as $E_4$ at a \emph{CM point} such as $i$ has an explicit expression
  is a consequence of complex multiplication.

\begin{verbatim}
? mf = mfinit([12,4],1); F = mfbasis(mf)[1];
? mfeval(mf, F, 1/Pi + 10^(-6)*I)
% = -89811.049350396250531782882568405506024\
   - 58409.940965200894541585402642924371696*I
? mfeval(mf, F, 1/Pi + 10^(-7)*I)
% = 4.8212468504661113183253396691813292261 E-52\
  + 6.7885262281520647908871247541561415340 E-52*I
\end{verbatim}

  Several remarks are in order.

\begin{enumerate}
\item We are evaluating a modular form very near the
  real axis. If the form was in level $1$ such as $E_4$ above, we could use a
  modular transformation to reduce to the evaluation in the fundamental domain
  of $\G$, which would be very fast. Here we do something similar but much
  more sophisticated.
\item The result at height $10^{-7}$ is not a numerical approximation of
  $0$, the exact value is indeed as printed to the given accuracy.
\item It is amusing to see the large oscillations of the value: at height
  $10^{-6}$ the value is still in the $10^5$ range, and at $10^{-7}$ it is
  in the $10^{-52}$ range. Of course it must eventually tend to $0$ since
  $F$ is a cusp form (for $E_4$ it would tend to infinity).
\item When applying \kbd{mfeval} at a \emph{cusp} (as above), the result is
  the value at the cusp, but is in general \emph{not} equal
  to the limit of the value of the modular form when the argument tends to
  the cusp, since this limit is often infinite for a non-cusp form.
\end{enumerate}

Note that when dealing with \emph{eigenforms}, which may have several
embeddings into $\C$, the result will have several components, one for each
embedding:

\begin{verbatim}
? mf = mfinit([23,2],0); F=mfeigenbasis(mf)[1];
? mfeval(mf,F,I)
% = [0.0018695834459685012330841605500720163964,\
     0.0018618146628840767703527958851699552194]
\end{verbatim}

  More generally, this embedding problem affects all numerical functions.
  Continuing the above example:

\begin{verbatim}
? mfparams(F)
% = [23, 2, 1, y^2 - y - 1]
? mfslashexpansion(mf,F,[0,-1;1,0],5,1)
% = [0, -1/23, 1/23*y, -2/23*y + 1/23, -1/23*y + 1/23, 2/23*y]
? FS = mfsymbol(mf,F); mfpetersson(FS,FS)
% =
[0.0039488965740025031688548076498662860147 -1.0827196147167250830 E-40]

[-1.2120247024777595243 E-40 0.0056442542987647835101583821368582485387]
\end{verbatim}

  The $y$ in the second result is thus understood to be \emph{one} of
  the roots of the polynomial $y^2-y-1$, and the result of \kbd{mfpetersson}
  is a $2\times 2$ \emph{diagonal} matrix because of the two embeddings of $F$.

  \smallskip

  The other important evaluation function is that of the $L$-function
  attached to a modular form. In fact, the modular form package only
  creates (in a clever way) the $L$-function, all the rest of the work
  is done by the $L$-function package. Note the important fact that
  the modular form need not be an eigenform or even stable under the
  Fricke involution.

\begin{verbatim}
? E4=mfEk(4); mf=mfinit(E4); LE = lfunmf(mf,E4); lfun(LE, 2)/Pi^2
% = -3.3333333333333333333333333333333333333
? lfun(LE, 0)
% -1
? D = mfDelta(); mf=mfinit(D); LD = lfunmf(mf,D);
? lfunlambda(LD, 3)/lfunlambda(LD, 5)
% = 1.5555555555555555555555555555555555556
? lfunlambda(LD, 1)/lfunlambda(LD, 3)
% = 2.3444283646888567293777134587554269175
? bestappr(%)
% = 1620/691
? mf = mfinit([23,2],0); F = mfbasis(mf)[1]; L = lfunmf(mf,F);
? lfun(L, 2)
% = 1.5959983753450272580976413437480171832
? G = mfeigenbasis(mf)[1]; M = lfunmf(mf,G);
? apply(x->lfun(x,I),M)
% = [-0.15856033373254740657327844579672155664\
    + 0.79671369922504818377602680344686311969*I,\
     -0.10230278816509023908993775663030712037\
    + 0.65954223983092583287784522268295299513*I]
\end{verbatim}

  Note that the constant term $a(0)$ is ignored by the $L$-function, but
  can be recovered thanks to the formula $a(0)=-L(F,0)$.

  The last commands illustrate first the fact that the $L$-functions can be
  computed for non-eigenforms ($F$ is not an eigenform), and second that
  if there are several embeddings, the \kbd{lfunmf} function returns a
  vector of \kbd{lfunmf}, one for each embedding.

Another illustration of the $L$-function package:

\begin{verbatim}
? LIN = lfuninit(LD, [6, 6, 50]);
? ploth(t = 0, 50, lfunhardy(LIN, t))
\end{verbatim}

%\includegraphics[width=\textwidth]{pari3.pdf}

\medskip

\section{The mfeigensearch and mfsearch commands}

The last commands that we want to illustrate are \emph{searching} commands,
i.e., you know some Fourier coefficients and you would like to know which
modular forms match these coefficients in a given range. The
\kbd{mfsearch} command does this very naively, and may in fact disappear
since the user may easily write a taylor-made script for his needs.

The \kbd{mfeigensearch} command is more interesting: the idea
of this command is simple: you believe that you have a modular form,
but you do not know its level, weight, character, or field of definition
of its coefficients, but only a number of its Fourier coefficients, perhaps
only modulo $p$, and you would like to find forms which ``match'' your
given form. In this degree of generality, the search space is too wide
and would take much too long. We have therefore decided to reduce
the generality, so as to make the search more reasonable. Note that this
will probably vary with the different versions of the program, so what is
described here may be more or less restrictive than future versions.

In the present implementation, we assume that the form we are looking
for is a cuspidal \emph{eigenform}, and that its field of definition is $\Q$,
so that its Fourier coefficients are integers, and its character is
(trivial or) quadratic. An example is as follows:

\begin{verbatim}
? L = mfeigensearch([30,4], [[2,2],[3,-1]]); #L
% = 1
? [N, F] = L[1]; mfparams(F)
% = [26, 4, 1, y]
? mfcoefs(F, 10)
% = [0, 1, 2, -1, 4, 17, -2, -35, 8, -26, 34]
\end{verbatim}

The first command asks for all forms as above in weight $4$ and level
up to $30$, such that $a(2)=2$ and $a(3)=-1$. The answer is that there is a
single such form, given in format $[N,F]$, where $F$ is the form
and $N$ is its level, here $N = 26$. We compute its Fourier coefficients up
to $10$ and we see that indeed $a(2)=2$ and $a(3)=-1$.

Note that the way to specify levels and weights may change, for now we
stick to the above format, a two-component vector $[N_0,k]$,
where we look in weight $k$ at levels up to $N_0$, with trivial
or quadratic characters.

To specify the coefficients that we want there are a number of ways. The
simplest, as above, is to give the list of pairs of integers $[p,a(p)]$.
For instance:

\begin{verbatim}
? L = mfeigensearch([80,2], [[2,2], [7,-3]]); #L
% = 1
? [N, F] = L[1]; mfparams(F)
% = [75, 2, 1, y]
? mfcoefs(F, 12)
% = [0, 1, 2, -1, 2, 0, -2, -3, 0, 1, 0, 2, -2]
\end{verbatim}

The coefficient $a(p)$ may also be given as an \kbd{intmod} \kbd{Mod}$(a,m)$
then one looks for a match for $a(p)$ modulo $m$. For instance, we come
back to our first example:
\begin{verbatim}
? L=mfeigensearch([30,4], [[2,Mod(2,5)], [3,Mod(-1,5)]]); #L
% = 2
? apply(x->x[1], L)
% = [26, 26]
? F1 = L[1][2]; mfcoefs(F1, 10)
% = [0, 1, 2, -1, 4, 17, -2, -35, 8, -26, 34]
? F2 = L[2][2]; mfcoefs(F2, 10)
% = [0, 1, 2, 4, 4, -18, 8, 20, 8, -11, -36]
? F = mflinear([F1, F2], [-1, 1]);
? content(mfcoefs(F, mfsturm([26,4])+1))
% = 5
\end{verbatim}

Working modulo $5$, we now find that there are \emph{two} eigenforms
satisfying our criteria, and perhaps surprisingly, both are in weight $4$
and level $26$. The first, \kbd{F1}, is the one found above, with
$a(2)=2$ and $a(3)=-1$. The second, \kbd{F2}, has $a(2)=2$ but
$a(3)=4\equiv-1\pmod5$.

But we can go further and see that this is not a simple coincidence:
the next command shows that both eigenforms seem to be congruent modulo $5$,
at least up to $a(16)$. In fact they are indeed congruent modulo $5$:
to prove this, we use the fact that the basic sturm bound (the one obtained
using \kbd{mfsturm([N,k])}, not \kbd{mfsturm(mf)}) is also valid modulo
$p$. Here the bound is equal to $15$, so the fact that the coefficients
are congruent up to $n=16$ shows that they are congruent for all $n$.

\section{Half-Integral Weight Functions}

\subsection{General Functions}

Many of the commands that we have seen, and most importantly the
\kbd{mfinit} and \kbd{mfdim} command, can be used verbatim in the case
of modular forms of half-integral weight, sometimes with small differences.
\begin{itemize}\item The only preprogrammed leaf functions created from
scratch are \kbd{mfEH}, which gives the Cohen--Hurwitz Eisenstein series
of half-integral weight, and \kbd{mfTheta} which gives the standard Jacobi
theta function of weight $1/2$.
\item Two functions created from mathematical objects can give half-integral
weight forms, \kbd{mfetaquo} and \kbd{mffromqf}:
\end{itemize}

\begin{verbatim}
? F=mffrometaquo([2,5;1,-2;4,-2]);Ser(mfcoefs(F,10),q)
% = 1 + 2*q + 2*q^4 + 2*q^9 + O(q^11)
? T=mfTheta();mfisequal(F,T)
% = 1
? F=mffromqf(2*matid(3))[2];Ser(mfcoefs(F,5),q)
% = 1 + 6*q + 12*q^2 + 8*q^3 + 6*q^4 + 24*q^5 + O(q^6)
? mfisequal(F,mfpow(T,3))
% = 1
\end{verbatim}

\begin{itemize}\item The only spaces which are \emph{directly} available
by \kbd{mfinit} and \kbd{mfdim} are the full cuspidal space and the full
modular form space. The new space can be defined in some cases but indirectly,
using Kohnen's theory, see below.
\item The only Hecke operators $T(n)$ which are nonzero are those where
$n$ is a square (we have not programmed the $T(p)$ with $p$ dividing the
level).
\end{itemize}

\subsection{Specific Functions}

The most important specific function in half-integral weight is
\kbd{mfshimura}, which computes the Shimura lift corresponding to a
fundamental discriminant $D$ (1 by default) and also returns an \kbd{mf} space
containing the lift:

\begin{verbatim}
? mf=mfinit([60,5/2],1); F=mfbasis(mf)[1]; D = [1,5,8,12,13,17,21];
? for (i=1, #D, \
    [mf2,G] = mfshimura(mf,F,D[i]); print(mfcoefs(G,10)))
[0, 1, 2, 0, 2, -1, -2, 6, 6, -3, -10]
[0, 0, 0, -1, 0, 0, 20, 0, 0, -2, 0]
[0, 0, 0, 0, 0, 0, 0, 0, 0, 0, 0]
[0, 0, 0, 0, 0, 24, 0, 0, 0, 0, -48]
[0, 0, 0, 0, 0, 0, 0, 0, 0, 0, 0]
[0, 0, 0, 0, 0, 0, 0, 0, 0, 0, 0]
[0, 1, 0, 3, 52, -5, -120, 14, -156, 3, 0]
? mfdescribe(mf)
% = "S_5/2(G_0(60, 1))"
? mfdescribe(mf2)
% = "S_4(G_0(30, 1))"
\end{verbatim}

Two things to notice: first the image can be identically $0$.
Second, the program takes some time (20 seconds for the above), because
computing a Shimura image takes time proportional to $D^4$.

\smallskip

The other specific functions are related to the Kohnen $+$-space. Continuing
the above example:

\begin{verbatim}
? K=mfkohnenbasis(mf); matsize(K)
% = [14, 4]
? K[,1]
% = [-1, 0, 0, 2, 0, 0, 0, 0, 0, 0, 0, 0, 0, 0]~
? F=mflinear(mf,K[,1]);
? Ser(mfcoefs(F,40),q)
% = -q + 2*q^4 - 4*q^16 + 3*q^21 - 6*q^24 + 5*q^25 + 12*q^36 + O(q^41)
\end{verbatim}

The first command shows that although the dimension of the cuspidal space
is $14$, that of the Kohnen $+$-space is $4$; if desired, the corresponding
modular forms can be obtained by \kbd{mflinear(mf,K[,j])} for each $j$ as
done in the next command. The Fourier expansion of $F$ given
on the last line shows that the only nonzero coefficients of $q^n$ occur
when $n\equiv0,1\pmod4$. Continuing:

\begin{verbatim}
? [mf2,FS]=mfshimura(mf,F); mfparams(FS)
% = [15, 4, 1, y]
? [mf2,FS]=mfshimura(mf,mfbasis(mf)[1]); mfparams(FS)
% = [30, 4, 1, y]
\end{verbatim}

These commmands show that the image of an element of the Kohnen $+$-space has
level $15=60/4$, while the second shows that the image of a random form has
level $30=60/2$.

\smallskip

The last important command related to the Kohnen $+$-space is
\kbd{mfkohnenbijection}. This allows, in half-integral weight, to compute
the new space, its splitting, and the eigenforms:

\begin{verbatim}
? [mf3,M,K,shi] = mfkohnenbijection(mf);
? M*mfheckemat(mf3,11)*M^(-1)
% =
[ 48  24  24  24]

[  0  32   0 -20]

[-48 -72 -40 -72]

[  0   0   0  52]
? mf30=mfinit(mf3,0); B0=mfbasis(mf30);
? BNEW=apply(x->mflinear(mf,K*M*mftobasis(mf3,x)),B0); #BNEW
% = 2
? BE=mfeigenbasis(mf30);
? BEIGEN = [mflinear(mf,K*M*mftobasis(mf3,f)) | f<-BE ];
? Ser(mfcoefs(BEIGEN[1],36),q)
% = q + q^4 - 3*q^9 - 5*q^16 + 6*q^21 + 3*q^24 - 5*q^25 - 3*q^36 + O(q^37)
? Ser(mfcoefs(BEIGEN[2],40),q)
% = q^5 + q^8 - 3*q^12 - 4*q^17 + 3*q^20 + q^32 + 6*q^33 + O(q^37)
? mfcoefs(BEIGEN[1],10^4);
time = 43,060 ms.
? T=mfTheta();F1T=mfmul(BEIGEN[1],T);mf1t=mfinit(F1T,1);V=mftobasis(mf1t,F1T);
? COEFS=mfcoefs(mf1t,10^4)*V;
time = 11,080 ms.
\end{verbatim}

The \kbd{mfkohnenbijection} command computes a square matrix $M$ giving
a Hecke-module isomorphism from the space $S_{2k-1}(\G_0(N),\chi^2)$ to the
Kohnen $+$-space $S_k^+(\G_0(4N,\chi))$. Note that this
makes sense only when $N$ is squarefree.

Thus, $M$ allows to transport all problems from the ``difficult'' space
$S=S_k^+(\G_0(4N),\chi)$ to the ``easy'' space $S_{2k-1}(\G_0(N),\chi^2)$.
For instance, the next command (essentially instantaneous) gives the matrix of
the Hecke operator $T(121)$ on $S$; a direct implementation using the action
of $T(121)$ would take $10.6$ seconds.

The vector \kbd{BNEW} computed afterwards gives a basis of the Kohnen new
space $S_k^{+,\text{new}}(\G_0(4N),\chi)$, here of dimension $2$.

The vector \kbd{BEIGEN} computed in a similar way contains the eigenfunctions
of this new space. The example of \kbd{BEIGEN[2]} shows that, contrary to
the integral weight case, these eigenfunctions can have vanishing coefficient
of $q^1$. Note that we \emph{know} by construction that the image of
\kbd{BEIGEN[j]} by any Shimura lift is a multiple of \kbd{BE[j]} (with the
same index $j$).

Computing $10^4$ coefficients of these eigenforms (or more generally forms
of half-integral weight) takes some time, here $43$ seconds. The last commands
show a more efficient way to compute them: we first multiply by
\kbd{mfTheta()}; computing the coefficients of the resulting \kbd{F1T}
would not be any faster. However, the \kbd{mfcoefs(mf1t,10\^{}4)} command
uses the special structure of the modular form space to compute the
coefficients more efficiently, here in $11$ seconds, almost $4$ times faster.
Of course, to recover the coefficients of \kbd{BEIGEN[1]} itself one can write
something like:

\begin{verbatim}
? concat(0, Vec(Ser(COEFS)/Ser(mfcoefs(T,10^4))));
\end{verbatim}

This takes negligible time. The initial \kbd{concat} is due to the fact that
the \kbd{Pari/GP} \kbd{Vec} command ignores the initial valuation of a power
series.

\smallskip

The above construction of the new space and the eigenforms being so useful,
a specific function exists for this purpose: instead of all the above,
simply write \kbd{[mf30,BNEW,BEIGEN]=mfkoheneigenbasis(mf,bij)}. Here
\kbd{BNEW} and \kbd{BEIGEN} will be matrices whose columns are the
coefficients of a basis of the Kohnen new space and of the eigenforms
respectively, and \kbd{mf30} the corresponding new space of integral weight.

\section{Reference Manual for the Modular Forms Package}

We give a brief description in alphabetical order of all the functions
specific to the package. To use the package, it is sometimes necessary to
use functions on characters or functions of the \kbd{lfun} package, but
those will not be described here.

Note that when a modular form $F$ can be embedded in $\C$ in several ways
(typically for eigenforms), some functions give a vector (or even a matrix
for bilinear operations) of results, one for each embedding: this occurs
specifically for \kbd{lfunmf}, \kbd{mfeval}, \kbd{mfmanin}, \kbd{mfpetersson},
\kbd{mfsymboleval}. This will not always be specified.

\smallskip

\kbd{getcache()}: returns technical information about auto-growing caches.

\kbd{lfunmf(mf,\{F\})}: creates the $L$-function associated to $F$, for use
in the \kbd{lfun} package, where $F$ need not be an eigenform. If $F$ is
omitted, output all $L$-functions associated to the eigenforms. If
$F$ (or the eigenforms) have several embeddings in $\C$, output the vector
of the corresponding \kbd{lfunmf}.

\kbd{mfatkin(mfatk, F)}: computes $F|_k W_Q$, where $Q\Vert N$, where
\kbd{mfatk} must have been initialized by \kbd{mfatk=mfatkininit(mf,Q)}.

\kbd{mfatkineigenvalues(mf, Q)}: \kbd{mf} being a cuspidal or new space
and $Q$ a primitive divisor of $N$, output the vector of Atkin--Lehner
eigenvalues or pseudo-eigenvalues for each Galois eigenspace.

\kbd{mfatkininit(mf,Q)}: necessary initialization function for the
\kbd{mfatkin} function. The output is \kbd{[mfb,M,C,mf]}, where $C$ is a
complex constant, $M/C$ is the matrix of the Atkin--Lehner operator
$W_Q$ from the space \kbd{mf} to the space \kbd{mfb} (set equal to $0$ if
equal to \kbd{mf}). The matrix $M$ is guaranteed to be with exact coefficients
(rational or \kbd{polmods}).

\kbd{mfbasis(mf,\{space=4\})}: gives a basis of the space of modular forms
\kbd{mf}, either output by an \kbd{mfinit} command, in which case \kbd{space}
is ignored, or \kbd{mf=[N,k,CHI]} (use \kbd{mfeigenbasis} for the eigenforms).

\kbd{mfbd(F,d)}: gives $B(d)(F)$, $B(d)$ expanding operator.

\kbd{mfbracket(F,G,\{m=0\})}: $m$th Rankin--Cohen bracket of $F$ and $G$.

\kbd{mfcoef(F,n)}: $n$th Fourier coefficient $a(n)$ of $F$.

\kbd{mfcoefs(F,n)}: vector $[a(0),a(1),...,a(n)]$ of the Fourier coefficients
of $F$ up to $n$. If $F$ is a modular form \emph{space}, give the matrix
whose columns are the vectors of the Fourier coefficients of the basis.

\kbd{mfconductor(F)}: smallest $M$ such that $F$ belongs to
$M_k(\G_0(M),\chi)$.

\kbd{mfcosets(N)}: list of right cosets of $\G$ modulo $\G_0(N)$. In the
present implementation, the trivial coset is the last and represented by
the matrix $[1,0;N,1]$.

\kbd{mfcuspisregular(NK,cusp)}: \kbd{NK} being $[N,k,\chi]$ or an \kbd{mf},
determine if the cusp is regular or not.

\kbd{mfcusps(N)}: list of cusps of $\G_0(N)$.

\kbd{mfcuspval(mf,F,cusp)}: valuation of modular form $F$ at \kbd{cusp}, which
can be a rational number or \kbd{oo}.

\kbd{mfcuspwidth(N,cusp)}: width of \kbd{cusp} in $\G_0(N)$.

\kbd{mfDelta()}: Ramanujan's Delta function of weight $12$.

\kbd{mfderiv(F,\{m=1\})}: $m$th derivative $q.d/dq$ of $F$, where
$m$ can be negative, corresponding to integration (the constant term
is then set to $0$ by convention). The result is
only quasi-modular.

\kbd{mfderivE2(F,\{m=1\})}: $m$th Serre derivative $q.d/dq F-kE_2F/12$.

\kbd{mfdescribe(F,\{\&G\})}: $F$ being a modular form or a modular form space,
gives a human-readable description of $F$. If the address of $G$ is given,
put in it the vector of parameters of the outmost operator defining $F$
(empty vector if $F$ is a leaf or a modular form space).

\kbd{mfdim(mf,\{space=4\})}: dimension of the space \kbd{mf}, where \kbd{mf}
can also be of the form $[N,k,\chi]$ in which case \kbd{space} is taken into
account. \kbd{mf} can also be of the form $[N,k,0]$, where $0$ is a wildcard,
in which case it gives detailed information for each character $\chi$
for which the corresponding space of level $N$, weight $k$ and given
character is nonzero: each result is of the form
\kbd{[order,Conrey,dim,dimdih]}, where \kbd{Conrey} is the
Conrey label for the character, \kbd{order} is its order, \kbd{dim} is
the dimension of the corresponding space, and \kbd{dimdih}, which is
computed only in weight $1$, is the dimension of the subspace of dihedral
forms.

\kbd{mfdiv(F,G)}: division of \kbd{F} by \kbd{G}.

\kbd{mfEH(k)}: $k$ being half-integral, gives the Cohen--Eisenstein series
of weight $k$ on $\G_0(4)$.

\kbd{mfeigenbasis(mf)}: \kbd{mf} being a new or cuspidal space, gives
(in some order) the basis of normalized eigenforms.

\kbd{mfeigensearch(NK,AP)}: search for normalized eigenforms with
integer coefficients in spaces specified by \kbd{NK}, satisfying conditions
satisfied by \kbd{AP}. \kbd{NK} is a pair $[N_0,k]$, the search being in
weight $k$ up to level $N_0$ with trivial or quadratic character.
\kbd{AP} is a list of pairs $[[p_1,a(p_1)],...,[p_n,a(p_n)]]$, where $a(p)$
is either an integer or an \kbd{intmod} (match modulo $a(p)\kbd{.mod}$).

\kbd{mfeisenstein(k,\{CHI1\},\{CHI2\})}: Eisenstein series $E_k(\chi_1)$ or
$E_k(\chi_1,\chi_2)$, normalized so that $a(1)=1$ (so \kbd{mfeisenstein(k)} without
any character argument is equal to \kbd{mfEk(k)} multiplied by $-B_k/(2k)$).

\kbd{mfEk(k)}: Eisenstein series $E_k$ for the full modular group normalized
so that $a(0)=1$, including for $k=2$.

\kbd{mfeval(mf,F,vtau)}: evaluation of $F$ at the point \kbd{vtau} (or a
vector of points) in the completed upper half-plane. If $F$ is an eigenform
with several embeddings in $\C$, evaluate at each embedding.

\kbd{mffields(mf)}: \kbd{mf} being a new or cuspidal space, gives the list of
relative polynomials defining the number field extensions for all the Galois
orbits of the eigenforms. \kbd{mf} can also be a modular form, in which case
the result is the number field extension of $\Q(\chi)$ in which the Fourier
coefficients of \kbd{mf} lie.

\kbd{mffromell(e)}: \kbd{e} being an elliptic curve defined over $\Q$ in
\kbd{ellinit} format, gives \kbd{[mf,F,coe]}, where \kbd{F} is the eigenform
corresponding to \kbd{e} by modularity, \kbd{mf} the corresponding new space,
and \kbd{coe} the coefficients of \kbd{F} on the basis of \kbd{mf}.

\kbd{mffrometaquo(eta,\{flag=0\})}: \kbd{eta} being a matrix representing an
eta quotient, gives the corresponding modular form or function. If the
result is not a modular form or function, return an error if \kbd{flag=0}, or
$0$ otherwise. If the result has negative valuation, normalize to valuation
$0$.

\kbd{mffromlfun(L)}: \kbd{L} being the $L$-function of a self-dual
modular form with rational coefficients, for instance a rational eigenform,
retun \kbd{[mf,F,coe]}, where \kbd{F} is the form, \kbd{mf} the corresponding
space and \kbd{coe} the coefficients of \kbd{F} on the basis of \kbd{mf}.
More generally when \kbd{L} is the $L$-function of a modular form with
inexact complex coefficients, only return \kbd{mf}, a modular form space
which \emph{possibly} contains the form we are looking for.

\kbd{mffromqf(Q,\{P\})}: \kbd{Q} being an even integral quadratic form of even
dimension and \kbd{P} an optional homogeneous spherical polynomial with
respect to \kbd{Q}, gives \kbd{[mf,F,coe]}, where \kbd{F} is the theta
function associated to \kbd{Q} and \kbd{P}, \kbd{mf} the corresponding space,
and \kbd{coe} the coefficients of \kbd{F} on the basis of \kbd{mf}.

\kbd{mfgaloistype(mf,\{F\})}: \kbd{mf} being either $[N,1,\chi]$ or
a new or cuspidal space of weight $1$ forms, outputs the type of the projective
representations attached to all the eigenforms in \kbd{mf}, or only that of
\kbd{F} if it is given. The output is $n$ for $D_n$, or $-12$, $-24$, $-60$ for
$A_4$, $S_4$, $A_5$.

\kbd{mfhecke(mf,F,n)}: Computes $T(n)(f)$, where $T(n)$ is the $n$th Hecke
operator. Note that the level which is used is that of the modular form space
\kbd{mf}, not that of $F$ if it is different.

\kbd{mfheckemat(mf,n)}: matrix of $T(n)$ on the space \kbd{mf}.

\kbd{mfinit(NK,\{space=4\})}: create the space of modular forms associated to
$NK=[N,k,\chi]$ or $NK=[N,k]$. Codes for \kbd{space} is $0$, new space,
$1$ cuspidal space, $2$ old space, $3$ space of Eisenstein series,
$4$ full space $M_k$ (default). $NK$ can also be of the form $NK=[N,k,0]$,
where $0$ is a wildcard, in which case it gives the vector of all nonzero
\kbd{mfinit} for each Galois orbit of characters $\chi$.

\kbd{mfisCM(F)}: returns $0$ if $F$ does not have complex multiplication,
and the CM discriminant(s) if it does. Note that in weight $1$ $F$ may have
two CM discriminants.

\kbd{mfisequal(F,G,\{lim=0\})}: Are $F$ and $G$ equal, or at least are their
first \kbd{lim+1} Fourier coefficients equal ?

\kbd{mfkohnenbasis}(mf): \kbd{mf} being a cuspidal space of half-integral
weight and level $4N$ with $N$ squarefree, computes a basis $B$ of the Kohnen
$+$-space as a matrix whose columns are the coefficients of $B$ on the basis
of \kbd{mf}.

\kbd{mfkohnenbijection}(mf): \kbd{mf} being a cuspidal space of
half-integral weight, computes \kbd{[mf2,M,K,shi]}, where \kbd{M} is a matrix
giving a Hecke-module isomorphism from the cuspidal space \kbd{mf2} of weight
$2k-1$ and level $N$ to the Kohnen $+$-space of weight $k$ and level $4N$,
the columns of the matrix \kbd{K} are the coefficients of the Kohnen $+$-space
on the basis of \kbd{mf}, and \kbd{shi} gives technical information about
which linear combination of Shimura lifts has been chosen.

\kbd{mfkohneneigenbasis}(mf,bij): \kbd{mf} being a cuspidal space of
half-integral weight and \kbd{bij} the output of \kbd{mfkohnenbijection(mf)},
computes \kbd{[mf0,BNEW,BEIGEN]}, where \kbd{BNEW} and \kbd{BEIGEN} are two
matrices whose columns are the coefficients of a basis of the Kohnen new space
and of the eigenforms on the basis of \kbd{mf} respectively, and \kbd{mf0} is
the corresponding new space of integral weight $2k-1$.

\kbd{mflinear(vecF,vecL)}: linear combination of the forms in \kbd{vecF}
with coefficients in \kbd{vecL}. Forms must have the same weight and
character, but not necessarily the same level. This function must be used
for simpler operations such as scalar multiplication (\kbd{mflinear([F],[s])}),
addition (\kbd{mflinear([F,G],[1,1])}), and subtraction
(\kbd{mflinear([F,G],[1,-1])}).

\kbd{mfmanin(FS)}: $FS$ being a modular symbol associated to an eigenform,
returns $[[P^+,P^-],[\omega^+,\omega^-,r]]$ where the $P^{\pm}$ are the
even/odd polynomials of special values, the $\omega^{\pm}$ the
corresponding periods, and $r=\Im(\omega^+\overline{\omega^-})/<F,F>$.

\kbd{mfmul(F,G)}: product of the modular forms $F$ and $G$.

\kbd{mfnumcusps(N)}: number of cusps of $\G_0(N)$.

\kbd{mfparams(F)}: returns parameters $[N,k,\chi,P]$ of the modular form $F$,
where $K$ is the polynomial defining the number field containing the
coefficients of $F$ (e.g., $y$ if $F$ is rational), or $[-1,-1,-1,0]$ if
it is not defined. If $F$ is a modular form space, returns $[N,k,\chi,space]$.

\kbd{mfperiodpol(mf,F,\{parity=0\})}: period polynomial of the form $F$,
even/odd period polynomial if \kbd{parity} is $1$ or $-1$.

\kbd{mfperiodpolbasis(k,\{parity=0\})}: basis of period polynomials of weight
$k$ for the full modular group, even/odd ones if \kbd{parity} is $1$ or $-1$.

\kbd{mfpetersson(FS,GS)}: $FS$ and $GS$ being the modular symbols associated
to $F$ and $G$ with \kbd{mfsymbol}, computes the Petersson product of $F$ and
$G$ with the usual normalization $1/[\G:\G_0(N)]$.

\kbd{mfpow(F, n)}: Modular form $F$ to the power $n$.

\kbd{mfsearch([N0,k],V,\{space=4\})}: search for rational modular forms
of weight $k$ and level $N\le N0$ in the specified modular form spaces
whose Fourier expansion up to the length of $V$ exactly matches $V$.
The output is a list of $[N,k,D]$, where $D$ is a fundamental discriminant
dividing $N$.

\kbd{mfshift(F,m)}: \kbd{F} divided by $q^m$, omitting the remainder if there
is one, where $m$ can be positive or negative. The result is usually not
a modular form.

\kbd{mfshimura(mf,F,\{D = 1\})}: $F$ being a modular form of
half-integral weight $k\ge3/2$ and $D$ a fundamental discriminant, return
\kbd{[mf2,FS,v]}, where \kbd{FS} is the corresponding Shimura lift of integral
weight $2k-1$, \kbd{mf2} the corresponding modular form space and \kbd{v} the
coefficients of \kbd{FS} on the basis of \kbd{mf2}. By extension, $D$ can also
be a positive squarefree integer.

\kbd{mfslashexpansion(mf,F,g,n,\{flrat=1\},\{\&A\})}: compute the Fourier
expansion of $F|_kg$ to order $n$, where $F$ is a form in \kbd{mf} and
$g\in M_2^+(\Q)$. If \kbd{flrat} is set (default), try to ``rationalize''
(error if unsuccessful). Optional $A$ contains parameters $[\al,w]$ such that
the output $[a(0),...,a(n)]$ corresponds to the expansion
$q^{\al}\sum_{0\le j\le n}a(j)q^{j/w}$, with $q=\exp(2\pi i\tau)$.

\kbd{mfspace(mf,\{F\})}: type of modular space \kbd{mf} if $F$ is omitted,
or of a modular form $F$ in \kbd{mf}: result is $0$ for new, $1$ for cuspidal,
$2$ for old, $3$ for full, $4$ for Eisenstein, and $-1$ if form is not in
the space.

\kbd{mfsplit(mf,\{dimlim=0\},\{flag=0\})}: compute the eigenforms in \kbd{mf},
and limit the dimension of each Galois orbit to \kbd{dimlim} if set.
\kbd{flag} is used to avoid some long computations (see doc). The space
\kbd{mf} \emph{must} be a new space. Note that the result is only a
two-component vector \kbd{vF,vK}, where \kbd{vF} is a vector of eigenforms
and \kbd{vK} the corresponding number fields, but is \emph{not} similar to
the output of an \kbd{mfinit} command.

\kbd{mfsturm(mf)}: If \kbd{mf} is a space, true Sturm bound of \kbd{mf}, i.e.,
largest valuation at infinity of a nonzero form. If \kbd{mf} is $[N,k,\chi]$,
only an upper bound.

\kbd{mfsymbol(mf,F)}: initialize data for working with integrals related
to $F$ such as \kbd{mfsymboleval}, \kbd{mfpetersson}, and \kbd{mfmanin}.

\kbd{mfsymboleval(FS,path,\{ga\})}: $FS$ being the modular symbol assocated to
some form $F$ and \kbd{path} being $[s1,s2]$ where $s1$ and $s2$ are cusps
or points in the upper half-plane, evaluate the symbol on the path, i.e.,
compute the polynomial
$\int_{s1}^{s2}(X-\tau)^{k-2}F(\tau)\,d\tau$. If $\ga\in GL_2^+(\Q)$ is given,
replace $F$ by $F|_k\ga$. If the integral diverges, the result will be
a rational function.

\kbd{mftaylor(F,n,\{fl=0\})}: for now, only for $F\in M_k(\G)$ and at the
point $i$. Compute the first $n$ Taylor coefficients of $F$ around $i$;
if \kbd{fl} is set compute in fact $p_n$ such that
$$f(\tau)=(2i/(\tau+i))^k\sum_{n>=0}p_n((\tau-i)/(\tau+i))^n\;.$$

\kbd{mfTheta(\{CHI\})}: unary theta series corresponding to the primitive
Dirichlet character \kbd{CHI}, thus in weight $1/2$ (resp., $3/2$)
if \kbd{CHI} is even (resp., odd).

\kbd{mftobasis(mf,F,\{flag=0\})}: coefficients of form $F$ on the basis in
\kbd{mf}. If \kbd{flag} is set, do not return an error if $F$ does not
belong to \kbd{mf} or not enough coefficients.

\kbd{mftocoset(N,M,L)}: $L$ being the list of cosets output by
\kbd{L=mfcosets(N)} and $M$ being in $\SL_2(\Z)$, output a pair
$[\ga,i]$ such that $M=\ga L[i]$, where $\ga\in \G_0(N)$.

\kbd{mftonew(mf,F)}: Decompose $F$ is the cuspidal space \kbd{mf} as
a sum of $B(d)G_M$ where $G_M\in S_k^{\new}(\G_0(M),\chi)$ and $dM\mid N$,
return the vector of $[M,d,G]$.

\kbd{mftraceform(NK,\{space=0\})}: gives the trace form corresponding to
$NK=[N,k,\chi]$ and \kbd{space}.

\kbd{mftwist(F,D)}: twist of the form $F$ by the quadratic character
$(D/n)$.

\end{document}
