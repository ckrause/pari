% Copyright (c) 2007-2016 Karim Belabas.
% Permission is granted to copy, distribute and/or modify this document
% under the terms of the GNU General Public License

% Reference Card for PARI-GP, Algebraic Number Theory.
% Author:
%  Karim Belabas
%  Universite de Bordeaux, 351 avenue de la Liberation, F-33405 Talence
%  email: Karim.Belabas@math.u-bordeaux.fr
%
% See refcard.tex for acknowledgements and thanks.
\def\TITLE{L-functions}
\input refmacro.tex

\section{Characters}
A character on the abelian group $G = \sum_{j \leq k} (\ZZ/d_j\ZZ) \cdot
g_j$, e.g. from \kbd{znstar(q,1)} $\leftrightarrow (\ZZ/q\ZZ)^*$ or
\kbd{bnrinit} $\leftrightarrow \text{Cl}_{\goth{f}}(K)$, is coded by $\chi = [c_1,\dots,c_k]$
such that $\chi(g_j) = e(c_j/d_j)$. Our $L$-functions consider the
attached \emph{primitive} character.

Dirichlet characters $\chi_q(m,\cdot)$ in Conrey labelling system are
alternatively concisely coded by \kbd{Mod(m,q)}. Finally, a quadratic
character $(D/\cdot)$ can also be coded by the integer $D$.

\section{L-function Constructors}
An \kbd{Ldata} is a GP structure describing the functional equation
for $L(s) = \sum_{n\geq 1} a_n n^{-s}$.

\item Dirichlet coefficients given by closure $a: N \mapsto [a_1,\dots,a_N]$.

\item Dirichlet coefficients $a^*(n)$ for dual $L$-function $L^*$.

\item Euler factor $A = [a_1,\dots,a_d]$ for $\gamma_A(s) = \prod_i
\Gamma_{\RR}(s + a_i)$,

\item classical weight $k$ (values at $s$ and $k-s$ are related),

\item conductor $N$, $\Lambda(s) = N^{s/2} \gamma_A(s)$,

\item root number $\varepsilon$; $\Lambda(a,k-s) = \varepsilon \Lambda(a^*,s)$.

\item polar part: list of $[\beta,P_\beta(x)]$.
\medskip

An \kbd{Linit} is a GP structure containing an \kbd{Ldata} $L$ and an
evaluation \emph{domain} fixing a maximal order of derivation $m$ and bit
accuracy (\kbd{realbitprecision}), together with complex ranges

\item for $L$-function: $R=[c,w,h]$
(coding $|\Re z - c| \leq w$, $|\Im z| \leq h$); or $R = [w,h]$ (for $c =
k/2$); or $R = [h]$ (for $c = k/2$, $w = 0$).

\item for $\theta$-function: $T=[\rho,\alpha]$ (for $|t|\geq
\rho$, $|\arg t| \leq \alpha$); or $T = \rho$ (for $\alpha = 0$).

\subsec{Ldata constructors}
\li{Riemann $\zeta$}{lfuncreate$(1)$}
\li{Dirichlet for quadratic char.~$(D/\cdot)$}{lfuncreate$(D)$}
\li{Dirichlet series $L(\chi_q(m,\cdot),s)$}{lfuncreate(Mod(m,N))}
\li{Dedekind $\zeta_K$, $K = \QQ[x]/(T)$}
   {lfuncreate$(\var{bnf})${\rm, }lfuncreate$(T)$}
\li{Hecke for $\chi$ mod $\goth{f}$}{lfuncreate$([\var{bnr},\chi])$}
\li{Artin $L$-function}{lfunartin$(\var{nf},\var{gal},M,n)$}
\li{Lattice $\Theta$-function}{lfunqf$(Q)$}
\li{From eigenform $F$}{lfunmf$(F)$}
\li{Quotients of Dedekind $\eta$: $\prod_i \eta(m_{i,1}\cdot\tau)^{m_{i,2}}$}
   {lfunetaquo$(M)$}
\li{$L(E,s)$, $E$ elliptic curve}{E = ellinit(\dots)}
\li{genus $2$ curve, $y^2+xQ = P$}{lfungenus2$([P,Q])$}
\smallskip

\li{$L_1 \cdot L_2$}{lfunmul$(L_1,L_2)$}
\li{$L_1 / L_2$}{lfundiv$(L_1,L_2)$}
\smallskip

\li{low-level constructor}
   {lfuncreate$([a,a^*,A,k,N,\var{eps},\var{poles}])$}
\li{check functional equation (at $t$)}{lfuncheckfeq$(L,\{t\})$}

\subsec{Linit constructors}
\li{initialize for $L$}{lfuninit$(L, R, \{m = 0\})$}
\li{initialize for $\theta$}{lfunthetainit$(L, \{T = 1\}, \{m = 0\})$}
\li{cost of \kbd{lfuninit}}{lfuncost$(L, R, \{m = 0\})$}
\li{cost of \kbd{lfunthetainit}}{lfunthetacost$(L, T, \{m = 0\})$}

\li{Dedekind $\zeta_L$, $L$ abelian over a subfield}{lfunabelianrelinit}

\section{L-functions}
$L$ is either an \kbd{Ldata} or an \kbd{Linit} (more efficient for many
values).\hfil\break
\subsec{Evaluate}
\li{$L^{(k)}(s)$}{lfun$(L,s,\{k=0\})$}
\li{$\Lambda^{(k)}(s)$}{lfunlambda$(L,s,\{k=0\})$}
\li{$\theta^{(k)}(t)$}{lfuntheta$(L,t,\{k=0\})$}
\li{generalized Hardy $Z$-function at $t$}{lfunhardy$(L,t)$}
%lfunmfspec

\subsec{Zeros}
\li{order of zero at $s = k/2$}{lfunorderzero$(L,\{m=-1\})$}
\li{zeros $s = k/2 + it$, $0 \leq t \leq T$}{lfunzeros$(L, T, \{h\})$}

\subsec{Dirichlet series and functional equation}
\li{$[a_n\colon 1\leq n\leq N]$}{lfunan$(L, N)$}
\li{conductor $N$ of $L$}{lfunconductor$(L)$}
\li{root number and residues}{lfunrootres$(L)$}

\subsec{$G$-functions}
Attached to inverse Mellin transform for $\gamma_A(s)$,
$A = [a_1,\dots,a_d]$.\hfil\break
\li{initialize for $G$ attached to $A$}{gammamellininvinit$(A)$}
\li{$G^{(k)}(t)$}{gammamellininv$(G,t,\{k=0\})$}
\li{asymp. expansion of $G^{(k)}(t)$}
   {gammamellininvasymp$(A,n,\{k=0\})$}
\copyrightnotice
\bye
