% Copyright (c) 2000  The PARI Group
%
% This file is part of the PARI/GP documentation
%
% Permission is granted to copy, distribute and/or modify this document
% under the terms of the GNU General Public License

\catcode`\@=11
%
% GENERAL FORMATTING
%
\newif\ifSUBSECTOC\SUBSECTOCtrue
\newif\ifSUBSECDOT\SUBSECDOTtrue
\newif\ifGPHELP
\newif\ifPDF
\newread\std
\def\checkfile#1{\def\@stdfile{#1}\openin\std=#1\relax}
\long\def\@ifundef#1#2#3{\expandafter\ifx\csname
  #1\endcsname\relax#2\else#3\fi}

% do we come from gphelp ?
\@ifundef{fromgphelp}{\GPHELPfalse}{\GPHELPtrue}
\ifGPHELP %YES
  \overfullrule=0pt
\else     %NO
  \magnification=\magstephalf
  \PDFfalse
  \ifx\pdfoutput\undefined
  \else
     \ifnum\pdfoutput<1
     \else
       \PDFtrue
     \fi
  \fi
\fi

% if paricfg.tex is there (Configure succeeded), input it, otherwise use
% default values
\checkfile{paricfg.tex}
\ifeof\std
  \checkfile{doc/paricfg.tex}
\fi
\ifeof\std
  \ifGPHELP\else% OK for gphelp to use default values
    \message{paricfg.tex not found. You should run Configure.}
  \fi
  \def\vers{2.0.x}
  \def\includedir{/usr/local/include/pari}
  \def\libdir{/usr/local/lib}
\else
  \input\@stdfile
\fi

\font\chaptertitlefont=cmr12 scaled \magstep1
\font\chaptertitlebf=cmbx10 scaled \magstep2
\font\sectiontitlebf=cmbx12
\font\seventt=cmtt8 scaled 875
\scriptfont\ttfam=\seventt % we should really set the \hyphenchar etc first
\parskip=6pt plus 3pt minus 1.5pt
%\overfullrule=0pt

%%
%% TABLE OF CONTENTS
%%
\newwrite\toc
\def\tableofcontents{\begintitle
\openin\std=\jobname.toc
\ifeof\std
\else
  \begingroup
  \centerline{\bf Table of Contents}\medskip
  \parskip=0pt plus 1pt
  \parindent=0pt
  \catcode`\_=11 % make _ an ordinary char (frequent in function names)
  \catcode`\@=11 % make @ an ordinary char (appears in \_ expansion)
  \obeylines\input\jobname.toc
  \endgroup
\fi
\openout\toc=\jobname.toc
\endtitle}

%%
%% CROSS REFERENCING & INDEX
%%
\newif\ifsecondpass
\newwrite\out
\newwrite\aux
\newwrite\index
\ifGPHELP % disable crossreferences
  \def\condwrite#1#2{}
  \def\idx#1{#1}
  \def\toindex#1{}
  \def\tocwrite#1{}
  \def\label#1{}
  \def\gphelpref#1{[Label: {\tt #1}]}
  \def\gphelpsecref#1{Section~\gphelpref{#1}}
  \def\ref{\let\do=\gphelpref\doverb}
  \def\secref{\let\do=\gphelpsecref\doverb}
\else % none of the following is needed by gphelp
  \def\typeout#1{\immediate\write\out{#1}}
  \def\@namedef#1{\expandafter\def\csname#1\endcsname}
  \def\newlabel#1#2{\@ifundef{r@#1}{}{\message{Label `#1' multiply
    defined}}\global\@namedef{r@#1}{#2}}

  \openin\std=\jobname.std
  \ifeof\std
    \secondpassfalse
    \typeout{FIRST PASS}
    \openout\index=\jobname.idx
    \let\condwrite=\write
  \else
    \secondpasstrue
    \typeout{SECOND PASS}
    \let\immediate\relax
    \def\condwrite#1#2{}
  \fi

%default font for index entry
  \def\f@nt{rm}

%% \toindex{#1} = put #1 in index; use font \f@nt, indicate \pageno.
%% If PDF, associate the page to a unique integer (\pdfdestcntr).
  \ifPDF
    \def\toindex#1{\putdest
      \immediate\condwrite\index{!#1!\f@nt!\the\pageno!\number\pdfdestcntr}}
  \else
    \def\toindex#1{%
      \immediate\condwrite\index{!#1!\f@nt!\the\pageno!}}
  \fi

  \checkfile{\jobname.aux}
  \ifeof\std
    \message{No aux file.}
  \else
    \input\@stdfile% input aux file if present
  \fi

% \ref, \label. We need an auxiliary file written during first pass
  \openout\aux=\jobname.aux

  \ifx\inputlineno\undefined
    \let\on@line\empty
  \else
    \def\on@line{ on input line \the\inputlineno}
  \fi
  \def\@errundef#1{\typeout{Reference `#1' on page \the\pageno \space
    undefined\on@line}}

  \def\@car#1#2\@nil{#1}
  \def\@cdr#1#2\@nil{#2}

  \def\@ref {\expandafter\@cdr\@temp \@nil\null}
  \def\@cref{\expandafter\@car\@temp \@nil\null}

  \def\label#1{\immediate\write\aux{\string
    \newlabel{#1}{{\the\chapno}{\currentlabel}}}}
  \def\ref#1{\@ifundef{r@#1}
    {{\bf ??}\@errundef{#1}}
    {\edef\@temp{\csname r@#1\endcsname}%
      \def\lbl{\@ref}\def\chp{\@cref}%
      \ifx\chp{\the\chapno}\lbl\else\chp.\lbl\fi}}
  \def\secref#1{Section~\ref{#1}}
\fi % end of non-gphelp section

%%
%% VERBATIM MODE
%%
% \doverb: setup verbatim mode for the first argument, and execute \do{\arg}
\def\setupverb{\def\do##1{\catcode`##1=12}\dospecials
  \catcode`\ =10% standard space
  \catcode`\f=13% to break ugly ligatures as in nf{}init, use \EFF instead
}
% f won't produce any ligature if catcode 13, e.g in verbatim mode (cf above).
{ \let\GDEF=\gdef \global\let\EFF=f \catcode`\f=13
  \GDEFf{\EFF{}}}

\begingroup
\catcode`<=1\catcode`\{=12
\catcode`>=2\catcode`\}=12
\gdef\d@verb<%
  \def\next{##1}<\gdef\@va<##1>\endgroup%
  \do<\@va>>% \@va is the verbatim argument
  \next>
%
\gdef\d@verbb<%
  \def\next{##1}{##2}<\gdef\@va<##1>\gdef\@vb<##2>\endgroup%
  \do<\@va><\@vb>>% \@vxxx are the verbatim arguments
  \next>
%
\gdef\d@verbbb<%
  \def\next{##1}{##2}{##3}<\gdef\@va<##1>\gdef\@vb<##2>\gdef\@vc<##3>\endgroup%
  \do<\@va><\@vb><\@vc>>% \@vxxx are the verbatim arguments
  \next>
%
\endgroup
\def\doverb {\begingroup\setupverb\d@verb}
\def\doverbb{\begingroup\setupverb\d@verbb}
% make ',' active to allow automatic glue between function arguments, see \fun
\def\doverbbbcomma{\begingroup\catcode`\,=13\setupverb\d@verbbb}
\def\func@mma{,\hskip 0.5em plus 0.6em minus 0.3em}
{\catcode`\,=13\gdef,{\func@mma}}

% argument (silently) goes to index
\def\sidx{\gdef\f@nt{rm}\let\do=\toindex\doverb}   % \rm
\def\kbdsidx{\gdef\f@nt{tt}\let\do=\toindex\doverb}% \tt
\def\varsidx{\gdef\f@nt{it}\let\do=\toindex\doverb}% \tt

% argument printed + sent to index
\def\@idx#1{#1\toindex{#1}}
\def\idx{\gdef\f@nt{rm}\let\do=\@idx\doverb}

% to index + set up as key (keyword)
\def\@keyidx#1{{\bf\@idx{#1}}}
\def\teb{\gdef\f@nt{tt}\let\do=\@keyidx\doverb}

% to index + set up as kbd (verbatim)
\def\@kbdidx#1{{\tt\@idx{#1}}}
\def\tet{\gdef\f@nt{tt}\let\do=\@kbdidx\doverb}

\def\@url#1{\hbox{\tt#1}}
\def\url{\let\do=\@url\doverb}

% to index + set up as var (variable)
\def\@kbdvar#1{{\it\@idx{#1}\/}}
\def\tev{\gdef\f@nt{it}\let\do=\@kbdvar\doverb}

\def\@synt#1#2{\gdef\f@nt{tt}\toindex{#1}
  The library syntax is \key{#1}({\tt #2})}
\def\synt{\let\do=\@synt\doverb}

% function prototypes
\def\funno#1#2#3{\tolerance 1000%
\noindent{\tt#1 #2(#3)}\catcode`\,=12\penalty-100{}} % no index
\def\@fun#1#2#3{\gdef\f@nt{tt}\toindex{#2}\funno{#1}{#2}{#3}}
\def\fun{\let\do=\@fun\doverbbbcomma}

\def\@doc#1#2{\gdef\f@nt{tt}\toindex{#1}\noindent{\tt#2}}
\def\doc{\let\do=\@doc\doverbb}

%%
%% SECTIONS
%%
\newcount\appno
\newcount\chapno
\newcount\secno
\newcount\subsecno
\newcount\subsubsecno

\def\newpage{\hbox{}\vfill\eject}
%Table of contents. cf TeXBook Exercise 21.10
\def\tocwrite#1{{\let\the=0\edef\next{\condwrite\toc{#1}}\next}}
\let\putchapdest\relax
\let\sectionhook\relax

\def\title#1#2{%
  \ifodd\pageno\else\newpage\fi
  \tocwrite{{\bf #1 #2\string\dotfill\the\pageno}}
  \ifGPHELP\else
     \putchapdest
     \centerline{\chaptertitlefont #1}\medskip
  \fi
  \centerline{\let\bf\chaptertitlebf \chaptertitlefont #2}\vskip1cm}
\def\sectitle#1{%
  \ifGPHELP\else%
    \vskip 0pt plus 54pt\penalty-600% good break
    \vskip 24pt plus -45pt minus 9pt\fi%
  \putchapdest%
  \tocwrite{{\hskip0.5cm#1\string\dotfill\the\pageno}}%
  \leftline{\sectionhook{\sectiontitlebf #1}.}
  \penalty10000 % impossible break
  \smallskip}% whatever follows will add a \parskip

\def\subsectitle#1{%
  \ifGPHELP\else%
    \vskip 0pt plus 45pt\penalty-300
    \vskip 6pt plus -42pt minus 3pt\fi%
  \ifSUBSECTOC\tocwrite{{\sevenrm \hskip1cm#1\string\dotfill\the\pageno}}\fi%
  \sectionhook\noindent{\bf#1}.}

\def\subsubsectitle#1{% a \parskip is being added anyway by \noindent
  \ifGPHELP\else%
    \par\vskip 0pt plus 39pt\penalty-200
    \vskip 0pt plus -37pt minus 1.5pt\fi%
  \sectionhook\noindent{\bf#1}.}

\ifGPHELP
  \let\maketitle\relax
\else
  \def\maketitle{%
  \ifnum\chapno=0
    \currentlabel.
  \else
    \number\chapno.\currentlabel\
  \fi}
\fi

\def\misctitle#1{\par\vskip 0pt plus 66pt\penalty-400
  \vskip 3pt plus -64pt minus 1.5pt\noindent{\bf #1.}}

% Chapter headings occupy two lines in the manual (only one in
% INSTALL.tex and gphelp-extracted bits).  Syntax requirement: After calling
% \chapter{...} and possibly \label{...}  etc, an empty line _must_ follow
% before the first paragraph of text or section heading begins. [GN]
\def\chapter#1#2\par{
  \secno=0\global\advance\chapno by 1
  \title{Chapter \number\chapno:}{#1}#2\noindent\ignorespaces}
\def\appendix#1\par{
  \secno=0\global\advance\appno by 1
  \def\applet{\ifcase\appno\or A\or B\or C\or D\or E\or F\or G\fi}
  \title{Appendix \applet:}{#1}\noindent\ignorespaces}
\def\section#1{%
  \subsecno=0\global\advance\secno by 1%
  \gdef\currentlabel{\number\secno}%
  \sectitle{\maketitle#1}}
\def\subsec#1{
  \subsubsecno=0\global\advance\subsecno by 1
  \gdef\currentlabel{\number\secno.\number\subsecno}
  \subsectitle{\maketitle#1}}
\def\subsubsec#1{
  \global\advance\subsubsecno by 1
  \gdef\currentlabel{\number\secno.\number\subsecno.\number\subsubsecno}
  \subsubsectitle{\maketitle#1}}

\def\annotepar#1{\noindent\llap{#1:\ \ }}
\def\emacs{\annotepar{EMACS}}

\def\subseckbd#1{\subsec{\kbd{#1}}}

%
% General purpose
%
\def\begintitle{
  \begingroup\nopagenumbers
  \font\mine=cmb10 scaled 1893
  \hbox{}
}
\def\authors{
  \centerline{The PARI Group}
  \vskip 1.truecm
  \centerline{Institut de Math\'ematiques de Bordeaux, UMR 5251 du CNRS.}
  \centerline{Universit\'e de Bordeaux, 351 Cours de la Lib\'eration}
  \centerline{F-33405 TALENCE Cedex, FRANCE}
  \centerline{\tt e-mail: pari@math.u-bordeaux.fr}
  \vskip 1.5truecm
  \centerline{\sectiontitlebf Home Page:}
  \centerline{\kbd{http://pari.math.u-bordeaux.fr/}}
  \vskip 2.truecm
}

\def\copyrightpage{
\begintitle
\vskip 14cm
\noindent Copyright \copyright\ 2000--2021 The PARI Group
\medskip\par
\noindent Permission is granted to make and distribute verbatim copies of
this manual provided the copyright notice and this permission notice are
preserved on all copies.
\medskip\par
\noindent Permission is granted to copy and distribute modified versions, or
translations, of this manual under the conditions for verbatim copying,
provided also that the entire resulting derived work is distributed under the
terms of a permission notice identical to this one.
\bigskip\par
\noindent PARI/GP is Copyright \copyright\ 2000--2021 The PARI Group
\medskip\par
\noindent PARI/GP is free software;  you can redistribute it and/or modify it
under the terms of the GNU General Public License as published by the Free
Software Foundation. It is distributed in the hope that it will be useful,
but WITHOUT ANY WARRANTY WHATSOEVER.
\endtitle
}

\def\endtitle{\newpage\endgroup}

% fraktur font
\newfam\euffam
\font\teneuf=eufm10
\font\eighteuf=eufm8
\font\seveneuf=eufm7
\textfont\euffam=\teneuf
\scriptfont\euffam=\eighteuf
\scriptscriptfont\euffam=\seveneuf
\def\goth#1{{\fam\euffam#1}}

\def\kbd#1{{\tt #1}}
\def\key#1{{\bf #1}}
\def\emph#1{{\it #1\/}}
\def\var#1{\hbox{\it #1\/}}
\def\floor#1{\left\lfloor #1 \right\rfloor}
\def\ceil#1{\left\lceil #1 \right\rceil}
\def\round#1{\left\lfloor #1 \right\rceil}
\def\fl{\var{f\kern0pt lag}}
\def\Cl{\text{Cl}}
\def\gcd{\text{gcd}}
\def\lcm{\text{lcm}}
\def\Norm{\text{Norm}}
\def\Hom{\text{Hom}}
\def\Frob{\text{Frob}}
\def\Id{\text{Id}}
\def\disc{\text{disc}}
\def\item{$\bullet$~}
%\def\nmid{\not\mid} % Ugly

\def\bs{{\char'134}}
\def\obr{{\char'173}}
\def\cbr{{\char'175}}
\def\pow{\^{}\hskip0pt}
\def\til{\raise-0.3em\hbox{\~{}}}
\def\b#1{{\tt \bs#1}}
\def\mod{\,{\rm mod}\,}
\def\text#1{{\rm#1}}
\def\dfrac#1#2{{{#1}\over{#2}}}
\def\binom#1#2{\pmatrix{{#1}\cr{#2}}}
\def\Bbb#1{{\bf #1}}
{\catcode`\_=11
\gdef\typ#1{\kbd{t_#1}}}
\def\Z{\Bbb Z}
\def\Q{\Bbb Q}
\def\F{\Bbb F}
\def\P{\Bbb P}
\def\R{\Bbb R}
\def\C{\Bbb C}
\def\dotfill{\leaders\hbox to 3truemm{\hfil.\hfil}\hfill}
\def\B{\kbd{BITS\_IN\_LONG}}
\def\op{{\it op\/}}

% verbatim mode
% @ made active: assume verbatim text doesn't contain it
\newif\ifnopar
{\catcode`\^=13\global\let^=\pow\obeyspaces\global\let \ }
\def\ttverb{%
  \nopartrue
  \catcode`\_=12%
  \catcode`\$=12%
  \catcode`\\=12%
  \catcode`\{=12%
  \catcode`\}=12%
  \catcode`\&=12%
  \catcode`\#=12%
  \catcode`\%=12%
  \catcode`\^=13%
  \catcode`\~=13\def~{{\til}}%
  \catcode`\@=0%
  \def\par{\futurelet\next\dopars}%
  \def\dopars{%
    \ifnopar \noparfalse%
    \else% treat two consecutive \par specialy
      \ifx\next\par \vskip4pt plus 1pt\nopartrue%
      \else \leavevmode\endgraf\fi\fi}%
  \obeyspaces\obeylines\tt}

% back to normalcy
\def\unverb{%
  \catcode`\\=0%
  \catcode`\{=1%
  \catcode`\}=2%
  \catcode`\$=3%
  \catcode`\&=4%
  \catcode`\#=6%
  \catcode`\^=7%
  \catcode`\_=8%
  \catcode`\^^I=10}

\def\bprog{\begingroup%
  \vskip 0pt plus 1pt%
  \leavevmode\parskip=0pt plus 1pt%
  \interlinepenalty2000\clubpenalty9000\widowpenalty9000%
  \ttverb}

\def\bprogfile#1{\bprog\input#1\eprog}
\def\eprog{\endgroup\par}
{\obeylines
  \gdef\com{\begingroup\unverb\comstart}
  \gdef\comstart#1^^M{\it#1\endgroup
} % newline after @endgroup is important
}
% comments
\def\Ccom{\begingroup\unverb\Ccomstart}
\def\Ccomstart#1*/{\rm#1\endgroup*/}

\ifPDF
  \input pdfmacs.tex
\fi
\catcode`\@=12
