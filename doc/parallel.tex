\def\TITLE{Introduction to parallel GP programming}
\input parimacro.tex

% START TYPESET
\begintitle
\vskip 2.5truecm
\centerline{\mine Introduction}
\vskip 1.truecm
\centerline{\mine to}
\vskip 1.truecm
\centerline{\mine parallel GP}
\vskip 1.truecm
\centerline{\sectiontitlebf (version \vers)}
\vskip 1.truecm
\authors
\endtitle

\copyrightpage
\tableofcontents
\openin\std=parallel.aux
\ifeof\std
\else
  \input parallel.aux
\fi
\chapno=0

\chapter{Parallel GP interface}

\section{Configuration}

This booklet documents the parallel GP interface. The first chapter describes
configuration and basic concepts, the second one explains how to write GP
codes suitable for parallel execution. Two multithread interfaces
are supported:

\item POSIX threads

\item Message passing interface (MPI)

POSIX threads are well-suited for single systems, for instance a personal
computer equiped with a multi-core processor, while MPI is used by most
clusters. However the parallel GP interface does not depend on the
multithread interface: a properly written GP program will work identically
with both. The interfaces are mutually exclusive and one must be chosen at
configuration time when installing the PARI/GP suite.

\subsec{POSIX threads}

POSIX threads are selected by passing the flag \kbd{--mt=pthread} to
\kbd{Configure}. The required library (\kbd{libpthread}) and header files
(\kbd{pthread.h}) are installed by default on most Linux system. This option
implies \kbd{--enable-tls} which builds a thread-safe version of the PARI
library. This unfortunately makes the dynamically linked \kbd{gp-dyn} binary
about $25\%$ slower; since \kbd{gp-sta} is only $5\%$ slower, you definitely
want to use the latter binary.

You can test whether parallel GP support is working with

\bprog
make test-parallel
@eprog

\subsec{Message Passing Interface}

Configuring MPI is somewhat more difficult, but your MPI installation should
include a script \kbd{mpicc} that takes care of the necessary configuration.
If you have a choice between several MPI implementations, choose OpenMPI.

To configure PARI/GP for MPI, use
\bprog
env CC=mpicc ./Configure --mt=mpi
@eprog

To run a program, you must use the launcher provided by your implementation,
usually \kbd{mpiexec} or \kbd{mpirun}. For instance, to run
the program \kbd{prog.gp} on $10$ nodes, you would use
\bprog
  mpirun -np 10 gp prog.gp
@eprog\noindent (or \kbd{mpiexec} instead of \kbd{mpirun}). PARI
requires at least $3$ MPI nodes to work properly. \emph{Caveats:}
\kbd{mpirun} is not suited for interactive use because it does not provide
tty emulation. Moreover, it is not currently possible to interrupt parallel
tasks.

You can test parallel GP support (here using 3 nodes) with

\bprog
make test-parallel RUNTEST="mpirun -np 3"
@eprog

\section{Concept}

In a GP program, the \emph{main thread} executes instructions sequentially
(one after the other) and GP provides functions, that execute subtasks in
\emph{secondary threads} in parallel (at the same time). Those functions all
share the prefix \kbd{par}, e.g., \kbd{parfor}, a parallel version of the
sequential \kbd{for}-loop construct.

The subtasks are subject to a stringent limitation, the parallel code
must be free of side effects: it cannot set or modify a global variable
for instance. In fact, it may not even \emph{access} global variables or
local variables declared with \kbd{local()}.

Due to the overhead of parallelism, we recommend to split the computation so
that each parallel computation requires at least a few seconds. On the other
hand, it is generally more efficient to split the computation in small chunks
rather than large chunks.

\subsec{Resources}

The number of secondary threads to use is controlled by
\kbd{default(nbthreads)}. The default value of nbthreads depends on the
mutithread interface.

\item POSIX threads: the number of CPU threads, i.e., the number of CPU cores
multiplied by the hyperthreading factor. The default can be freely modified.

\item MPI: the number of available process slots minus $1$ (one slot is used by
the master thread), as configured with \kbd{mpirun} (or \kbd{mpiexec}). E.g.,
\kbd{nbthreads} is $9$ after \kbd{mpirun -np 10 gp}.
It is possible to change the default to a lower value, but increasing it will
not work: MPI does not allow to spawn new threads at run time. PARI requires
at least $3$ MPI nodes to work properly.

The PARI stack size in secondary threads is controlled by
\kbd{default(threadsize)}, so the total memory allocated is equal to
$\kbd{parisize}+\kbd{nbthreads}\times\kbd{threadsize}$.  By default,
$\kbd{threadsize}=\kbd{parisize}$. Setting the \kbd{threadsizemax} default
allows \kbd{threadsize} to grow as needed up to that limit, analogously to
the behaviour of \kbd{parisize} / \kbd{parisizemax}. We strongly recommend
to set this parameter since it is very hard to control in advance the amount
of memory threads will require: a too small stack size will result in a
stack overflow, aborting the computation, and a too large value is
very wasteful (since the extra reserved but unneeded memory is multiplied by
the number of threads).

\subsec{Timing parallel GP scripts}

There are different notions of timers for parallel programs.
The PARI timers behavior depends on the mutithread interface:

\item POSIX threads: \kbd{getabstime} returns the total CPU time used
by all threads, and \kbd{getwalltime} returns the real time elapsed.
The GP timer will reports both.

Using $8$ POSIX threads:

\bprog
? polmodular(101);
? ##
  ***   last result: cpu time 5,785 ms, real time 878 ms.
@eprog

The CPU time is larger than the real time because threads CPU time is counted.

\item MPI: \kbd{getabstime} returns the CPU time use by the main thread,
and \kbd{getwalltime} returns the real time elapsed.
The GP timer will reports both.

Using $8$ MPI threads:

\bprog
? polmodular(101);
? ##
  ***   last result: cpu time 916 ms, real time 939 ms.
@eprog

The CPU time is less than the real time because the main thread waits for
the secondary threads to complete.

\subsec{GP functions}

GP provides the following functions for parallel operations, please
see the documentation of each function for details:

\item \tet{parvector}: parallel version of \kbd{vector};

\item \tet{parapply}:  parallel version of \kbd{apply};

\item \tet{parsum}:    parallel version of \kbd{sum};

\item \tet{parselect}:    parallel version of \kbd{select};

\item \tet{pareval}:   evaluate a vector of closures in parallel;

\item \tet{parfor}:    parallel version of \kbd{for};

\item \tet{parforprime}:   parallel version of \kbd{forprime};

\item \tet{parforvec}: parallel version of \kbd{forvec};

\item \tet{parploth}: parallel version of \kbd{ploth}.

\subsec{PARI functions}
The low-level \kbd{libpari} interface for parallelism is documented
in the \emph{Developer's guide to the PARI library}.
\newpage

\chapter{Writing code suitable for parallel execution}

\section{Exporting global variables}

When parallel execution encounters a global variable, say $V$,
an error such as the following is reported:
\bprog
  *** parapply: mt: please use export(V)
@eprog\noindent A global variable is not visible in the parallel execution
unless it is explicitly exported. This may occur in a number of contexts.

\subsec{Example 1: data}

\bprog
? V = [2^256 + 1, 2^193 - 1];
? parvector(#V, i, factor(V[i]))
  *** parvector: mt: please use export(V).
@eprog\noindent The problem is fixed as follows:
\bprog
? V = [2^256 + 1, 2^193 - 1];
? export(V)
? parvector(#V, i, factor(V[i]))
@eprog\noindent The following short form is also available, with a different
semantic:
\bprog
? export(V = [2^256 + 1, 2^193 - 1]);
? parvector(#V, i, factor(V[i]))
@eprog\noindent In the latter case the variable $V$ does not exist in the
main thread, only in parallel threads.

\subsec{Example 2: polynomial variable}

\bprog
? f(n) = bnfinit(x^n-2).no;
? parapply(f, [1..50])
  *** parapply: mt: please use export(x).
@eprog\noindent You may fix this as in the first example using \kbd{export}
but here there is a more natural solution: use the polynomial indeterminate
\kbd{'x} instead the global variable \kbd{x} (whose value is \kbd{'x} on
startup, but may or may no longer be \kbd{'x} at this point):

\bprog
? f(n) = bnfinit('x^n-2).no;
@eprog\noindent or alternatively
\bprog
? f(n) = my(x='x); bnfinit(x^n-2).no;
@eprog
which is more readable if the same polynomial variable is used several times
in the function.

\subsec{Example 3: function}

\bprog
? f(a) = bnfinit('x^8-a).no;
? g(a,b) = parsum(i = a, b, f(i));
? g(37,48)
  *** parsum: mt: please use export(f).
? export(f)
? g(37,48)
%4 = 81
@eprog\noindent Note that \kbd{export}$(v)$ freezes the value of $v$
for parallel execution at the time of the export: you may certainly modify
its value later in the main thread but you need to re-export $v$ if you want
the new value to be used in parallel threads. You may export more than one
variable at once, e.g., \kbd{export}$(a,b,c)$ is accepted. You may also
export \emph{all} variables with dynamic scope (all global variables
and all variables declared with \kbd{local}) using \kbd{exportall()}.
Although convenient, this may be wasteful if most variables are not meant to
be used from parallel threads. We recommend to

\item use \kbd{exportall} in the \kbd{gp} interpreter interactively, while
  developping code;

\item \kbd{export} a function meant to be called from parallel
  threads, just after its definition;

\item use $v = \var{value}$; \kbd{export}$(v)$ when the value
  is needed both in the main thread and in secondary threads;

\item use \kbd{export}($v = \var{value}$) when the value is
  not needed in the main thread.

\noindent In the two latter forms, $v$ should be considered read-only. It is
actually read-only in secondary threads, trying to change it will raise an
exception:
\bprog
  ***   mt: attempt to change exported variable 'v'.
@eprog\noindent You \emph{can} modify it in the main thread, but it must be
exported again so that the new value is accessible to secondary threads:
barring a new \kbd{export}, secondary threads continue to access the old value.

\section{Input and output} If your parallel code needs to write data to
files, split the output in as many files as the number of parallel
computations, to avoid concurrent writes to the same file, with a high risk
of data corruption. For example a parallel version of
\bprog
? f(a) = write("bnf",bnfinit('x^8-a));
? for (a = 37, 48, f(a))
@eprog\noindent could be
\bprog
? f(a) = write(Str("bnf-",a), bnfinit('x^8-a).no);
? export(f);
? parfor(i = 37, 48, f(i))
@eprog\noindent which creates the files \kbd{bnf-37} to \kbd{bnf-48}. Of
course you may want to group these file in a subdirectory, which must be
created first.

\section{Using \kbd{parfor} and \kbd{parforprime}}
\kbd{parfor} and \kbd{parforprime} are the most powerful of all parallel GP
functions but, since they have a different interface than \kbd{for} or
\kbd{forprime}, sequential code needs to be adapted. Consider the example
\bprog
for(i = a, b,
  my(c = f(i));
  g(i,c));
@eprog\noindent

where \kbd{f} is a function without side effects.  This can be run in parallel
as follows:
\bprog
parfor(i = a, b,
  f(i),
  c,     /*@Ccom the value of \kbd{f(i)} is assigned to \kbd{c} */
  g(i,c));
@eprog\noindent For each $i$, $a \leq i \leq b$, in random order,
this construction assigns \kbd{f(i)} to (lexically scoped, as per~\kbd{my})
variable $c$, then calls \kbd{g}$(i,c)$. Only the function \kbd{f} is
evaluated in parallel (in secondary threads), the function \kbd{g} is
evaluated sequentially (in the main thread). Writing \kbd{c = f(i)} in the
parallel section of the code would not work since the main thread would then
know nothing about $c$ or its content. The only data sent from the main
thread to secondary threads are $f$ and the index $i$, and only $c$ (which
is equal to $f(i)$) is returned from the secondary thread to the main thread.

The following function finds the index of the first component of a vector
$V$ satisfying a predicate, and $0$ if none satisfies it:
\bprog
parfirst(pred, V) =
{
  parfor(i = 1, #V,
    pred(V[i]),
    cond,
    if (cond, return(i)));
  return(0);
}
@eprog\noindent This works because, if the second expression
in \kbd{parfor} exits the loop via \kbd{break} / \kbd{return} at index~$i$,
it is guaranteed that all indexes $< i$ are also evaluated and the one with
smallest index is the one that triggers the exit. See \kbd{??parfor} for
details.

The following function is similar to \kbd{parsum}:
\bprog
myparsum(a, b, expr) =
{ my(s = 0);

  parfor(i = a, b,
    expr(i),
    val,
    s += val);
  return(s);
}
@eprog

\section{Sizing parallel tasks}

Dispatching tasks to parallel threads takes time. To limit overhead, split
the computation so that each parallel task requires at least a few
seconds. Consider the following sequential example:
\bprog
  ? thuemorse(n) = (-1)^n * hammingweight(n);
  ? s = 0; for(n = 1, 2*10^6, s += thuemorse(n) / n * 1.)
  cpu time = 1,064 ms, real time = 1,064 ms.
@eprog\noindent
It is natural to try
\bprog
  ? export(thuemorse);
  ? s = 0; parfor(n = 1, 2*10^6, thuemorse(n) / n * 1., S, s += S)
  cpu time = 5,637 ms, real time = 3,216 ms.
@eprog\noindent
However, due to the overhead, this is not faster than the sequential
version; in fact it is much \emph{slower}. To limit overhead, we group
the summation by blocks:
\bprog
  ? s = 0; parfor(N = 1, 20, sum(n = 1+(N-1)*10^5, N*10^5,
                                 thuemorse(n) / n*1.), S, s += S)
  ? cpu time = 1,997 ms, real time = 186 ms.
@eprog\noindent
Try to create at least as many groups as the number of available threads,
to take full advantage of parallelism. Since some of the floating point
additions are done in random order (the ones in a given block occur
successively, in deterministic order), it is possible that some of the
results will differ slightly from one run to the next.

Of course, in this example, \kbd{parsum} is much easier to use since it will
group the summations for you behind the scenes:
\bprog
  ? parsum(n = 1, 2*10^6, thuemorse(n) / n * 1.)
  cpu time = 1,905 ms, real time = 177 ms.
@eprog

\section{Load balancing}

If the parallel tasks require varying time to complete, it is preferable to
perform the slower ones first, when there are more tasks than available
parallel threads. Instead of
\bprog
  parvector(36, i, bnfinit('x^i - 2).no)
@eprog\noindent doing
\bprog
  parvector(36, i, bnfinit('x^(37-i) - 2).no)
@eprog\noindent will be faster if you have fewer than $36$ threads.
Indeed, \kbd{parvector} schedules tasks by increasing $i$ values, and the
computation time increases steeply with $i$. With $18$ threads, say:

\item in the first form, thread~1 handles both $i = 1$ and $i = 19$,
  while thread~18 will likely handle $i = 18$ and $i = 36$. In fact, it is
  likely that the first batch of tasks $i\leq 18$ runs relatively quickly,
  but that none of the threads  handling a value $i > 18$ (second task) will
  have time to complete before $i = 18$. When that thread finishes
  $i = 18$, it will pick the remaining task $i = 36$.

\item in the second form, thread~1 will likely handle
only $i = 36$: tasks $i = 36, 35, \dots 19$ go to the available
18 threads, and $i = 36$ is likely to finish last, when
$i = 18,\dots, 2$ are already assigned to the other 17 threads.
Since the small values of $i$ will finish almost instantly, $i = 1$ will have
been allocated before the initial thread handling $i = 36$ becomes ready again.

Load distribution is clearly more favourable in the second form.

\vfill\eject
\input index\end
