\def\TITLE{Developer's Guide to the PARI library}
\input parimacro.tex

% START TYPESET
\begintitle
\vskip 2.5truecm
\centerline{\mine Developer's Guide}
\vskip 1.truecm
\centerline{\mine to}
\vskip 1.truecm
\centerline{\mine the PARI library}
\vskip 1.truecm
\centerline{\sectiontitlebf (version \vers)}
\vskip 1.truecm
\authors
\endtitle

\copyrightpage
\tableofcontents
\openin\std=develop.aux
\ifeof\std
\else
  \input develop.aux
\fi
\chapno=0

\chapter{Work in progress}

This draft documents private internal functions and structures for hard-core
PARI developers. Anything in here is liable to change on short notice. Don't
use anything in the present document, unless you are implementing new
features for the PARI library. Try to fix the interfaces before using them,
or document them in a better way.
If you find an undocumented hack somewhere, add it here.

Hopefully, this will eventually document everything that we buried in
\kbd{paripriv.h} or even more private header files like \kbd{anal.h}.
Possibly, even implementation choices! Way to go.

\section{The type \typ{CLOSURE}}\kbdsidx{t_CLOSURE}\sidx{closure}
This type holds closures and functions in compiled form, so is deeply
linked to the internals of the GP compiler and evaluator.
The length of this type can be $6$, $7$ or $8$ depending whether the
object is an ``inline closure'', a ``function'' or a ``true closure''.

A function is a regular GP function. The GP input line is treated as a
function of arity $0$.

A true closure is a GP function defined in a non-empty lexical context.

An inline closure is a closure that appears in the code without
the preceding \kbd{->} token. They are generally attached to the prototype
code 'E' and 'I'. Inline closures can only exist as data of other closures,
see below.

In the following example,
\bprog
f(a=Euler)=x->sin(x+a);
g=f(Pi/2);
plot(x=0,2*Pi,g(x))
@eprog\noindent
\kbd{f} is a function, \kbd{g} is a true closure and both \kbd{Euler} and
\kbd{g(x)} are inline closures.

This type has a second codeword \kbd{z[1]}, which is the arity of the
function or closure. This is zero for inline closures. To access it, use

\fun{long}{closure_arity}{GEN C}

\item \kbd{z[2]} points to a \typ{STR} which holds the opcodes. To access it, use

\fun{GEN}{closure_get_code}{GEN C}.

\fun{const char *}{closure_codestr}{GEN C} returns as an array of \kbd{char}
starting at $1$.

\item \kbd{z[3]} points to a \typ{VECSMALL} which holds the operands of the opcodes.
To access it, use

\fun{GEN}{closure_get_oper}{GEN C}

\item \kbd{z[4]} points to a \typ{VEC} which hold the data referenced by the
\kbd{pushgen} opcodes, which can be \typ{CLOSURE}, and in particular
inline closures. To access it, use

\fun{GEN}{closure_get_data}{GEN C}

\item \kbd{z[5]} points to a \typ{VEC} which hold extra data needed for
error-reporting and debugging. See \secref{se:dbgclosure} for details.
To access it, use

\fun{GEN}{closure_get_dbg}{GEN C}

Additionally, for functions and true closures,

\item \kbd{z[6]} usually points to a \typ{VEC} with two components which are \typ{STR}.
The first one displays the list of arguments of the closure without the
enclosing parentheses, the second one the GP code of the function at the
right of the \kbd{->} token. They are used to display the closure, either in
implicit or explicit form. However for closures that were not generated from GP
code, \kbd{z[6]} can point to a \typ{STR} instead. To access it, use

\fun{GEN}{closure_get_text}{GEN C}

Additionally, for true closure,

\item \kbd{z[7]} points to a \typ{VEC} which holds the values of all lexical
variables defined in the scope the closure was defined. To access it, use

\fun{GEN}{closure_get_frame}{GEN C}

\subsec{Debugging information in closure}\label{se:dbgclosure}

Every \typ{CLOSURE} object \kbd{z} has a component \kbd{dbg=z[5]}
which hold extra data needed for error-reporting and debugging.
The object \kbd{dbg} is a \typ{VEC} with $3$ components:

\kbd{dbg[1]} is a \typ{VECSMALL} of the same length than \kbd{z[3]}. For each
opcode, it holds the position of the corresponding GP source code in the
strings stored in \kbd{z[6]} for function or true closures, positive indices
referring to the second strings, and negative indices referring to the first
strings, the last element being indexed as $-1$. For inline closures, the
string of the parent function or true closure is used instead.

\kbd{dbg[2]} is a \typ{VECSMALL} that lists opcodes index where new lexical
local variables are created. The value $0$ denotes the position before the
first offset and variables created by the prototype code 'V'.

\kbd{dbg[3]} is a \typ{VEC} of \typ{VECSMALL}s that give the list of
\kbd{entree*} of the lexical local variables created at a given index in
\kbd{dbg[2]}.

\section{The type \typ{LIST}}\kbdsidx{t_LIST}\sidx{list} This type needs to go
through various hoops to support GP's inconvenient memory model. Don't
use \typ{LIST}s in pure library mode, reimplement ordinary lists! This
dynamic type is implemented by a \kbd{GEN} of length 3: two codewords and a
vector containing the actual entries. In a normal setup (a finished list,
ready to be used),

\item the vector is malloc'ed, so that it can be realloc'ated without moving
the parent \kbd{GEN}.

\item all the entries are clones, possibly with cloned subcomponents; they
must be deleted with \tet{gunclone_deep}, not \tet{gunclone}.

The following macros are proper lvalues and access the components

\fun{long}{list_nmax}{GEN L}: current maximal number of elements. This grows
as needed.

\fun{GEN}{list_data}{GEN L}: the elements. If \kbd{v = list\_data(L)}, then
either \kbd{v} is \kbd{NULL} (empty list) or \kbd{l = lg(v)} is defined, and
the elements are \kbd{v[1]}, \dots, \kbd{v[l-1]}.

In most \kbd{gerepile} scenarios, the list components are not inspected
and a shallow copy of the malloc'ed vector is made. The functions
\kbd{gclone}, \kbd{copy\_bin\_canon} are exceptions, and make a full copy of
the list.

The main problem with lists is to avoid memory leaks; in the above setup,
a statement like \kbd{a = List(1)} would already leak memory, since
\kbd{List(1)} allocates memory, which is cloned (second allocation) when
assigned to \kbd{a}; and the original list is lost. The solution we
implemented is

\item to create anonymous lists (from \kbd{List}, \kbd{gtolist},
\kbd{concat} or \kbd{vecsort}) entirely on the stack, \emph{not} as described
above, and to set \kbd{list\_nmax} to $0$. Such a list is not yet proper and
trying to append elements to it fails:
\bprog
? listput(List(),1)
  ***   variable name expected: listput(List(),1)
  ***                                   ^----------------
@eprog\noindent
If we had been malloc'ing memory for the
\kbd{List([1,2,3])}, it would have leaked already.

\item as soon as a list is assigned to a variable (or a component thereof)
by the GP evaluator, the assigned list is converted to the proper format
(with \kbd{list\_nmax} set) previously described.

\fun{GEN}{listcopy}{GEN L} return a full copy of the \typ{LIST}~\kbd{L},
allocated on the \emph{stack} (hence \kbd{list\_nmax} is $0$). Shortcut for
\kbd{gcopy}.

\fun{GEN}{mklistcopy}{GEN x} returns a list with a single element $x$,
allocated on the stack. Used to implement most cases of \kbd{gtolist}
(except vectors and lists).

A typical low-level construct:
\bprog
  long l;
  /* assume L is a t_LIST */
  L = list_data(L); /* discard t_LIST wrapper */
  l = L? lg(L): 1;
  for (i = 1; i < l; i++) output( gel(L, i) );
  for (i = 1; i < l; i++) gel(L, i) = gclone( ... );
@eprog

\subsec{Maps as Lists}

GP's maps are implemented on top of \typ{LIST}s so as to benefit from
their peculiar memory models. Lists thus come in two subtypes: \typ{LIST\_RAW}
(actual lists) and \typ{LIST\_MAP} (a map).

\fun{GEN}{mklist_typ}{long t} create a list of subtype $t$.
\fun{GEN}{mklist}{void} is an alias for
\bprog
  mklist_typ(t_LIST_RAW);
@eprog
and
\fun{GEN}{mkmap}{void} is an alias for
\bprog
  mklist_typ(t_LIST_MAP);
@eprog

\fun{long}{list_typ}{GEN L} return the list subtype, either \typ{LIST\_RAW} or
\typ{LIST\_MAP}.

\fun{void}{listpop}{GEN L, long index} as \kbd{listpop0},
assuming that $L$ is a \typ{LIST\_RAW}.

\fun{GEN}{listput}{GEN list, GEN object, long index} as \kbd{listput0},
assuming that $L$ is a \typ{LIST\_RAW}.

\fun{GEN}{mapdomain}{GEN T} vector of keys of the map $T$.

\fun{GEN}{mapdomain_shallow}{GEN T} shallow version of \kbd{mapdomain}.

\fun{GEN}{maptomat}{GEN T} convert a map to a factorization matrix.

\fun{GEN}{maptomat_shallow}{GEN T} shallow version of \kbd{maptomat}.

\section{Protection of non-interruptible code}

GP allows the user to interrupt a computation by issuing SIGINT
(usually by entering control-C) or SIGALRM (usually using alarm()).
To avoid such interruption to occurs in section of code which are not
reentrant (in particular \kbd{malloc} and \kbd{free})
the following mechanism is provided:

\fun{}{BLOCK_SIGINT_START}{}
  Start a non-interruptible block code. Block both \kbd{SIGINT} and \kbd{SIGARLM}.

\fun{}{BLOCK_SIGALRM_START}{}
  Start a non-interruptible block code. Block only \kbd{SIGARLM}.
This is used in the \kbd{SIGINT} handler itself to delay an eventual pending
alarm.

\fun{}{BLOCK_SIGINT_END}{}
  End a non-interruptible block code

The above macros make use of the following global variables:

\tet{PARI_SIGINT_block}: set to $1$ (resp. $2$) by \kbd{BLOCK\_SIGINT\_START}
(resp. \kbd{BLOCK\_SIGALRM\_START}).

\tet{PARI_SIGINT_pending}: Either $0$ (no signal was blocked), \kbd{SIGINT}
(\kbd{SIGINT} was blocked) or \kbd{SIGALRM} (\kbd{SIGALRM} was blocked).
This need to be set by the signal handler.

Within a block, an automatic variable \kbd{int block} is defined which
records the value of \kbd{PARI\_SIGINT\_block} when entering the block.

\subsec{Multithread interruptions}

To support multithreaded programs, \kbd{BLOCK\_SIGINT\_START} and
\kbd{BLOCK\_SIGALRM\_START} call \kbd{MT\_SIGINT\_BLOCK(block)}, and
\kbd{BLOCK\_SIGINT\_END} calls \kbd{MT\_SIGINT\_UNBLOCK(block)}.

\tet{MT_SIGINT_BLOCK} and \tet{MT_SIGINT_UNBLOCK} are defined by the
multithread engine. They can calls the following public functions defined by
the multithread engine.

\fun{void}{mt_sigint_block}{void}

\fun{void}{mt_sigint_unblock}{void}

In practice this mechanism is used by the POSIX thread engine to protect against
asychronous cancellation.

\section{$\F_{l^2}$ field for small primes $l$}
Let $l>2$ be a prime \kbd{ulong}.  A \kbd{Fl2} is an element of the finite
field $\F_{l^2}$ represented (currently) by a \kbd{Flx} of degree at most $1$
modulo a polynomial of the form $x^2-D$ for some non square $0\leq D<p$.
Below \kbd{pi} denotes the pseudo inverse of \kbd{p}, see \kbd{Fl\_mul\_pre}

\fun{int}{Fl2_equal1}{GEN x} return $1$ if $x=1$, else return $0$.

\fun{GEN}{Fl2_mul_pre}{GEN x, GEN y, ulong D, ulong p, ulong pi} return $x\*y$.

\fun{GEN}{Fl2_sqr_pre}{GEN x, ulong D, ulong p, ulong pi} return $x^2$.

\fun{GEN}{Fl2_inv_pre}{GEN x, ulong D, ulong p, ulong pi} return $x^{-1}$.

\fun{GEN}{Fl2_pow_pre}{GEN x, GEN n, ulong D, ulong p, ulong pi} return
$x^n$.

\fun{GEN}{Fl2_sqrtn_pre}{GEN a, GEN n, ulong D, ulong p, ulong pi, GEN *zeta}
$n$-th root, as \kbd{Flxq\_sqrtn}.

\fun{GEN}{Fl2_norm_pre}{GEN x, GEN n, ulong D, ulong p, ulong pi} return the
norm of $x$.

\fun{GEN}{Flx_Fl2_eval_pre}{GEN P, GEN x, ulong D, ulong p, ulong pi}
return $P(x)$.

\section{Public functions useless outside of GP context}

These functions implement GP functionality for which the C language or
other libpari routines provide a better equivalent; or which are so tied
to the \kbd{gp} interpreter as to be virtually useless in \kbd{libpari}. Some
may be generated by \kbd{gp2c}. We document them here for completeness.

\subsec{Conversions}

\fun{GEN}{toser_i}{GEN x} internal shallow function, used to implement
automatic conversions to power series in GP (as in \kbd{cos(x)}).
Converts a \typ{POL} or a \typ{RFRAC} to a \typ{SER} in the same variable and
precision \kbd{precdl} (the global variable corresponding to
\kbd{seriesprecision}). Returns $x$ itself for a \typ{SER}, and \kbd{NULL}
for other argument types. The fact that it uses a global variable makes it
awkward whenever you're not implementing a new transcendental function in GP.
Use \tet{RgX_to_ser} or \tet{rfrac_to_ser} for a fast clean alternative to
\kbd{gtoser}.

\subsec{Output}

\fun{void}{print0}{GEN g, long flag} internal function underlying the
\kbd{print} GP function. Prints the entries of the \typ{VEC} $g$, one by one,
without any separator; entries of type \typ{STR} are printed without enclosing
quotes. \fl is one of \tet{f_RAW}, \tet{f_PRETTYMAT} or \tet{f_TEX}, using the
current default output context.

\fun{void}{out_print0}{PariOUT *out, const char *sep, GEN g, long flag} as
\tet{print0}, using output context \kbd{out} and separator \kbd{sep} between
successive entries (no separator if \kbd{NULL}).

\fun{void}{printsep}{const char *s, GEN g, long flag} \tet{out_print0} on
\tet{pariOut} followed by a newline.

\fun{void}{printsep1}{const char *s, GEN g, long flag} \tet{out_print0} on
\tet{pariOut}.

\fun{char*}{pari_sprint0}{const char *s, GEN g, long flag} displays $s$,
then \kbd{print0(g, flag)}.

\fun{void}{print}{GEN g} equivalent to \kbd{print0(g, f\_RAW)}, followed
by a \kbd{\bs n} then an \kbd{fflush}.

\fun{void}{print1}{GEN g} as above, without the \kbd{\bs n}. Use
\tet{pari_printf} or \tet{output} instead.

\fun{void}{printtex}{GEN g} equivalent to \kbd{print0(g, t\_TEX)}, followed
by a \kbd{\bs n} then an \kbd{fflush}. Use \tet{GENtoTeXstr} and
\tet{pari_printf} instead.

\fun{void}{write0}{const char *s, GEN g}

\fun{void}{write1}{const char *s, GEN g} use \kbd{fprintf}

\fun{void}{writetex}{const char *s, GEN g} use \tet{GENtoTeXstr} and
\kbd{fprintf}.

\fun{void}{printf0}{GEN fmt, GEN args} use \tet{pari_printf}.

\fun{GEN}{Strprintf}{GEN fmt, GEN args} use \tet{pari_sprintf}.

\subsec{Input}

\kbd{gp}'s input is read from the stream \tet{pari_infile}, which is changed
using

\fun{FILE*}{switchin}{const char *name}

Note that this function is quite complicated, maintaining stacks of files
to allow smooth error recovery and \kbd{gp} interaction. You will be better
off using \tet{gp_read_file}.

\subsec{Control flow statements}

\fun{GEN}{break0}{long n}. Use the C control statement \kbd{break}. Since
\kbd{break(2)} is invalid in C, either rework your code or use \kbd{goto}.

\fun{GEN}{next0}{long n}. Use the C control statement \kbd{continue}. Since
\kbd{continue(2)} is invalid in C, either rework your code or use \kbd{goto}.

\fun{GEN}{return0}{GEN x}. Use \kbd{return}!

\fun{void}{error0}{GEN g}. Use \kbd{pari\_err(e\_USER,)}

\fun{void}{warning0}{GEN g}. Use \kbd{pari\_warn(e\_USER,)}

\subsec{Accessors}

\fun{GEN}{vecslice0}{GEN A, long a, long b} implements $A[a..b]$.

\fun{GEN}{matslice0}{GEN A, long a, long b, long c, long d}
implements $A[a..b, c..d]$.

\subsec{Iterators}

\fun{GEN}{apply0}{GEN f, GEN A} gp wrapper calling \tet{genapply}, where $f$
is a \typ{CLOSURE}, applied to $A$. Use \kbd{genapply} or a standard C loop.

\fun{GEN}{select0}{GEN f, GEN A} gp wrapper calling \tet{genselect}, where $f$
is a \typ{CLOSURE} selecting from $A$. Use \kbd{genselect} or a standard C loop.

\fun{GEN}{vecapply}{void *E, GEN (*f)(void* E, GEN x), GEN x} implements
\kbd{[a(x)|x<-b]}.

\fun{GEN}{veccatapply}{void *E, GEN (*f)(void* E, GEN x), GEN x} implements
\kbd{concat([a(x)|x<-b])} which used to implement \kbd{[a0(x,y)|x<-b;y<-c(b)]}
which is equal to \kbd{concat([[a0(x,y)|y<-c(b)]|x<-b])}.

\fun{GEN}{vecselect}{void *E, long (*f)(void* E, GEN x), GEN A}
implements \kbd{[x<-b,c(x)]}.

\fun{GEN}{vecselapply}{void *Epred, long (*pred)(void* E, GEN x), void *Efun, GEN (*fun)(void* E, GEN x), GEN A}
implements \kbd{[a(x)|x<-b,c(x)]}.

\subsec{Local precision}

Theses functions allow to change \kbd{realprecision} locally when
calling the GP interpretor.

\fun{void}{push_localprec}{long p} set the local precision to $p$.

\fun{void}{push_localbitprec}{long b} set the local precision to $b$ bits.

\fun{void}{pop_localprec}{void} reset the local precision to the previous
value.

\fun{long}{get_localprec}{void} returns the current local precision.

\fun{long}{get_localbitprec}{void} returns the current local precision in bits.

\fun{void}{localprec}{long p} trivial wrapper around \kbd{push\_localprec}
(sanity checks and convert from decimal digits to a number of codewords).
Use \kbd{push\_localprec}.

\fun{void}{localbitprec}{long p}
 trivial wrapper around \kbd{push\_localbitprec}
(sanity checks). Use \kbd{push\_localbitprec}.

\subsec{Functions related to the GP evaluator}

The prototype code \kbd{C} instructs the GP compiler to save the current
lexical context (pairs made of a lexical variable name and its value)
in a \kbd{GEN}, called \kbd{pack} in the sequel. This \kbd{pack} can be used
to evaluate expressions in the corresponding lexical context, providing it is
current.

\fun{GEN}{localvars_read_str}{const char *s, GEN pack} evaluate the string $s$
in the lexical context given by \kbd{pack}.  Used by \tet{geval_gp} in GP
to implement the behaviour below:
\bprog
? my(z=3);eval("z=z^2");z
%1 = 9
@eprog

\fun{long}{localvars_find}{GEN pack, entree *ep} does \kbd{pack} contain
a pair whose variable corresponds to \kbd{ep}? If so, where is the
corresponding value? (returns an offset on the value stack).

\subsec{Miscellaneous}

\fun{char*}{os_getenv}{const char *s} either calls \kbd{getenv}, or directly
return \kbd{NULL} if the \kbd{libc} does not provide it. Use \tet{getenv}.

\fun{sighandler_t}{os_signal}{int sig, pari_sighandler_t fun} after a
\bprog
  typedef void (*pari_sighandler_t)(int);
@eprog\noindent
(private type, not exported). Installs signal handler \kbd{fun} for
signal \kbd{sig}, using \tet{sigaction} with flag \tet{SA_NODEFER}. If
\kbd{sigaction} is not available use \tet{signal}. If even the latter is not
available, just return \tet{SIG_IGN}. Use \tet{sigaction}.

\section{Embedded GP interpretor}
These function provide a simplified interface to embed a GP
interpretor in a program.

\fun{void}{gp_embedded_init}{long rsize, long vsize}
Initialize the GP interpretor (like \kbd{pari\_init} does) with
\kbd{parisize=rsize} \kbd{rsize} and \kbd{parisizemax=vsize}.

\fun{char *}{gp_embedded}{const char *s}
Evaluate the string \kbd{s} with GP and return the result as a string,
in a format similar to what GP displays (with the history index).
The resulting string is allocated on the PARI stack, so subsequent call
to \kbd{gp\_embedded} will destroy it.

\chapter{Regression tests, benches}

This chapter documents how to write an automated test module, say \kbd{fun},
so that \kbd{make test-fun} executes the statements in the \kbd{fun} module
and times them, compares the output to a template, and prints an error
message if they do not match.

\item Pick a \emph{new} name for your test, say \kbd{fun}, and write down a
GP script named \kbd{fun}. Make sure it produces some useful output and tests
adequately a set of routines.

\item The script should not be too long: one minute runs should be enough.
Try to break your script into independent easily reproducible tests, this way
regressions are easier to debug; e.g. include \kbd{setrand(1)} statement before
a randomized computation. The expected output may be different on 32-bit and
64-bit machines but should otherwise be platform-independent. If possible, the
output shouldn't even depend on \kbd{sizeof(long)}; using a \kbd{realprecision}
that exists on both 32-bit and 64-bit architectures, e.g. \kbd{\bs p 38} is a
good first step.

\item Dump your script into \kbd{src/test/in/} and run \kbd{Configure}.

\item \kbd{make test-fun} now runs the new test, producing a \kbd{[BUG]} error
message and a \kbd{.dif} file in the relevant object directory \kbd{Oxxx}.
In fact, we compared the output to a non-existing template, so this must fail.

\item Now
\bprog
  patch -p1 < Oxxx/fun.dif
@eprog\noindent
generates a template output in the right place \kbd{src/test/32/fun}, for
instance on a 32-bit machine.

\item If different output is expected on 32-bit and 64-bit machines, run the
test on a 64-bit machine and patch again, thereby
producing \kbd{src/test/64/fun}. If, on the contrary, the output must be the
same, make sure the output template land in the \kbd{src/test/32/} directory
(which provides a default template when the 64-bit output file is missing);
in particular move the file from \kbd{src/test/64/} to \kbd{src/test/32/}
if the test was run on a 64-bit machine.

\item You can now re-run the test to check for regressions: no \kbd{[BUG]}
is expected this time! Of course you can at any time add some checks, and
iterate the test / patch phases. In particular, each time a bug in the
\kbd{fun} module is fixed, it is a good idea to add a minimal test case to
the test suite.

\item By default, your new test is now included in \kbd{make test-all}. If
it is particularly annoying, e.g. opens tons of graphical windows as
\kbd{make test-ploth} or just much longer than the recommended minute, you
may edit \kbd{config/get\_tests} and add the \kbd{fun} test to the list of
excluded tests, in the \kbd{test\_extra\_out} variable.

\item By default, the test is run on both the \kbd{gp-sta} and \kbd{gp-dyn}
binaries, making it twice as slow. If the test is somewhat long, it can
be annoying; you can restrict to one binary only using the \kbd{dotestSUF}
environment variable (whose value is \kbd{sta} or \kbd{dyn}). This will run
the test (resp.~all tests) only under \kbd{gp-sta}:

\bprog
  env dotestSUF=sta make test-fun
  env dotestSUF=sta make test-all
@eprog\noindent The \kbd{test-all} variant is so common that a shortcut
is provided: \kbd{make statest-all}.

\item Besides \kbd{statest-all}, which runs all the tests, you can run a
subset of existing tests by calling \kbd{dotest} directly from a script:

\bprog
  cd Oxxx     # call from relevant build directory
  env dotestSUF=sta ../src/test/dotest 64 "" lfuntype lfun gamma
@eprog\noindent will run the \kbd{lfuntype}, \kbd{lfun} and \kbd{gamma} tests.
This produces a combined output whereas the simpler

\bprog
  make test-lfuntype test-lfun gamma
@eprog\noindent would not.

\item The \kbd{get\_tests} script also defines the recipe for
\kbd{make bench} timings, via the variable \kbd{test\_basic}. A test is
included as \kbd{fun} or \kbd{fun\_$n$}, where $n$ is an integer $\leq 1000$;
the latter means that the timing is weighted by a factor $n/1000$. (This was
introduced a long time ago, when the \kbd{nfields} bench was so much slower
than the others that it hid slowdowns elsewhere.)

\section{Functions for GP2C}

\subsec{Functions for safe access to components}

Theses function returns the adress of the requested component after checking
it is actually valid. This is used by GP2C -C.

\fun{GEN*}{safegel}{GEN x, long l}, safe version of \kbd{gel(x,l)} for \typ{VEC},
\typ{COL} and \typ{MAT}.

\fun{long*}{safeel}{GEN x, long l}, safe version of \kbd{x[l]} for \typ{VECSMALL}.

\fun{GEN*}{safelistel}{GEN x, long l} safe access to \typ{LIST} component.

\fun{GEN*}{safegcoeff}{GEN x, long a, long b} safe version of
\kbd{gcoeff(x,a, b)} for \typ{MAT}.

\chapter{Parallelism}

\section{The PARI MT interface}

PARI provides an abstraction for doing parallel computations.

\fun{void}{mt_queue_start}{struct pari_mt *pt, GEN worker} Let \kbd{worker}
be a \typ{CLOSURE} object of arity $1$.  Initialize the structure \kbd{pt}
to evaluate \kbd{worker} in parallel.

\fun{void}{mt_queue_submit}{struct pari_mt *pt, long taskid, GEN task} Submit
\kbd{task} to be evaluated by \kbd{worker}, or \kbd{NULL} if no further task
is left to be submitted. The value \kbd{taskid} is user-specified and allows
to later match up results and submitted tasks.

\fun{GEN}{mt_queue_get}{struct pari_mt *pt, long *taskid, long *pending}
Return the result of the evaluation by \kbd{worker} of one of the previously
submitted tasks. Set \kbd{pending} to the number of remaining pending tasks.
Set \kbd{taskid} to the value associate to this task by
\kbd{mt\_queue\_submit}.  Returns \kbd{NULL} if more tasks need to be
submitted.

\fun{void}{mt_queue_end}{struct pari_mt *pt} End the parallel execution.

Calls to \tet{mt_queue_submit} and \tet{mt_queue_get} must alternate: each
call to \tet{mt_queue_submit} must be followed by a call to
\tet{mt_queue_get} before any other call to \tet{mt_queue_submit},
and conversely.

The first call to \tet{mt_queue_get} will return \kbd{NULL} until a
sufficient number of tasks have been submitted. If no more tasks are left
to be submitted, use
\bprog
  mt_queue_submit(handle, id, NULL)
@eprog\noindent
to allow further calls to \tet{mt_queue_get}.  If \tet{mt_queue_get} sets
\kbd{pending} to $0$, then no more tasks are pending and it is safe to call
\tet{mt_queue_end}.

The parameter \kbd{taskid} can be chosen arbitrarily. It is attached to a
task but is not available to \kbd{worker}.  It provides an efficient way to
match a tasks and results. It is ignored when the parameter \kbd{task} is
\kbd{NULL}.

\subsec{Miscellaneous}

\fun{void}{mt_broadcast}{GEN code}: do nothing unless the MPI threading engine
is in use. In that case, it evaluates the closure  \kbd{code} on all secondary
nodes. This can be sued to change the states of the MPI child nodes.
This is used by \tet{install}.

\section{Initialization}

This section is technical.

\fun{void}{pari_mt_init}{void} \label{pari_mt_init}
When using MPI, it is sometimes necessary to run initialization code on the
child nodes after PARI is initialized. This can be done as follow:

\item call \tet{pari_init_opts} with the flag \tet{INIT_noIMTm}.
This initializes PARI, but not the MT engine.

\item call the required initialization code.

\item call \tet{pari_mt_init} to initialize the MT engine.
Note that under MPI, this function only returns on the master node. On the
child nodes, it enters slave mode. Thus it is no longer possible to run
initialization code on the child nodes.

See the file \kbd{examples/pari-mt.c} for an example.

\fun{void}{pari_mt_close}{void} \label{pari_mt_close}
When using MPI, calling \tet{pari_close} will terminate the MPI execution
environment. If this is undesirable, you should call \tet{pari_close_opts} with
the flag \tet{INIT_noIMTm}.  This closes PARI without terminating the MPI
execution environment It is allowed to call \kbd{pari\_mt\_close} later to
terminate it.  Note that the once MPI is terminated it cannot be restarted, and
that it is considered an error for a program to end without having terminated
the MPI execution environment.

\vfill\eject
\input index\end
