\def\TITLE{Developer's Guide to the PARI library}
\input parimacro.tex

% START TYPESET
\begintitle
\vskip 2.5truecm
\centerline{\mine Developer's Guide}
\vskip 1.truecm
\centerline{\mine to}
\vskip 1.truecm
\centerline{\mine the PARI library}
\vskip 1.truecm
\centerline{\sectiontitlebf (version \vers)}
\vskip 1.truecm
\authors
\endtitle

\copyrightpage
\tableofcontents
\openin\std=develop.aux
\ifeof\std
\else
  \input develop.aux
\fi
\chapno=0

\chapter{Work in progress}

This draft documents private internal functions and structures for hard-core
PARI developers. Anything in here is liable to change on short notice. Don't
use anything in the present document, unless you are implementing new
features for the PARI library. Try to fix the interfaces before using them,
or document them in a better way.
If you find an undocumented hack somewhere, add it here.

Hopefully, this will eventually document everything that we buried in
\kbd{paripriv.h} or even more private header files like \kbd{anal.h}.
Possibly, even implementation choices! Way to go.

\section{The type \typ{CLOSURE}}\kbdsidx{t_CLOSURE}\sidx{closure}
This type holds closures and functions in compiled form, so is deeply
linked to the internals of the GP compiler and evaluator.
The length of this type can be $6$, $7$ or $8$ depending whether the
object is an ``inline closure'', a ``function'' or a ``true closure''.

A function is a regular GP function. The GP input line is treated as a
function of arity $0$.

A true closure is a GP function defined in a nonempty lexical context.

An inline closure is a closure that appears in the code without
the preceding \kbd{->} token. They are generally attached to the prototype
code 'E' and 'I'. Inline closures can only exist as data of other closures,
see below.

In the following example,
\bprog
f(a=Euler)=x->sin(x+a);
g=f(Pi/2);
plot(x=0,2*Pi,g(x))
@eprog\noindent
\kbd{f} is a function, \kbd{g} is a true closure and both \kbd{Euler} and
\kbd{g(x)} are inline closures.

This type has a second codeword \kbd{z[1]}, which is the arity of the
function or closure. This is zero for inline closures. To access it, use

\fun{long}{closure_arity}{GEN C}

\item \kbd{z[2]} points to a \typ{STR} which holds the opcodes. To access it, use

\fun{GEN}{closure_get_code}{GEN C}.

\fun{const char *}{closure_codestr}{GEN C} returns as an array of \kbd{char}
starting at $1$.

\item \kbd{z[3]} points to a \typ{VECSMALL} which holds the operands of the opcodes.
To access it, use

\fun{GEN}{closure_get_oper}{GEN C}

\item \kbd{z[4]} points to a \typ{VEC} which hold the data referenced by the
\kbd{pushgen} opcodes, which can be \typ{CLOSURE}, and in particular
inline closures. To access it, use

\fun{GEN}{closure_get_data}{GEN C}

\item \kbd{z[5]} points to a \typ{VEC} which hold extra data needed for
error-reporting and debugging. See \secref{se:dbgclosure} for details.
To access it, use

\fun{GEN}{closure_get_dbg}{GEN C}

Additionally, for functions and true closures,

\item \kbd{z[6]} usually points to a \typ{VEC} with two components which are \typ{STR}.
The first one displays the list of arguments of the closure without the
enclosing parentheses, the second one the GP code of the function at the
right of the \kbd{->} token. They are used to display the closure, either in
implicit or explicit form. However for closures that were not generated from GP
code, \kbd{z[6]} can point to a \typ{STR} instead. To access it, use

\fun{GEN}{closure_get_text}{GEN C}

Additionally, for true closure,

\item \kbd{z[7]} points to a \typ{VEC} which holds the values of all lexical
variables defined in the scope the closure was defined. To access it, use

\fun{GEN}{closure_get_frame}{GEN C}

\subsec{Debugging information in closure}\label{se:dbgclosure}

Every \typ{CLOSURE} object \kbd{z} has a component \kbd{dbg=z[5]}
which hold extra data needed for error-reporting and debugging.
The object \kbd{dbg} is a \typ{VEC} with $3$ components:

\kbd{dbg[1]} is a \typ{VECSMALL} of the same length than \kbd{z[3]}. For each
opcode, it holds the position of the corresponding GP source code in the
strings stored in \kbd{z[6]} for function or true closures, positive indices
referring to the second strings, and negative indices referring to the first
strings, the last element being indexed as $-1$. For inline closures, the
string of the parent function or true closure is used instead.

\kbd{dbg[2]} is a \typ{VECSMALL} that lists opcodes index where new lexical
local variables are created. The value $0$ denotes the position before the
first offset and variables created by the prototype code 'V'.

\kbd{dbg[3]} is a \typ{VEC} of \typ{VECSMALL}s that give the list of
\kbd{entree*} of the lexical local variables created at a given index in
\kbd{dbg[2]}.

\section{The type \typ{LIST}}\kbdsidx{t_LIST}\sidx{list} This type needs to go
through various hoops to support GP's inconvenient memory model. Don't
use \typ{LIST}s in pure library mode, reimplement ordinary lists! This
dynamic type is implemented by a \kbd{GEN} of length 3: two codewords and a
vector containing the actual entries. In a normal setup (a finished list,
ready to be used),

\item the vector is malloc'ed, so that it can be realloc'ated without moving
the parent \kbd{GEN}.

\item all the entries are clones, possibly with cloned subcomponents; they
must be deleted with \tet{gunclone_deep}, not \tet{gunclone}.

The following macros are proper lvalues and access the components

\fun{long}{list_nmax}{GEN L}: current maximal number of elements. This grows
as needed.

\fun{GEN}{list_data}{GEN L}: the elements. If \kbd{v = list\_data(L)}, then
either \kbd{v} is \kbd{NULL} (empty list) or \kbd{l = lg(v)} is defined, and
the elements are \kbd{v[1]}, \dots, \kbd{v[l-1]}.

In most \kbd{gerepile} scenarios, the list components are not inspected
and a shallow copy of the malloc'ed vector is made. The functions
\kbd{gclone}, \kbd{copy\_bin\_canon} are exceptions, and make a full copy of
the list.

The main problem with lists is to avoid memory leaks; in the above setup,
a statement like \kbd{a = List(1)} would already leak memory, since
\kbd{List(1)} allocates memory, which is cloned (second allocation) when
assigned to \kbd{a}; and the original list is lost. The solution we
implemented is

\item to create anonymous lists (from \kbd{List}, \kbd{gtolist},
\kbd{concat} or \kbd{vecsort}) entirely on the stack, \emph{not} as described
above, and to set \kbd{list\_nmax} to $0$. Such a list is not yet proper and
trying to append elements to it fails:
\bprog
? listput(List(),1)
  ***   variable name expected: listput(List(),1)
  ***                                   ^----------------
@eprog\noindent
If we had been malloc'ing memory for the
\kbd{List([1,2,3])}, it would have leaked already.

\item as soon as a list is assigned to a variable (or a component thereof)
by the GP evaluator, the assigned list is converted to the proper format
(with \kbd{list\_nmax} set) previously described.

\fun{GEN}{listcopy}{GEN L} return a full copy of the \typ{LIST}~\kbd{L},
allocated on the \emph{stack} (hence \kbd{list\_nmax} is $0$). Shortcut for
\kbd{gcopy}.

\fun{GEN}{mklistcopy}{GEN x} returns a list with a single element $x$,
allocated on the stack. Used to implement most cases of \kbd{gtolist}
(except vectors and lists).

A typical low-level construct:
\bprog
  long l;
  /* assume L is a t_LIST */
  L = list_data(L); /* discard t_LIST wrapper */
  l = L? lg(L): 1;
  for (i = 1; i < l; i++) output( gel(L, i) );
  for (i = 1; i < l; i++) gel(L, i) = gclone( ... );
@eprog

\subsec{Maps as Lists}

GP's maps are implemented on top of \typ{LIST}s so as to benefit from
their peculiar memory models. Lists thus come in two subtypes: \typ{LIST\_RAW}
(actual lists) and \typ{LIST\_MAP} (a map).

\fun{GEN}{mklist_typ}{long t} create a list of subtype $t$.
\fun{GEN}{mklist}{void} is an alias for
\bprog
  mklist_typ(t_LIST_RAW);
@eprog
and
\fun{GEN}{mkmap}{void} is an alias for
\bprog
  mklist_typ(t_LIST_MAP);
@eprog

\fun{long}{list_typ}{GEN L} return the list subtype, either \typ{LIST\_RAW} or
\typ{LIST\_MAP}.

\fun{void}{listpop}{GEN L, long index} as \kbd{listpop0},
assuming that $L$ is a \typ{LIST\_RAW}.

\fun{GEN}{listput}{GEN list, GEN object, long index} as \kbd{listput0},
assuming that $L$ is a \typ{LIST\_RAW}.

\fun{GEN}{mapdomain}{GEN T} vector of keys of the map $T$.

\fun{GEN}{mapdomain_shallow}{GEN T} shallow version of \kbd{mapdomain}.

\fun{GEN}{maptomat}{GEN T} convert a map to a factorization matrix.

\fun{GEN}{maptomat_shallow}{GEN T} shallow version of \kbd{maptomat}.

\section{Protection of noninterruptible code}

GP allows the user to interrupt a computation by issuing SIGINT
(usually by entering control-C) or SIGALRM (usually using alarm()).
To avoid such interruption to occurs in section of code which are not
reentrant (in particular \kbd{malloc} and \kbd{free})
the following mechanism is provided:

\fun{}{BLOCK_SIGINT_START}{}
  Start a noninterruptible block code. Block both \kbd{SIGINT} and \kbd{SIGARLM}.

\fun{}{BLOCK_SIGALRM_START}{}
  Start a noninterruptible block code. Block only \kbd{SIGARLM}.
This is used in the \kbd{SIGINT} handler itself to delay an eventual pending
alarm.

\fun{}{BLOCK_SIGINT_END}{}
  End a noninterruptible block code

The above macros make use of the following global variables:

\tet{PARI_SIGINT_block}: set to $1$ (resp. $2$) by \kbd{BLOCK\_SIGINT\_START}
(resp. \kbd{BLOCK\_SIGALRM\_START}).

\tet{PARI_SIGINT_pending}: Either $0$ (no signal was blocked), \kbd{SIGINT}
(\kbd{SIGINT} was blocked) or \kbd{SIGALRM} (\kbd{SIGALRM} was blocked).
This need to be set by the signal handler.

Within a block, an automatic variable \kbd{int block} is defined which
records the value of \kbd{PARI\_SIGINT\_block} when entering the block.

\subsec{Multithread interruptions}

To support multithreaded programs, \kbd{BLOCK\_SIGINT\_START} and
\kbd{BLOCK\_SIGALRM\_START} call \kbd{MT\_SIGINT\_BLOCK(block)}, and
\kbd{BLOCK\_SIGINT\_END} calls \kbd{MT\_SIGINT\_UNBLOCK(block)}.

\tet{MT_SIGINT_BLOCK} and \tet{MT_SIGINT_UNBLOCK} are defined by the
multithread engine. They can calls the following public functions defined by
the multithread engine.

\fun{void}{mt_sigint_block}{void}

\fun{void}{mt_sigint_unblock}{void}

In practice this mechanism is used by the POSIX thread engine to protect against
asychronous cancellation.

\section{$\F_{l^2}$ field for small primes $l$}
Let $l>2$ be a prime \kbd{ulong}.  A \kbd{Fl2} is an element of the finite
field $\F_{l^2}$ represented (currently) by a \kbd{Flx} of degree at most $1$
modulo a polynomial of the form $x^2-D$ for some non square $0\leq D<p$.
Below \kbd{pi} denotes the pseudo inverse of \kbd{p}, see \kbd{Fl\_mul\_pre}

\fun{int}{Fl2_equal1}{GEN x} return $1$ if $x=1$, else return $0$.

\fun{GEN}{Fl2_mul_pre}{GEN x, GEN y, ulong D, ulong p, ulong pi} return $x\*y$.

\fun{GEN}{Fl2_sqr_pre}{GEN x, ulong D, ulong p, ulong pi} return $x^2$.

\fun{GEN}{Fl2_inv_pre}{GEN x, ulong D, ulong p, ulong pi} return $x^{-1}$.

\fun{GEN}{Fl2_pow_pre}{GEN x, GEN n, ulong D, ulong p, ulong pi} return
$x^n$.

\fun{GEN}{Fl2_sqrtn_pre}{GEN a, GEN n, ulong D, ulong p, ulong pi, GEN *zeta}
$n$-th root, as \kbd{Flxq\_sqrtn}.

\fun{GEN}{Fl2_norm_pre}{GEN x, GEN n, ulong D, ulong p, ulong pi} return the
norm of $x$.

\fun{GEN}{Flx_Fl2_eval_pre}{GEN P, GEN x, ulong D, ulong p, ulong pi}
return $P(x)$.

\section{Public functions useless outside of GP context}

These functions implement GP functionality for which the C language or
other libpari routines provide a better equivalent; or which are so tied
to the \kbd{gp} interpreter as to be virtually useless in \kbd{libpari}. Some
may be generated by \kbd{gp2c}. We document them here for completeness.

\subsec{Conversions}

\fun{GEN}{toser_i}{GEN x} internal shallow function, used to implement
automatic conversions to power series in GP (as in \kbd{cos(x)}).
Converts a \typ{POL} or a \typ{RFRAC} to a \typ{SER} in the same variable and
precision \kbd{precdl} (the global variable corresponding to
\kbd{seriesprecision}). Returns $x$ itself for a \typ{SER}, and \kbd{NULL}
for other argument types. The fact that it uses a global variable makes it
awkward whenever you're not implementing a new transcendental function in GP.
Use \tet{RgX_to_ser} or \tet{rfrac_to_ser} for a fast clean alternative to
\kbd{gtoser}.

\fun{GEN}{listinit}{GEN x} a \typ{LIST} (from \kbd{List} or \kbd{Map}) may
exist in two different forms due to GP memory model:

\item an ordinary \emph{read-only} copy on the PARI stack (as produced by
\kbd{gtolist} or \kbd{gtomap}) to which one may not assign elements
(\kbd{listput} will fail) unless the list is empty.

\item a feature-complete GP list using (malloc'ed) \kbd{block}s to
allow dynamic insertions. An empty list is automaticaly promoted to this
status on first insertion.

The \kbd{listinit} function creates a copy of existing \typ{SER} $x$ and
makes sure it is of the second kind. Variants of this are automatically
called by \kbd{gp} when assigning a \typ{LIST} to a GP variable; the
mecanism avoid memory leaks when creating a constant list, e.g.
\kbd{List([1,2,3])} (read-only), without assigning it to a variable. Whereas
after \kbd{L = List([1,2,3])} (GP list), we keep a pointer to the object and
may delete it when $L$ goes out of scope.

This \kbd{libpari} function allows \kbd{gp2c} to simulate this process by
generating \kbd{listinit} calls at appropriate places.

\subsec{Output}

\fun{void}{print0}{GEN g, long flag} internal function underlying the
\kbd{print} GP function. Prints the entries of the \typ{VEC} $g$, one by one,
without any separator; entries of type \typ{STR} are printed without enclosing
quotes. \fl is one of \tet{f_RAW}, \tet{f_PRETTYMAT} or \tet{f_TEX}, using the
current default output context.

\fun{void}{out_print0}{PariOUT *out, const char *sep, GEN g, long flag} as
\tet{print0}, using output context \kbd{out} and separator \kbd{sep} between
successive entries (no separator if \kbd{NULL}).

\fun{char*}{pari_sprint0}{const char *s, GEN g, long flag} displays $s$,
then \kbd{print0(g, flag)}.

\subsec{Input}

\kbd{gp}'s input is read from the stream \tet{pari_infile}, which is changed
using

\fun{FILE*}{switchin}{const char *name}

Note that this function is quite complicated, maintaining stacks of files
to allow smooth error recovery and \kbd{gp} interaction. You will be better
off using \tet{gp_read_file}.

\subsec{Control flow statements}

\fun{GEN}{break0}{long n}. Use the C control statement \kbd{break}. Since
\kbd{break(2)} is invalid in C, either rework your code or use \kbd{goto}.

\fun{GEN}{next0}{long n}. Use the C control statement \kbd{continue}. Since
\kbd{continue(2)} is invalid in C, either rework your code or use \kbd{goto}.

\fun{GEN}{return0}{GEN x}. Use \kbd{return}!

\subsec{Accessors}

\fun{GEN}{vecslice0}{GEN A, long a, long b} implements $A[a..b]$.

\fun{GEN}{matslice0}{GEN A, long a, long b, long c, long d}
implements $A[a..b, c..d]$.

\subsec{Iterators}

\fun{GEN}{apply0}{GEN f, GEN A} gp wrapper calling \tet{genapply}, where $f$
is a \typ{CLOSURE}, applied to $A$. Use \kbd{genapply} or a standard C loop.

\fun{GEN}{select0}{GEN f, GEN A} gp wrapper calling \tet{genselect}, where $f$
is a \typ{CLOSURE} selecting from $A$. Use \kbd{genselect} or a standard C loop.

\fun{GEN}{vecapply}{void *E, GEN (*f)(void* E, GEN x), GEN x} implements
\kbd{[a(x)|x<-b]}.

\fun{GEN}{veccatapply}{void *E, GEN (*f)(void* E, GEN x), GEN x} implements
\kbd{concat([a(x)|x<-b])} which used to implement \kbd{[a0(x,y)|x<-b;y<-c(b)]}
which is equal to \kbd{concat([[a0(x,y)|y<-c(b)]|x<-b])}.

\fun{GEN}{vecselect}{void *E, long (*f)(void* E, GEN x), GEN A}
implements \kbd{[x<-b,c(x)]}.

\fun{GEN}{vecselapply}{void *Epred, long (*pred)(void* E, GEN x), void *Efun, GEN (*fun)(void* E, GEN x), GEN A}
implements \kbd{[a(x)|x<-b,c(x)]}.

\subsec{Local precision}

These functions allow to change \kbd{realprecision} locally when
calling the GP interpretor.

\fun{void}{push_localprec}{long p} set the local precision to $p$.

\fun{void}{push_localbitprec}{long b} set the local precision to $b$ bits.

\fun{void}{pop_localprec}{void} reset the local precision to the previous
value.

\fun{long}{get_localprec}{void} returns the current local precision.

\fun{long}{get_localbitprec}{void} returns the current local precision in bits.

\fun{void}{localprec}{long p} trivial wrapper around \kbd{push\_localprec}
(sanity checks and convert from decimal digits to a number of codewords).
Use \kbd{push\_localprec}.

\fun{void}{localbitprec}{long p}
 trivial wrapper around \kbd{push\_localbitprec}
(sanity checks). Use \kbd{push\_localbitprec}.

These two function are used to implement \kbd{getlocalprec} and
\kbd{getlocalbitprec} for the GP interpreter and essentially return their
argument (the current dynamic precision, respectively in bits or as a
\kbd{prec} word count):

\fun{long}{getlocalbitprec}{long bit}

\fun{long}{getlocalprec}{long prec}


\subsec{Functions related to the GP evaluator}

The prototype code \kbd{C} instructs the GP compiler to save the current
lexical context (pairs made of a lexical variable name and its value)
in a \kbd{GEN}, called \kbd{pack} in the sequel. This \kbd{pack} can be used
to evaluate expressions in the corresponding lexical context, providing it is
current.

\fun{GEN}{localvars_read_str}{const char *s, GEN pack} evaluate the string $s$
in the lexical context given by \kbd{pack}.  Used by \tet{geval_gp} in GP
to implement the behavior below:
\bprog
? my(z=3);eval("z=z^2");z
%1 = 9
@eprog

\fun{long}{localvars_find}{GEN pack, entree *ep} does \kbd{pack} contain
a pair whose variable corresponds to \kbd{ep}? If so, where is the
corresponding value? (returns an offset on the value stack).

\subsec{Miscellaneous}

\fun{char*}{os_getenv}{const char *s} either calls \kbd{getenv}, or directly
return \kbd{NULL} if the \kbd{libc} does not provide it. Use \tet{getenv}.

\fun{sighandler_t}{os_signal}{int sig, pari_sighandler_t fun} after a
\bprog
  typedef void (*pari_sighandler_t)(int);
@eprog\noindent
(private type, not exported). Installs signal handler \kbd{fun} for
signal \kbd{sig}, using \tet{sigaction} with flag \tet{SA_NODEFER}. If
\kbd{sigaction} is not available use \tet{signal}. If even the latter is not
available, just return \tet{SIG_IGN}. Use \tet{sigaction}.

\section{Embedded GP interpretor}
These function provide a simplified interface to embed a GP
interpretor in a program.

\fun{void}{gp_embedded_init}{long rsize, long vsize}
Initialize the GP interpretor (like \kbd{pari\_init} does) with
\kbd{parisize=rsize} and \kbd{parisizemax=vsize}.

\fun{char *}{gp_embedded}{const char *s}
Evaluate the string \kbd{s} with GP and return the result as a string,
in a format similar to what GP displays (with the history index).
The resulting string is allocated on the PARI stack, so subsequent call
to \kbd{gp\_embedded} will destroy it.

\section{Readline interface}

Code which wants to use libpari readline (such as the Jupyter notebook)
needs to do the following:
\bprog
#include <readline.h>
#include <paripriv.h>
pari_rl_interface S;
...
pari_use_readline(S);
@eprog\noindent The variable $S$, as initialized above, encapsulates
the libpari readline interface. (And allow us to move gp's readline code
to libpari without introducing a mandatory dependency on readline in
libpari.) The following functions then become available:

\fun{char**}{pari_completion_matches}{pari_rl_interface *pS, const char *s,
long pos, long *wordpos} given a command string $s$, where the cursor
is at index \kbd{pos}, return an array of completion matches.

If \kbd{wordpos} is not \kbd{NULL}, set \kbd{*wordpos} to the index for the
start of the expression we complete.

\fun{char**}{pari_completion}{pari_rl_interface *pS, char *text, int start,
int end} the low-level completer called by \tet{pari_completion_matches}.
The following wrapper
\bprog
char**
gp_completion(char *text, int START, int END)
{ return pari_completion(&S, text, START, END);)
@eprog\noindent is a valid value for
\tet{rl_attempted_completion_function}.

\section{Constructors called by \kbd{pari\_init} functions}

\fun{void}{pari_init_buffers}{}

\fun{void}{pari_init_compiler}{}

\fun{void}{pari_init_defaults}{}

\fun{void}{pari_init_evaluator}{}

\fun{void}{pari_init_files}{}

\fun{void}{pari_init_floats}{}

\fun{void}{pari_init_graphics}{}

\fun{void}{pari_init_homedir}{}

\fun{void}{pari_init_parser}{}

\fun{void}{pari_init_paths}{}

\fun{void}{pari_init_primetab}{}

\fun{void}{pari_init_rand}{}

\section{Destructors called by \kbd{pari\_close}}

\fun{void}{pari_close_compiler}{}

\fun{void}{pari_close_evaluator}{}

\fun{void}{pari_close_files}{}

\fun{void}{pari_close_floats}{}

\fun{void}{pari_close_homedir}{}

\fun{void}{pari_close_mf}{}

\fun{void}{pari_close_parser}{}

\fun{void}{pari_close_paths}{}

\fun{void}{pari_close_primes}{}

\section{Constructors and destructors used by the \kbd{pthreads} interface}

\item Called by \tet{pari_thread_close}

\fun{void}{pari_thread_close_files}{}

\section{Functions for GP2C}

\subsec{Functions for safe access to components}

These functions return the address of the requested component after checking
it is actually valid. This is used by GP2C -C.

\fun{GEN*}{safegel}{GEN x, long l}, safe version of \kbd{gel(x,l)} for \typ{VEC},
\typ{COL} and \typ{MAT}.

\fun{long*}{safeel}{GEN x, long l}, safe version of \kbd{x[l]} for \typ{VECSMALL}.

\fun{GEN*}{safelistel}{GEN x, long l} safe access to \typ{LIST} component.

\fun{GEN*}{safegcoeff}{GEN x, long a, long b} safe version of
\kbd{gcoeff(x,a, b)} for \typ{MAT}.

\newpage

\chapter{Regression tests, benches}

This chapter documents how to write an automated test module, say \kbd{fun},
so that \kbd{make test-fun} executes the statements in the \kbd{fun} module
and times them, compares the output to a template, and prints an error
message if they do not match.

\item Pick a \emph{new} name for your test, say \kbd{fun}, and write down a
GP script named \kbd{fun}. Make sure it produces some useful output and tests
adequately a set of routines.

\item The script should not be too long: one minute runs should be enough.
Try to break your script into independent easily reproducible tests, this way
regressions are easier to debug; e.g. include \kbd{setrand(1)} statement before
a randomized computation. The expected output may be different on 32-bit and
64-bit machines but should otherwise be platform-independent. If possible, the
output shouldn't even depend on \kbd{sizeof(long)}; using a \kbd{realprecision}
that exists on both 32-bit and 64-bit architectures, e.g. \kbd{\bs p 38} is a
good first step. You can use \kbd{sizebyte(0)==16} to detect a 64-bit
architecture and \kbd{sizebyte(0)==8} for 32-bit.

\item Dump your script into \kbd{src/test/in/} and run \kbd{Configure}.

\item \kbd{make test-fun} now runs the new test, producing a \kbd{[BUG]} error
message and a \kbd{.dif} file in the relevant object directory \kbd{Oxxx}.
In fact, we compared the output to a nonexisting template, so this must fail.

\item Now
\bprog
  patch -p1 < Oxxx/fun.dif
@eprog\noindent
generates a template output in the right place \kbd{src/test/32/fun}, for
instance on a 32-bit machine.

\item If different output is expected on 32-bit and 64-bit machines, run the
test on a 64-bit machine and patch again, thereby
producing \kbd{src/test/64/fun}. If, on the contrary, the output must be the
same (preferred behavior!), make sure the output template land in the
\kbd{src/test/32/} directory which provides a default template when the
64-bit output file is missing; in particular move the file from
\kbd{src/test/64/} to \kbd{src/test/32/} if the test was run on a 64-bit
machine.

\item You can now re-run the test to check for regressions: no \kbd{[BUG]}
is expected this time! Of course you can at any time add some checks, and
iterate the test / patch phases. In particular, each time a bug in the
\kbd{fun} module is fixed, it is a good idea to add a minimal test case to
the test suite.

\item By default, your new test is now included in \kbd{make test-all}. If
it is particularly annoying, e.g. opens tons of graphical windows as
\kbd{make test-ploth} or just much longer than the recommended minute, you
may edit \kbd{config/get\_tests} and add the \kbd{fun} test to the list of
excluded tests, in the \kbd{test\_extra\_out} variable.

\item You can run a subset of existing tests by using the following idiom:
\bprog
  cd Oxxx     # call from relevant build directory
  make TESTS="lfuntype lfun gamma" test-all
@eprog\noindent will run the \kbd{lfuntype}, \kbd{lfun} and \kbd{gamma} tests.
This produces a combined output whereas the alternative
\bprog
  make test-lfuntype test-lfun test-gamma
@eprog\noindent would not.

\item By default, the test is run on both the \kbd{gp-sta} and \kbd{gp-dyn}
binaries, making it twice as slow. If the test is somewhat long, it can
be annoying; you can restrict to one binary only using the \kbd{statest-all}
or \kbd{dyntest-all} targets. Both accept the \kbd{TESTS} argument seen
above; again, the alternative

\bprog
  make test-lfuntype test-lfun test-gamma
@eprog\noindent would not.

\item Finally, the \kbd{get\_tests} script also defines the recipe for
\kbd{make bench} timings, via the variable \kbd{test\_basic}. A test is
included as \kbd{fun} or \kbd{fun\_$n$}, where $n$ is an integer $\leq 1000$;
the latter means that the timing is weighted by a factor $n/1000$. (This was
introduced a long time ago, when the \kbd{nfields} bench was so much slower
than the others that it hid slowdowns elsewhere.)

\newpage
\chapter{Parallelism}

PARI provides an abstraction, herafter called the MT engine, for doing
parallel computations. The exact same high level routines are used whether
the underlying communication protocol is POSIX threads or MPI and they behave
differently depending on how \kbd{libpari} was configured, specifically on
\kbd{Configure}'s \kbd{--mt} option. Sequential computation is also supported
(no \kbd{--mt} argument) which is helpful for debugging newly written
parallel code. The final section in this chapter comments a complete example.

\section{The PARI multithread interface}

\fun{void}{mt_queue_start}{struct pari_mt *pt, GEN worker} Let \kbd{worker}
be a \typ{CLOSURE} object of arity $1$.  Initialize the opaque structure
\kbd{pt} to evaluate \kbd{worker} in parallel, using \kbd{nbthreads} threads.
This allocates data in
various ways, e.g., on the PARI stack or as malloc'ed objects: you may not
collect garbage on the PARI stack starting from an earlier \kbd{avma} point
until the parallel computation is over, it could destroy something in \kbd{pt}.
All ressources allocated outside the PARI stack are freed by
\kbd{mt\_queue\_end}.

\fun{void}{mt_queue_start_lim}{struct pari_mt *pt, GEN worker, long lim}
as \kbd{mt\_queue\_start}, where \kbd{lim} is an upper bound on the number
of \kbd{tasks} to perform. Concretely the number of threads is the minimum
of \kbd{lim} and \kbd{nbthreads}. The values $0$ and $1$ of \kbd{lim} are
special:

\item $0$: no limit, equivalent to \kbd{mt\_queue\_start} (use
\kbd{nbthreads} threads).

\item $1$: no parallelism, evaluate the tasks sequentially.

\fun{void}{mt_queue_submit}{struct pari_mt *pt, long taskid, GEN task} submit
\kbd{task} to be evaluated by \kbd{worker}; use \kbd{task = NULL} if no
further task needs to be submitted. The parameter \kbd{taskid} is attached to
the \kbd{task} but not used in any way by the \kbd{worker} or the MT engine,
it will be returned to you by \kbd{mt\_queue\_get} together with the result
for the task, allowing to match up results and submitted tasks if desired.
For instance, if the tasks $(t_1,\dots, t_m)$ are known in advance, stored in
a vector, and you want to recover the evaluation results in the same order as
in that vector, you may use consecutive integers $1, \dots, m$ as
\kbd{taskid}s. If you do not care about the ordering, on the other hand, you
can just use $\kbd{taskid} = 0$ for all tasks.

The \kbd{taskid} parameter is ignored when \kbd{task} is \kbd{NULL}. It is
forbidden to call this function twice without an intervening
\kbd{mt\_queue\_get}.

\fun{GEN}{mt_queue_get}{struct pari_mt *pt, long *taskid, long *pending}
return \kbd{NULL} until \kbd{mt\_queue\_submit} has submitted
tasks for the required number (\kbd{nbthreads}) of threads; then return the
result of the evaluation by \kbd{worker} of one of the previously submitted
tasks, in random order. Set \kbd{pending} to the number of remaining pending
tasks: if this is $0$ then no more tasks are pending and it is safe to call
\tet{mt_queue_end}. Set \kbd{*taskid} to the value attached to this task by
\kbd{mt\_queue\_submit}, unless the \kbd{taskid} pointer is \kbd{NULL}. It is
forbidden to call this function twice without an intervening
\kbd{mt\_queue\_submit}.

\fun{void}{mt_queue_end}{struct pari_mt *pt} end the parallel execution
and free ressources attached to the opaque \kbd{pari\_mt} structure. For
instance malloc'ed data; in the \kbd{pthreads} interface, it would destroy
mutex locks, condition variables, etc. This must be called once there are no
longer pending tasks to avoid leaking ressources; but not before all tasks
have been processed else crashes will occur.

\fun{long}{mt_nbthreads}{void} return the effective number of parallel threads
that would be started by \tet{mt_queue_start} if it has been called in place
of \tet{mt_nbthreads}.

\section{Technical functions required by MPI}

The functions in this section are needed when writing complex independent
programs in order to support the MPI MT engine, as more flexible
complement/variants of \kbd{pari\_init} and \kbd{pari\_close}.

\fun{void}{mt_broadcast}{GEN code}: do nothing unless the MPI threading engine
is in use. In that case, evaluates the closure  \kbd{code} on all secondary
nodes. This can be used to change the state of all MPI child nodes, e.g.,
in \tet{gpinstall} run in the main thread, which allows all nodes to use the
new function.

\fun{void}{pari_mt_init}{void} \label{pari_mt_init}
when using MPI, it is often necessary to run initialization code on the child
nodes after PARI is initialized. This is done by calling successively:

\item \tet{pari_init_opts} with the flag \tet{INIT_noIMTm}:
this initializes PARI, but not the MT engine;

\item the required initialization code;

\item \tet{pari_mt_init} to initialize the MT engine.
Note that under MPI, this function returns on the master node but enters
slave mode on the child nodes. Thus it is no longer possible to run
initialization code on the child nodes.

\fun{void}{pari_mt_close}{void} \label{pari_mt_close}
when using MPI, calling \tet{pari_close} terminates the MPI execution
environment and it will not be possible to restart it. If this is
undesirable, call \tet{pari_close_opts} with the flag \tet{INIT_noIMTm}
instead of \kbd{pari\_close}: this closes PARI without terminating the MPI
execution environment. You may later call \kbd{pari\_mt\_close} to terminate
it. It is an error for a program to end without terminating the MPI execution
environment.

\section{A complete example}

We now proceed to an example exhibiting complex features of this
interface, in particular showing how to generate a valid \kbd{worker}.
Explanations and details follow.

\bprogfile{../examples/pari-mt.c}

We start from some arbitrary C function \kbd{Cworker} and create an
\kbd{entree} summarizing all that GP would need to know about it, in
particular

\item a GP name \kbd{\_worker}; the leading \kbd{\_} is not necessary,
we use it as a namespace mechanism grouping private functions;

\item the name of the C function;

\item and its prototype, see \kbd{install} for an introduction to Prototype
Codes.

\noindent The other three arguments ($0$, $20$ and \kbd{""}) are required in an
\kbd{entree} but not useful in our simple context: they are respectively a
valence ($0$ means ``nothing special''), a help section (20 is customary for
internal functions which need to be exported for technical reasons, see
\kbd{?20}), and a help text (no help).

Then we initialize the MT engine; doing things in this order with a two part
initialization ensures that nodes have access to our \kbd{Cworker}. We
convert the \kbd{ep} data to a \typ{CLOSURE} using \kbd{strtofunction}, which
provides a valid \kbd{worker} to \kbd{mt\_queue\_start}. This creates a
parallel evaluation queue \kbd{mt}, and we proceed to submit all tasks,
recording all results. Results are stored in the right order
by making good use of the \kbd{taskid} label, although we have no control
over \emph{when} each result is returned. We finally free all ressources
attached to the \kbd{mt} structure. If needed, we could have collected all
garbage on the PARI stack using \kbd{gerepilecopy} on the \kbd{out} array and
gone on working instead of quitting.

Note the argument passing convention for \kbd{Cworker}: the task consists of a
single vector containing all arguments as \kbd{GEN}s, which are interpreted
according to the function prototype, here \kbd{GL} so the first argument is
left as is and the second one is converted to a long integer. In more
complicated situations, this second (and possibly further) argument could
provide arbitrary evaluation contexts. In this example, we just used it as a
flag to indicate the kind of evaluation expected on the data: integer
factorization (0) or matrix determinant (1).

Note also that
\bprog
  gel(out, taskid) = mt_queue_get(&mt, &taskid, &pending);
@eprog \noindent instead of our use of a temporary \kbd{done} would have
undefined behaviour (\kbd{taskid} may be uninitialized in the left hand side).
\vfill\eject
\input index\end
