% Copyright (c) 2015  The PARI Group
%
% This file is part of the PARI/GP documentation
%
% Permission is granted to copy, distribute and/or modify this document
% under the terms of the GNU General Public License
\chapter{$L$-functions}

\section{Accessors}

\fun{long}{is_linit}{GEN data}

\fun{GEN}{ldata_get_an}{GEN ldata}

\fun{GEN}{ldata_get_dual}{GEN ldata}

\fun{long}{ldata_isreal}{GEN ldata}

\fun{GEN}{ldata_get_gammavec}{GEN ldata}

\fun{long}{ldata_get_degree}{GEN ldata}

\fun{GEN}{ldata_get_k}{GEN ldata}

\fun{GEN}{ldata_get_k1}{GEN ldata}

\fun{GEN}{ldata_get_conductor}{GEN ldata}

\fun{GEN}{ldata_get_rootno}{GEN ldata}

\fun{GEN}{ldata_get_residue}{GEN ldata}

\fun{long}{ldata_get_type}{GEN ldata}

\fun{long}{linit_get_type}{GEN linit}

\fun{GEN}{linit_get_ldata}{GEN linit}

\fun{GEN}{linit_get_tech}{GEN linit}

\fun{GEN}{lfun_get_domain}{GEN tech}

\fun{GEN}{lfun_get_dom}{GEN tech}

\fun{long}{lfun_get_bitprec}{GEN tech}

\fun{GEN}{lfun_get_factgammavec}{GEN tech}

\fun{GEN}{lfun_get_step}{GEN tech}

\fun{GEN}{lfun_get_pol}{GEN tech}

\fun{GEN}{lfun_get_Residue}{GEN tech}

\fun{GEN}{lfun_get_k2}{GEN tech}

\fun{GEN}{lfun_get_w2}{GEN tech}

\fun{GEN}{lfun_get_expot}{GEN tech}

\fun{long}{lfun_get_bitprec}{GEN tech}

\fun{GEN}{lfunprod_get_fact}{GEN tech}

\fun{GEN}{theta_get_an}{GEN tdata}

\fun{GEN}{theta_get_K}{GEN tdata}

\fun{GEN}{theta_get_R}{GEN tdata}

\fun{long}{theta_get_bitprec}{GEN tdata}

\fun{long}{theta_get_m}{GEN tdata}

\fun{GEN}{theta_get_tdom}{GEN tdata}

\fun{GEN}{theta_get_isqrtN}{GEN tdata}

\section{Conversions and constructors}

\fun{GEN}{lfunmisc_to_ldata}{GEN obj} converts \kbd{obj} to \kbd{Ldata} format.
Exception if \kbd{obj} cannot be converted.

\fun{GEN}{lfunmisc_to_ldata_shallow}{GEN obj} as \tet{lfunmisc_to_ldata},
shallow result. Exception if \kbd{obj} cannot be converted.

\fun{GEN}{lfunmisc_to_ldata_shallow_i}{GEN obj} as
\tet{lfunmisc_to_ldata_shallow_i}, returning \kbd{NULL} on failure.

\fun{GEN}{lfunrtopoles}{GEN r}

\fun{int}{sdomain_isincl}{double k, GEN dom, GEN dom0}

\fun{GEN}{ldata_vecan}{GEN ldata, long N, long prec}
return the vector of coefficients of indices $1$ to $N$ to precision
\kbd{prec}. The output is allowed to be a \typ{VECSMALL} when the
coefficients are known to be all integral and fit into a \kbd{long};
for instance the Dirichlet $L$ function of a real character or the
$L$-function of a rational elliptic curve.

\fun{GEN}{ldata_newprec}{GEN ldata, long prec}
return a shallow copy of \kbd{ldata} with fields accurate to precision
\kbd{prec}.

\fun{long}{etaquotype}{GEN *peta, GEN *pN, GEN *pk, GEN *pCHI, long *pv, long
*psd, long *pcusp} Let \kbd{eta} be the integer matrix factorization supposedly
attached to an $\eta$-quotient $f(z) = \prod_i \eta(n_i z)^{e_i}$. Assuming
\kbd{*peta} is initially set to \kbd{eta}, this function returns $0$ if
there is a type error or this does not define a function on some $X_0(N)$.
Else it returns $1$ and sets

\item \kbd{*peta} to a normalized factorization (as would be returned by
\kbd{factor}),

\item \kbd{*pN} to the level $N$ of $f$,

\item \kbd{*pk} to the modular weight $k$ of $f$,

\item \kbd{*pCHI} to the Nebentypus of $f$ (quadratic character)
as an integer,

\item \kbd{*pv} to the valuation at infinity $v_q(f)$,

\item \kbd{*psd} to $1$ if and only if $f$ is self-dual,

\item \kbd{*pcusp} to $1$ if $f$ is cuspidal, else to $0$ if $f$ holomorphic
at all cusps, else to $-1$.

The last three arguments \kbd{pCHI}, \kbd{pv} and \kbd{pcusp} can be set to
\kbd{NULL}, in which case the relevant information is not computed, which
saves time.

\section{Variants of GP functions}

\fun{GEN}{lfun}{GEN ldata, GEN s, long bitprec}

\fun{GEN}{lfuninit}{GEN ldata, GEN dom, long der, long bitprec}

\fun{GEN}{lfuninit_make}{long t, GEN ldata, GEN tech, GEN domain}

\fun{GEN}{lfunlambda}{GEN ldata, GEN s, long bitprec}

\fun{GEN}{lfunquadneg}{long D, long k} for $L(\chi_D, k)$, $D$ fundamental
discriminant and $k \geq 0$.

\fun{long}{lfunthetacost}{GEN ldata, GEN tdom, long m, long bitprec}:
 \kbd{lfunthetacost0} when the first argument is known to be an \kbd{Ldata}.

\fun{GEN}{lfunthetacheckinit}{GEN data, GEN tinf, long m, long bitprec}

\fun{GEN}{lfunrootno}{GEN data, long bitprec}

\fun{GEN}{lfunzetakinit}{GEN pol, GEN dom, long der, long bitprec}

\fun{GEN}{lfunellmfpeters}{GEN E, long bitprec}

\fun{GEN}{ellanalyticrank}{GEN E, long prec} DEPRECATED.

\fun{GEN}{ellL1}{GEN E, long prec} DEPRECATED.

\section{Inverse Mellin transforms of Gamma products}

\fun{GEN}{gammamellininv}{GEN Vga, GEN s, long m, long bitprec}

\fun{GEN}{gammamellininvinit}{GEN Vga, long m, long bitprec}

\fun{GEN}{gammamellininvrt}{GEN K, GEN s, long bitprec} no GC-clean, but
suitable for \kbd{gerepileupto}.

\fun{int}{Vgaeasytheta}{GEN Vga} return 1 if the inverse Mellin transform is
an exponential and 0 otherwise.

\fun{double}{dbllemma526}{double a, double b, double c, long B}

\fun{double}{dblcoro526}{double a, double c, long B}

\newpage

\chapter{Modular symbols}

\fun{void}{checkms}{GEN W} raise an exception if $W$ is not an \var{ms}
structure from \kbd{msinit}.

\fun{void}{checkmspadic}{GEN W} raise an exception if $W$ is not an
\var{mspadic} structure from \kbd{mspadicinit}.

\fun{GEN}{mseval2_ooQ}{GEN W, GEN phi, GEN c} let $W$ be a \kbd{msinit}
structure for $k = 2$, $\phi$ be a modular symbol with integral values
and $c$ be a rational number. Return the integer $\phi(p)$,
where $p$ is the path $\{\infty,c\}$.

\fun{void}{mspadic_parse_chi}{GEN s, GEN *s1, GEN *s2} see \kbd{mspadicL};
let $\chi$ be the  cyclotomic  character from
$\text{Gal}\big(\Q_p(mu_{p^\infty})/\Q_p\big)$ to $\Z_p^*$
and $\tau$ be the Teichm\"uller character for $p > 2$ and the character of
order $2$ on $(\Z/4\Z)^*$ if $p = 2$. Let $s$ encode
the $p$-adic characther $\chi^s := \langle\chi\rangle^{s_1} \tau^{s_2}$;
set \kbd{*s1} and \kbd{*s2} to the integers $s_1$ and $s_2$.

\fun{GEN}{mspadic_unit_eigenvalue}{GEN ap, long k, GEN p, long n}
let $p$ be a prime not dividing the trace of Frobenius \kbd{ap},
return the unit root of $x^2 - ap*x + p^(k-1)$ to $p$-adic accuracy $p^n$.

Variants of \kbd{mfnumcusps} :

\fun{ulong}{mfnumcuspsu}{ulong n}

\fun{GEN}{mfnumcusps_fact}{GEN fa} where \kbd{fa} is \kbd{factor}$(n)$.

\fun{ulong}{mfnumcuspsu_fact}{GEN fa} where \kbd{fa} is \kbd{factoru}$(n)$.

\chapter{Modular forms}

\section{Implementation of public data structures}

\fun{void}{checkMF}{GEN mf} raise an exception if the argument is not a
modular form space.

\fun{GEN}{checkMF_i}{GEN mf} return the underlying modular form space
if \kbd{mf} is either directly a modular form space from \kbd{mfinit}
or a symbol from \kbd{mfsymbol}. Return \kbd{NULL} otherwise.

\fun{int}{checkmf_i}{GEN mf} return $1$ if the argument is a modular form
and $0$ otherwise.

\fun{int}{checkfarey_i}{GEN F} return $1$ if the argument is a Farey symbol
(from \kbd{mspolygon} or \kbd{msfarey}) and $0$ otherwise.

\subsec{Accessors for modular form spaces}

Shallow functions; assume that their argument is a modular form space
is created by \kbd{mfinit} and checked using \kbd{checkMF}.

\fun{GEN}{MF_get_gN}{GEN mf} return the level $N$ as a \typ{INT}.

\fun{long}{MF_get_N}{GEN mf} return the level $N$ as a \kbd{long}.

\fun{GEN}{MF_get_gk}{GEN mf} return the level $k$ as a \typ{INT}.

\fun{long}{MF_get_k}{GEN mf} return the level $k$ as a \kbd{long}.

\fun{long}{MF_get_r}{GEN mf} assuming the level is a half-integer, return
the integer $r = k - (1/2)$.

\fun{GEN}{MF_get_CHI}{GEN mf} return the nebentypus $\chi$, which is a
 special form of character structure attached to Dirichlet characters (see
 next section). Its values are given as algebraic numbers: either $\pm1$ or
 \typ{POLMOD} in $t$.

\fun{long}{MF_get_space}{GEN mf} returns the space type, corresponding
to \kbd{mfinit}'s \kbd{space} flag. The current list is

\bprog
mf_NEW, mf_CUSP, mf_OLD, mf_EISEN, mf_FULL
@eprog

\fun{GEN}{MF_get_basis}{GEN mf} return the $\Q$-basis of the space,
concatenation of \tet{MF_get_E} and \tet{MF_get_S}, in this order; the forms
have coefficients in $\Q(\chi)$. Low-level version of \kbd{mfbasis}.

\fun{long}{MF_get_dim}{GEN mf} returns the dimension $d$ of the space. It
is the cardinality of \tet{MF_get_basis}.

\fun{GEN}{MF_get_E}{GEN mf} returns a $\Q$-basis for the subspace
spanned by Eisenstein series in the space; the forms have coefficients in
$\Q(\chi)$.

\fun{GEN}{MF_get_S}{GEN mf} returns a $\Q$-basis for the cuspidal subspace
in the space; the forms have coefficients in $\Q(\chi)$.

\fun{GEN}{MF_get_fields}{GEN mf} returns the vector of polynomials defining
each Galois orbit of newforms over $\Q(\chi)$. Uses memoization: a first call
splits the space and may be costly; subsequent calls return the cached
result.

\fun{GEN}{MF_get_newforms}{GEN mf} returns a vector \kbd{vF} containing the
coordinates of the eigenforms on \tet{MF_get_basis} (\kbd{mftobasis} form).
Low-level version of
\kbd{mfeigenbasis}, whose elements are recovered as \kbd{mflinear(mf,
gel(vF,i))}. Uses memoization, sharing the same data as
\kbd{MF\_get\_fields}. Note that it is much more efficient to use
\kbd{mfcoefs(mf,)} then multiply by this vector than to compute the
coefficients of eigenforms from \kbd{mfeigenbasis} individually.

The following accessors are technical,

\fun{GEN}{MF_get_M}{GEN mf} the $(1+m) \times d$ matrix whose $j$-th column
contain the coefficients of the $j$-th entry in \tet{MF_get_basis}, $m$
is the optimal ``Sturm bound'' for the space: the maximum of the $v_\infty(f)$
over nonzero forms. It has entries in $\Q(\chi)$.

\fun{GEN}{MF_get_Mindex}{GEN mf} is a \typ{VECSMALL} containing $d$ row
indices, the corresponding rows of $M$ form an invertible matrix $M_0$.

\fun{GEN}{MF_get_Minv}{GEN mf} the inverse of $M_0$ in a form suitable for fast
multiplication.

\fun{GEN}{MFcusp_get_vMjd}{GEN mf} valid only for a full
\emph{cuspidal} space. Then the functions in \tet{MF_get_S} are
of the form $B_d T_j Tr^{new}_M$. This returns the vector of triples
(\typ{VECSMALL}) $[M,j,d]$, in the same order.

\fun{GEN}{MFnew_get_vj}{GEN mf} valid only for a \emph{new} space. Then
the functions in \tet{MF_get_S} are of the form $T_j Tr^{new}_N$. This
returns a \typ{VECSMALL} of the Hecke indices $j$, in the same order.

\subsec{Accessors for individual modular forms}

\fun{GEN}{mf_get_gN}{GEN F} return the level of $F$, which may be a multiple of
the conductor, as a \typ{INT}

\fun{long}{mf_get_N}{GEN F} return the level as a \kbd{long}.

\fun{GEN}{mf_get_gk}{GEN F} return the weight of $F$ as a \typ{INT}
or a \typ{FRAC} with denominator $2$ (half-integral weight).

\fun{long}{mf_get_k}{GEN F} return the weight as a \kbd{long}; if the weight is
not integral, this raises an exception.

\fun{long}{mf_get_r}{GEN F} assuming $F$ is a modular form of half-integral
weight $k = (2r+1)/2$, return $r = k - (1/2)$.

\fun{GEN}{mf_get_CHI}{GEN F} return the nebentypus, which is a special form of
character structure attached to Dirichlet characters (see next section). Its
values are given as algebraic numbers: either $\pm1$ or \typ{POLMOD}
in $t$.

\fun{GEN}{mf_get_field}{GEN F} return the polynomial (in variable $y$)
 defining $\Q(f)$ over $\Q(\chi)$.

\fun{GEN}{mf_get_NK}{GEN F} return the tag attached to $F$: a vector containing
\kbd{gN}, \kbd{gk}, \kbd{CHI}, \kbd{field}. Never use its component directly,
use individual accessors as above.

\fun{long}{mf_get_type}{GEN F} returns a symbolic name for the constructur
used to create the form, e.g. \kbd{t\_MF\_EISEN} for a general Eisenstein
series. A form has a recursive structure represented by a tree: its
definition may involve other forms, e.g. the tree attached to $T_n f$
contains $f$ as a subtree. Such trees have \emph{leaves}, forms
which do not contain a strict subtree, e.g. \kbd{t\_MF\_DELTA} is a leaf,
attached to Ramanujan's $\Delta$.

Here is the current list of types; since the names are liable to change,
they are not documented at this point. Use \kbd{mfdescribe} to visualize
their mathematical structure.
\bprog
/*leaves*/
  t_MF_CONST, t_MF_EISEN, t_MF_Ek, t_MF_DELTA, t_MF_ETAQUO, t_MF_ELL,
  t_MF_DIHEDRAL, t_MF_THETA, t_MF_TRACE, t_MF_NEWTRACE,
/*recursive*/
  t_MF_MUL, t_MF_POW, t_MF_DIV, t_MF_BRACKET, t_MF_LINEAR, t_MF_LINEAR_BHN,
  t_MF_SHIFT, t_MF_DERIV, t_MF_DERIVE2, t_MF_TWIST, t_MF_HECKE,
  t_MF_BD,
@eprog

\subsec{Nebentypus} The characters stored in modular forms and modular form
spaces have a special structure. One can recover the parameters
of an ordinary Dirichlet character by \kbd{G = gel(CHI,1)} (the underlying
\kbd{znstar}) and \kbd{chi = gel(CHI,2)} (the underlying character in
\kbd{znconreylog} form).

\fun{long}{mfcharmodulus}{GEN CHI} the modulus of $\chi$.

\fun{long}{mfcharorder}{GEN CHI} the order of $\chi$.

\fun{GEN}{mfcharpol}{GEN CHI} the cyclotomic polynomial $\Phi_n$
defining $\Q(\chi)$, always normalized so that $n$ is not $2$ mod $4$.

\subsec{Miscellaneous functions}

\fun{long}{mfnewdim}{long N, long k, GEN CHI} dimension of the new part
of the cuspidal space.

\fun{long}{mfcuspdim}{long N, long k, GEN CHI} dimension of the cuspidal space.

\fun{long}{mfolddim}{long N, long k, GEN CHI} dimension of the old part
of the cuspidal space.

\fun{long}{mfeisensteindim}{long N, long k, GEN CHI} dimension of the
Eisenstein subspace.

\fun{long}{mffulldim}{long N, long k, GEN CHI} dimension of the full space.



\fun{GEN}{mfeisensteinspaceinit}{GEN NK}

\fun{GEN}{mfdiv_val}{GEN F, GEN G, long vG}


\fun{GEN}{mfembed}{GEN E, GEN v}

\fun{GEN}{mfmatembed}{GEN E, GEN v}

\fun{GEN}{mfvecembed}{GEN E, GEN v}


\fun{long}{mfsturmNgk}{long N, GEN k}

\fun{long}{mfsturmNk}{long N, long k}

\fun{long}{mfsturm_mf}{GEN mf}

\fun{long}{mfiscuspidal}{GEN mf, GEN F}

\fun{GEN}{mftobasisES}{GEN mf, GEN F}

\fun{GEN}{mftocol}{GEN F, long lim, long d}

\fun{GEN}{mfvectomat}{GEN vF, long lim, long d}

\newpage

\chapter{Plots}

A \tet{PARI_plot} canvas is a record of dimensions, with the following fields:
\bprog
  long width;  /* window width */
  long height; /* window height */
  long hunit;  /* length of horizontal 'ticks' */
  long vunit;  /* length of vertical 'ticks' */
  long fwidth; /* font width */
  long fheight;/* font height */
  void (*draw)(PARI_plot *T, GEN w, GEN x, GEN y);
@eprog\noindent The \kbd{draw} method performs the actual drawing of
a \typ{VECSMALL} w (rectwindow indices); $x$ and $y$ are \typ{VECSMALL}s
of the same length and rectwindow $w[i]$ is drawn with its upper left
corner at offset $(x[i],y[i])$. No plot engine is available in \kbd{libpari}
by default, since thie would introduce a dependency on extra graphical
libraries. See the files \kbd{src/graph/plot*} for basic implementations of
various plot engines: \kbd{plotsvg} is particularly simple (\kbd{draw} is a
1-liner).

\fun{void}{pari_set_plot_engine}{void (*T)(PARI_plot *)} installs the
graphical engine $T$ and initializes the graphical subsystem. No routine in
this chapter will work without this initialization.

\fun{void}{pari_kill_plot_engine}{void} closes the graphical subsystem and
frees the ressources it occupies.

\subsec{Highlevel function}
Those functions plot $f(E,x)$ for $x\in [a,b]$, using $n$ regularly spaced
points (by default).

\fun{GEN}{ploth}{void *E, GEN(*f)(void*,GEN), GEN a, GEN b, long flags,long
n, long prec} draw physically.

\fun{GEN}{plotrecth}{void *E, GEN(*f)(void*,GEN), long w, GEN a,GEN b, ulong
flags,long n, long prec} draw in rectwindow $w$.

\subsec{Function }

\fun{void}{plotbox}{long ne, GEN gx2, GEN gy2}

\fun{void}{plotclip}{long rect}

\fun{void}{plotcolor}{long ne, long color}

\fun{void}{plotcopy}{long source, long dest, GEN xoff, GEN yoff, long flag}

\fun{GEN}{plotcursor}{long ne}

\fun{void}{plotdraw}{GEN list, long flag}

\fun{GEN}{plothraw}{GEN listx, GEN listy, long flag}

\fun{GEN}{plothsizes}{long flag}

\fun{void}{plotinit}{long ne, GEN x, GEN y, long flag}

\fun{void}{plotkill}{long ne}

\fun{void}{plotline}{long ne, GEN x2, GEN y2}

\fun{void}{plotlines}{long ne, GEN listx, GEN listy, long flag}

\fun{void}{plotlinetype}{long ne, long t}

\fun{void}{plotmove}{long ne, GEN x, GEN y}

\fun{void}{plotpoints}{long ne, GEN listx, GEN listy}

\fun{void}{plotpointsize}{long ne, GEN size}

\fun{void}{plotpointtype}{long ne, long t}

\fun{void}{plotrbox}{long ne, GEN x2, GEN y2}

\fun{GEN}{plotrecthraw}{long ne, GEN data, long flags}

\fun{void}{plotrline}{long ne, GEN x2, GEN y2}

\fun{void}{plotrmove}{long ne, GEN x, GEN y}

\fun{void}{plotrpoint}{long ne, GEN x, GEN y}

\fun{void}{plotscale}{long ne, GEN x1, GEN x2, GEN y1, GEN y2}

\fun{void}{plotstring}{long ne, char *x, long dir}

\subsec{Obsolete functions} These draw directly to a PostScript file
specified by a global variable and should no longer be used. Use
\kbd{plotexport} and friends instead.

\fun{void}{psdraw}{GEN list, long flag}

\fun{GEN}{psplothraw}{GEN listx, GEN listy, long flag}

\fun{GEN}{psploth}{void *E, GEN(*f)(void*,GEN), GEN a, GEN b, long flags,
long n, long prec} draw to a PostScript file.


\subsec{Dump rectwindows to a PostScript or SVG file}

$w,x,y$ are three \typ{VECSMALL}s indicating the rectwindows to dump, at
which offsets. If $T$ is \kbd{NULL}, rescale with respect to the installed
graphic engine dimensions; else with respect to $T$.

\fun{char*}{rect2ps}{GEN w, GEN x, GEN y, PARI_plot *T}

\fun{char*}{rect2ps_i}{GEN w, GEN x, GEN y, PARI_plot *T, int plotps}
if \kbd{plotps} is $0$, as above; else private version used to implement
the \kbd{plotps} graphic engine (do not rescale, rotate to portrait
orientation).

\fun{char*}{rect2svg}{GEN w, GEN x, GEN y, PARI_plot *T}

\subsec{Technical functions exported for convenience}

\fun{void}{pari_plot_by_file}{const char *env, const char *suf, const char *img}
backend used by the \kbd{plotps} and \kbd{plotsvg} graphic engines.

\fun{void}{colorname_to_rgb}{const char *s, int *r, int *g, int *b} convert
an \kbd{X11} colorname to RGB values.

\fun{void}{color_to_rgb}{GEN c, int *r, int *g, int *b} convert a pari color
(\typ{VECSMALL} RGB triple or \typ{STR} name) to RGB values.

\fun{void}{long_to_rgb}{long c, int *r, int *g, int *b} split a standard
hexadecimal color value \kbd{0xfdf5e6} to its rgb components (\kbd{0xfd},
\kbd{0xf5}, \kbd{0xe6}).
\newpage
