% Copyright (c) 2000  The PARI Group
%
% This file is part of the PARI/GP documentation
%
% Permission is granted to copy, distribute and/or modify this document
% under the terms of the GNU General Public License
\chapter{Elliptic curves and arithmetic geometry}

This chapter is quite short, but is added as a placeholder, since
we expect the library to expand in that direction.

\section{Elliptic curves}
Elliptic curves are represented in the Weierstrass model
$$ (E): y^2z + a_1xyz + a_3 yz = x^3 + a_2 x^2z + a_4 xz^2 + a_6z^3, $$
by the $5$-tuple $[a_1,a_2,a_3,a_4,a_6]$. Points in the projective
plane are represented as follows: the point at infinity $(0:1:0)$ is coded
as \kbd{[0]}, a finite point $(x:y:1)$ outside the projective line at infinity
$z = 0$ is coded as $[x,y]$. Note that other points at infinity than $(0:1:0)$
cannot be represented; this is harmless, since they do not belong to any of
the elliptic curves $E$ above.

\emph{Points on the curve} are just projective points as described above,
they are not tied to a curve in any way: the same point may be used in
conjunction with different curves, provided it satisfies their equations (if
it does not, the result is usually undefined). In particular, the point at
infinity belongs to all elliptic curves.

As with \tet{factor} for polynomial factorization, the $5$-tuple
$[a_1,a_2,a_3,a_4,a_6]$ implicitly defines a base ring over which the curve
is defined. Point coordinates must be operation-compatible with this
base ring (\kbd{gadd}, \kbd{gmul}, \kbd{gdiv} involving them should not give
errors).

\subsec{Types of elliptic curves}

We call a $5$-tuble as above an \kbd{ell5}; most functions require an
\kbd{ell} structure, as returned by \tet{ellinit}, which contains additional
data (usually dynamically computed as needed), depending on the base field.

\fun{GEN}{ellinit}{GEN E, GEN D, long prec}, returns an \tet{ell} structure,
attached to the elliptic curve $E$ : either an \kbd{ell5}, a pair $[a_4,a_6]$
or a \typ{STR} in Cremona's notation, e.g. \kbd{"11a1"}. The optional $D$
(\kbd{NULL} to omit) describes the domain over which the curve is defined.

\subsec{Type checking}

\fun{void}{checkell}{GEN e} raise an error unless $e$ is an \var{ell}.

\fun{int}{checkell_i}{GEN e} return $1$ if $e$ is an \var{ell} and $0$
otherwise.

\fun{void}{checkell5}{GEN e} raise an error unless $e$ is an \var{ell}
or an \var{ell5}.

\fun{void}{checkellpt}{GEN z} raise an error unless $z$ is a point
(either finite or at infinity).

\fun{long}{ell_get_type}{GEN e} returns the domain type over which the curve
is defined, one of

  \tet{t_ELL_Q} the field of rational numbers;

  \tet{t_ELL_NF} a number field;

  \tet{t_ELL_Qp} the field of $p$-adic numbers, for some prime $p$;

  \tet{t_ELL_Fp} a prime finite field, base field elements are represented as
  \kbd{Fp}, i.e.~a \typ{INT} reduced modulo~$p$;

  \tet{t_ELL_Fq} a nonprime finite field (a prime finite field can also be
  represented by this subtype, but this is inefficient), base field elements
  are represented as \typ{FFELT};

  \tet{t_ELL_Rg} none of the above.

\fun{void}{checkell_Fq}{GEN e} checks whether $e$ is an \kbd{ell}, defined
over a finite field (either prime or nonprime). Otherwise the function
raises a \tet{pari_err_TYPE} exception.

\fun{void}{checkell_Q}{GEN e} checks whether $e$ is an \kbd{ell}, defined
over $\Q$. Otherwise the function raises a \tet{pari_err_TYPE} exception.

\fun{void}{checkell_Qp}{GEN e} checks whether $e$ is an \kbd{ell}, defined
over some $\Q_p$. Otherwise the function raises a \tet{pari_err_TYPE}
exception.

\fun{void}{checkellisog}{GEN v} raise an error unless $v$ is an isogeny,
from \tet{ellisogeny}.

\subsec{Extracting info from an \kbd{ell} structure}

These functions expect an \kbd{ell} argument. If the required data is not
part of the structure, it is computed then inserted, and the new value is
returned.

\subsubsec{All domains}

\fun{GEN}{ell_get_a1}{GEN e}

\fun{GEN}{ell_get_a2}{GEN e}

\fun{GEN}{ell_get_a3}{GEN e}

\fun{GEN}{ell_get_a4}{GEN e}

\fun{GEN}{ell_get_a6}{GEN e}

\fun{GEN}{ell_get_b2}{GEN e}

\fun{GEN}{ell_get_b4}{GEN e}

\fun{GEN}{ell_get_b6}{GEN e}

\fun{GEN}{ell_get_b8}{GEN e}

\fun{GEN}{ell_get_c4}{GEN e}

\fun{GEN}{ell_get_c6}{GEN e}

\fun{GEN}{ell_get_disc}{GEN e}

\fun{GEN}{ell_get_j}{GEN e}

\subsubsec{Curves over $\Q$}

\fun{GEN}{ellQ_get_N}{GEN e} returns the curve conductor

\fun{void}{ellQ_get_Nfa}{GEN e, GEN *N, GEN *faN} sets $N$ to the conductor
and \kbd{faN} to its factorization

\fun{int}{ell_is_integral}{GEN e} return $1$ if $e$ is given by an integral
model, and $0$ otherwise.

\fun{long}{ellQ_get_CM}{GEN e} if $e$ has CM by a principal imaginary
quadratic order, return its discriminant. Else return $0$.

\fun{long}{ellap_CM_fast}{GEN e, ulong p, long CM} assuming that $p$
does not divide the discriminant of $E$ (in particular, $E$ has good
reduction at $p$), and that \kbd{CM} is as given by \tet{ellQ_get_CM},
return the trace of Frobenius for $E/\F_p$. This is meant to quickly compute
lots of $a_p$, esp.~when $e$ has CM by a principal quadratic order.

\fun{long}{ellrootno_global}{GEN e} returns the global root number
$c\in \{-1,1\}$.

\fun{GEN}{ellheightoo}{GEN E, GEN P, long prec} given $P = [x,y]$ an affine
point on $E$, return
$$
\lambda_\infty(P) + \dfrac{1}{12}\log|\disc E| =
 \dfrac{1}{2} \text{real}(z\eta(z)) - \log |\sigma(E,z)|
\in \R, $$
where $\lambda_\infty(P)$ is the canonical local height at infinity
and $z$ is \kbd{ellpointtoz}$(E,P)$. This
is computed using Mestre's (quadratically convergent) AGM algorithm.

\fun{long}{ellorder_Q}{GEN E, GEN P} return the order of $P\in E(\Q)$, using
the impossible value $0$ for a point of infinite order. Ultimately called
by the generic \tet{ellorder} function.

\fun{GEN}{point_to_a4a6}{GEN E, GEN P, GEN p, GEN *a4} given $E/\Q$,
$p\neq 2,3$ not dividing the discriminant of $E$ and $P\in E(\Q)$ outside the
kernel of reduction, return the image of $P$ on the short Weierstrass
model $y^2 = x^3 + a_4x + a_6$ isomorphic to the reduction $E_p$ of $E$ at $p$.
Also set \kbd{a4} to the $a_4$ coefficient in the above model. This function
allows quick computations modulo varying primes $p$, avoiding the overhead of
\kbd{ellinit}$(E,p)$, followed by a change of coordinates. It produces data
suitable for \kbd{FpE} routines.

\fun{GEN}{point_to_a4a6_Fl}{GEN E, GEN P, ulong p, ulong *pa4} as
\tet{point_to_a4a6}, returning a \kbd{Fle}.

\fun{GEN}{elldatagenerators}{GEN E} returns generators for $E(\Q)$
extracted from Cremona's table.

\fun{GEN}{ellanal_globalred}{GEN e, GEN *v} takes an \var{ell} over $\Q$
and returns a global minimal model $E$ (in \kbd{ellinit} form, over $\Q$) for
$e$ suitable for analytic computations related to the curve $L$ series: it
contains \kbd{ellglobalred} data, as well as global and local root numbers. If
\kbd{v} is not \kbd{NULL}, set \kbd{*v} to the needed change of variable:
\kbd{NULL} if $e$ was already the standard minimal model, such that $E =
\kbd{ellchangecurve(e,v)}$ otherwise. Compared to the direct use of
\kbd{ellchangecurve} followed by \kbd{ellrootno}, this function avoids
converting unneeded dynamic data and avoids potential memory leaks
(the changed curve would have had to be deleted using \tet{obj_free}). The
original curve $e$ is updated as well with the same information.

\fun{GEN}{ellanal_globalred_all}{GEN e, GEN *v, GEN *N, GEN *tam} as
\tet{ellanal_globalred}; further set \kbd{*N} to the curve conductor
and \kbd{*tam} to the product of the local Tamagawa numbers, including
the factor at infinity (multiply by the number of connected components
of $e(\R)$).

\fun{GEN}{ellintegralmodel}{GEN e, GEN *pv} return an integral model
for $e$ (in \kbd{ellinit} form, over $\Q$). Set $v = \kbd{NULL}$ (already
integral, we returned $e$ itself), else to the variable change
$[u,0,0,0]$ making $e$ integral. We have $u = 1/t$, $t > 1$.

\fun{GEN}{ellintegralmodel_i}{GEN e, GEN *pv} shallow version of
 \kbd{ellintegralmodel}.

\fun{GEN}{ellQtwist_bsdperiod}{GEN E, long s} let $E$ be a rational elliptic
curve given by a minimal model, $\Lambda_E$ its period lattice, and
$s\in\{-1,1\}$. Let $\Omega_E^\pm$ be the canonical periods in
$\sqrt{\pm 1}\R^+$ generating $\Lambda_E \cap \sqrt{\pm 1} \R$.
Return $\Omega_E^+$ if $s = 1$ and $\Omega_E^-$ if $s = -1$.

\fun{GEN}{elltors_psylow}{GEN e, ulong p} as \kbd{elltors}, but return the
$p$-Sylow subgroup of the torsion group.

\misctitle{Deprecated routines}

\fun{GEN}{elltors0}{GEN e, long flag} this function is deprecated; use
\tet{elltors}

\subsubsec{Curves over a number field \var{nf}}

Let $K$ be the number field over which $E$ is defined, given by
a \var{nf} or \var{bnf} structure.

\fun{GEN}{ellnf_get_nf}{GEN E} returns the underlying \kbd{nf}.

\fun{GEN}{ellnf_get_bnf}{GEN x} returns \kbd{NULL} if $K$ does not contain
a \var{bnf} structure, else return the \var{bnf}.

\fun{GEN}{ellnf_vecarea}{GEN E} returns the vector of the period lattices areas
of all the complex embeddings of \kbd{E} in the same order as \kbd{E.nf.roots}.

\fun{GEN}{ellnf_veceta}{GEN E} returns the vector of the quasi-periods of
all the complex embeddings of \kbd{E} in the same order as \kbd{E.nf.roots}.

\fun{GEN}{ellnf_vecomega}{GEN E} returns the vector of the periods of
all the complex embeddings of \kbd{E} in the same order as \kbd{E.nf.roots}.

\subsubsec{Curves over $\Q_p$}

\fun{GEN}{ellQp_get_p}{GEN E} returns $p$

\fun{long}{ellQp_get_prec}{GEN E} returns the default $p$-adic accuracy to
which we must compute approximate results attached to $E$.

\fun{GEN}{ellQp_get_zero}{GEN x} returns $O(p^n)$, where $n$ is the default
$p$-adic accuracy as above.

The following functions are only defined when $E$ has multiplicative
reduction (Tate curves):

\fun{GEN}{ellQp_Tate_uniformization}{GEN E, long prec} returns a
\typ{VEC} containing $u^2, u, q, [a,b]$, at $p$-adic precision \kbd{prec}.

\fun{GEN}{ellQp_u}{GEN E, long prec} returns $u$.

\fun{GEN}{ellQp_u2}{GEN E, long prec} returns $u^2$.

\fun{GEN}{ellQp_q}{GEN E, long prec} returns the Tate period $q$.

\fun{GEN}{ellQp_ab}{GEN E, long prec} returns $[a,b]$.

\fun{GEN}{ellQp_AGM}{GEN E, long prec} returns $[a,b,R,v]$, where
$v$ is an integer, $a, b, R$ are vectors describing the sequence of
$2$-isogenous curves $E_i: y^2 = x(x+A_i)(x+A_i-B_i)$, $i \geq 1$
converging to the singular curve $E_\infty: y^2 = x^2(x+M)$. We have
$a[i] = A[i] p^v$, $b[i] = B[i] p^v$, $R[i] = A_i - B_i$. These are used in
\kbd{ellpointtoz} and \kbd{ellztopoint}.

\fun{GEN}{ellQp_L}{GEN E, long prec} returns the ${\cal L}$-invariant $L$.

\fun{GEN}{ellQp_root}{GEN E, long prec} returns $e_1$.

\subsubsec{Curves over a finite field $\F_q$}

\fun{GEN}{ellff_get_p}{GEN E} returns the characteristic

\fun{GEN}{ellff_get_field}{GEN E} returns $p$ if $\F_q$ is a prime field, and
a \typ{FFELT} belonging to $\F_q$ otherwise.

\fun{GEN}{ellff_get_card}{GEN E} returns $\#E(\F_q)$

\fun{GEN}{ellff_get_gens}{GEN E} returns a minimal set of generators for
$E(\F_q)$.

\fun{GEN}{ellff_get_group}{GEN E} returns \kbd{ellgroup}$(E)$.

\fun{GEN}{ellff_get_m}{GEN E} returns the \typ{INT} $m$ as needed by the
\kbd{gen\_ellgroup} function (the order of the pairing required to verify a
generating set).

\fun{GEN}{ellff_get_o}{GEN E} returns $[d, \kbd{factor{d}}]$, where $d$ is
the exponent of $E(\F_q)$.

\fun{GEN}{ellff_get_D}{GEN E} returns the elementary divisors for $E(\F_q)$
in a form suitable for \tet{gen_ellgens}: either $[d_1]$ or $[d_1,d_2]$,
where $d_1$ is in \tet{elff_get_o} format.

$[d, \kbd{factor{d}}]$, where $d$ is
the exponent of $E(\F_q)$.

\fun{GEN}{ellff_get_a4a6}{GEN E} returns a canonical ``short model'' for $E$,
and the corresponding change of variable $[u,r,s,t]$. For $p\neq 2,3$,
this is $[A_4,A_6,[u,r,s,t]]$, corresponding to $y^2 = x^3 + A_4x + A_6$,
where $A_4 = -27c_4$, $A_6 = -54c_6$, $[u,r,s,t] = [6, 3b_2,3a_1,108a_3]$.

\item If $p = 3$ and the curve is ordinary ($b_2\neq 0$), this is
$[[b_2], A_6, [1,v,-a_1,-a_3]]$, corresponding to
$$y^2 = x^3 + b_2 x^2 + A_6,$$
where $v = b_4/b_2$, $A_6 = b_6 - v(b_4+v^2)$.

\item If $p = 3$ and the curve is supersingular ($b_2 = 0$), this is
$[-b_4, b_6, [1,0,-a_1,-a_3]]$, corresponding to
$$y^2 = x^3 + 2b_4 x + b_6.$$

\item If $p = 2$ and the curve is ordinary ($a_1 \neq 0$), return
$[A_2,A_6,[a_1^{-1}, da_1^{-2}, 0, (a_4+d^2)a_1^{-1}]]$, corresponding to
$$ y^2 + xy = x^3 + A_2 x^2 + A_6,$$
where
$d = a_3/a_1$, $a_1^2 A_2 = (a_2 + d)$ and
$$ a_1^6 A_6 = d^3 + a_2 d^2 + a_4 d + a_6 + (a_4^2 + d^4)a_1^{-2}.$$

\item If $p = 2$ and the curve is supersingular ($a_1 = 0$, $a_3\neq 0$), return
$[[a_3, A_4, 1/a_3], A_6, [1,a_2,0,0]]$, corresponding to
$$ y^2 + a_3 y = x^3 + A_4 x + A_6,$$
where $A_4 = a_2^2 + a_4$, $ A_6 = a_2a_4 + a_6$. The value $1/a_3$ is
included in the vector since it is frequently needed in computations.

\subsubsec{Curves over $\C$} (This includes curves over $\Q$!)

\fun{long}{ellR_get_prec}{GEN E} return the default accuracy to
which we must compute approximate results attached to $E$.

\fun{GEN}{ellR_ab}{GEN E, long prec} return $[a,b]$

\fun{GEN}{ellR_omega}{GEN x, long prec} return periods
$[\omega_1,\omega_2]$.

\fun{GEN}{ellR_eta}{GEN E, long prec} return quasi-periods
$[\eta_1,\eta_2]$.

\fun{GEN}{ellR_area}{GEN x, long prec} return the area
$(\Im(\omega_1\*\overline{\omega_2}))$.

\fun{GEN}{ellR_roots}{GEN E, long prec} return $[e_1,e_2,e_3]$. If $E$ is
defined over $\R$, then $e_1$ is real. If furthermore $\disc E > 0$, then
$e_1 > e_2 > e_3$.

\fun{long}{ellR_get_sign}{GEN E} if $E$ is defined over $\R$ returns the
signe of its discriminant, otherwise return $0$.

\subsec{Points}

\fun{int}{ell_is_inf}{GEN z} tests whether the point $z$ is the point at
infinity.

\fun{GEN}{ellinf}{} returns the point at infinity \kbd{[0]}.

\subsec{Change of variables}

\fun{GEN}{ellchangeinvert}{GEN w} given a change of variables $w =
[u,r,s,t]$, returns the inverse change of variables $w'$, such that if $E' =
\kbd{ellchangecurve(E, w)}$, then $E = \kbd{ellchangecurve}(E, w')$.

\subsec{Generic helper functions}

The naming scheme assumes an affine equation
$F(x,y) = f(x) - (y^2 + h(x)y) = 0$
in standard Weierstrass form: $f = x^3+a_2x^2+a_4x+a_6$, $h = a_1x + a_3$.
Unless mentionned otherwise, these routine assume that all arguments are
compatible with generic functions of \kbd{gadd} or \kbd{gmul} type. In
particular they do not handle elements in number field in \kbd{nfalgtobasis}
format.

\fun{GEN}{ellbasechar}{GEN E} returns the characteristic of the base ring over
which $E$ is defined.

\fun{GEN}{ec_bmodel}{GEN E} returns the polynomial $4x^3 + b_2x^2 + 2b_4x +
b_6$.

\fun{GEN}{ec_phi2}{GEN E} returns the polynomial $x^4 - b_4x^2 - 2b_6*X - b_8$.

\fun{GEN}{ec_f_evalx}{GEN E, GEN x} returns $f(x)$.

\fun{GEN}{ec_h_evalx}{GEN E, GEN x} returns $h(x)$.

\fun{GEN}{ec_dFdx_evalQ}{GEN E, GEN Q} returns $3x^2 + 2a_2x + a_4 -a_1y$,
where $Q = [x,y]$.

\fun{GEN}{ec_dFdy_evalQ}{GEN E, GEN Q} returns $-(2y + a_1 x + a_3)$,
where $Q = [x,y]$.

\fun{GEN}{ec_dmFdy_evalQ}{GEN e, GEN Q} returns $2y + a_1 x + a_3$,
where $Q = [x,y]$.

\fun{GEN}{ec_2divpol_evalx}{GEN E, GEN x} returns
$4x^3 + b_2\*x^2 + 2\*b_4x + b_6$. This function supports inputs
in \kbd{nfalgtobasis} format.

\fun{GEN}{ec_half_deriv_2divpol_evalx}{GEN E, GEN x} returns
$6\*x^2 + b_2\*x + b_4$.

\fun{GEN}{ec_3divpol_evalx}{GEN E, GEN x} returns
$3\*x^4 + b_2\*x^2 + 3\*b_4\*x^2 + 3\*b_6\*x + b_8$.

\subsec{Functions to handle elliptic curves over finite fields}

\subsubsec{Tolerant routines}

\fun{GEN}{ellap}{GEN E, GEN p} given a prime number $p$ and an elliptic curve
defined over $\Q$ or $\Q_p$ (assumed integral and minimal at $p$), computes
the  trace of  Frobenius  $a_p = p+1 - \#E(\F_p)$. If $E$ is defined over
a nonprime finite field $\F_q$, ignore $p$ and return $q+1 - \#E(\F_q)$.
When $p$ is implied ($E$ defined over $\Q_p$ or a finite field), $p$ can be
omitted (set to \kbd{NULL}).

\subsubsec{Curves defined a nonprime finite field}
In this subsection, we assume that \tet{ell_get_type}$(E)$ is \tet{t_ELL_Fq}.
(As noted above, a curve defined over $\Z/p\Z$ can be represented as a
\tet{t_ELL_Fq}.)

\fun{GEN}{FF_elltwist}{GEN E} returns the coefficients
$[a_1,a_2,a_3,a_4,a_6]$ of the quadratic twist of $E$.

\fun{GEN}{FF_ellmul}{GEN E, GEN P, GEN n} returns $[n]P$ where $n$ is an
integer and $P$ is a point on the curve $E$.

\fun{GEN}{FF_ellrandom}{GEN E} returns a random point in $E(\F_q)$.
This function never returns the point at infinity, unless this is the
only point on the curve.

\fun{GEN}{FF_ellorder}{GEN E, GEN P, GEN o} returns the order of the point
$P$, where $o$ is a multiple of the order of $P$, or its factorization.

\fun{GEN}{FF_ellcard}{GEN E} returns $\#E(\F_q)$.

\fun{GEN}{FF_ellcard_SEA}{GEN E, long s}
This function returns $\#E(\F_q)$, using the Schoof-Elkies-Atkin
algorithm.  Assume $p\neq 2,3$.
The parameter $s$ has the same meaning as in \kbd{Fp\_ellcard\_SEA}.

\fun{GEN}{FF_ellgens}{GEN E} returns the generators of the group $E(\F_q)$.

\fun{GEN}{FF_elllog}{GEN E, GEN P, GEN G, GEN o} Let \kbd{G} be a point of
order \kbd{o}, return $e$ such that $[e]P=G$. If $e$ does not exists, the
result is undefined.

\fun{GEN}{FF_ellgroup}{GEN E, GEN *pm} returns the structure of the Abelian
group $E(\F_q)$ and set \kbd{*pm} to $m$ (see \kbd{gen\_ellgens}).

\fun{GEN}{FF_ellweilpairing}{GEN E, GEN P, GEN Q, GEN m} returns the
Weil pairing of the points of $m$-torsion $P$ and $Q$.

\fun{GEN}{FF_elltatepairing}{GEN E, GEN P, GEN Q, GEN m} returns the Tate
pairing of $P$ and $Q$, where $[m]P = 0$.

\section{Arithmetic on elliptic curve over a finite field in simple form}

The functions in this section no longer operate on elliptic curve structures,
as seen up to now. They are used to implement those higher-level functions
without using cached information and thus require suitable explicitly
enumerated data.

\subsec{Helper functions}

\fun{GEN}{elltrace_extension}{GEN t, long n, GEN q} Let $E$ some elliptic curve
over $\F_q$ such that the trace of the Frobenius is $t$, returns the trace of
the Frobenius over $\F_q^n$.

\subsec{Elliptic curves over $\F_p$, $p>3$}

Let $p$ a prime number and $E$ the elliptic curve given by the equation
$E:y^2=x^3+a_4\*x+a_6$, with $a_4$ and $a_6$ in $\F_p$. A \kbd{FpE} is a
point of $E(\F_p)$.  Since an affine point and $a_4$ determine an unique
$a6$, most functions do not take $a_6$ as an argument. A \kbd{FpE} is either
the point at infinity (\kbd{ellinf()}) or a $FpV$ whith two components. The
parameters $a_4$ and $a_6$ are given as \typ{INT}s when required.

\fun{GEN}{Fp_ellj}{GEN a4, GEN a6, GEN p}
returns the $j$-invariant of the curve $E$.

\fun{int}{Fp_elljissupersingular}{GEN j, GEN p} returns $1$ if $j$ is the
$j$-invariant of a supersingular curve over $\F_p$, $0$ otherwise.

\fun{GEN}{Fp_ellcard}{GEN a4, GEN a6, GEN p} returns the cardinality of the
group $E(\F_p)$.

\fun{GEN}{Fp_ellcard_SEA}{GEN a4, GEN a6, GEN p, long s}
This function returns $\#E(\F_p)$, using the Schoof-Elkies-Atkin algorithm.
If the \kbd{seadata} package is installed, the function will be faster.

The extra flag \kbd{s}, if set to a nonzero value, causes the computation to
return \kbd{gen\_0} (an impossible cardinality) if one of the small primes
$\ell$ divides the curve order but does not divide $s$.
For cryptographic applications, where one is usually interested in curves of
prime order, setting $s=1$ efficiently weeds out most uninteresting curves; if
curves of order a power of $2$ times a prime are acceptable, set $s=2$.
If moreover \kbd{s} is negative, similar checks are performed for the
twist of the curve.

\fun{GEN}{Fp_ffellcard}{GEN a4, GEN a6, GEN q, long n, GEN p} returns the
cardinality of the group $E(\F_q)$ where $q=p^n$.

\fun{GEN}{Fp_ellgroup}{GEN a4, GEN a6, GEN N, GEN p, GEN *pm} returns the
group structure $D$ of the group $E(\F_p)$, which is assumed to be of order $N$
and set \kbd{*pm} to $m$.

\fun{GEN}{Fp_ellgens}{GEN a4, GEN a6, GEN ch, GEN D, GEN m, GEN p} returns
generators of the group $E(\F_p)$ with the base change \kbd{ch} (see
\kbd{FpE\_changepoint}), where $D$ and $m$ are as returned by
\kbd{Fp\_ellgroup}.

\fun{GEN}{Fp_elldivpol}{GEN a4, GEN a6, long n, GEN p} returns the $n$-division
polynomial of the elliptic curve $E$.

\fun{void}{Fp_elltwist}{GEN a4, GEN a6, GEN p, GEN *pA4, GEN *pA6}
sets \kbd{*pA4} and \kbd{*pA6} to the corresponding parameters for the
quadratic twist of $E$.

\subsec{\kbd{FpE}}

\fun{GEN}{FpE_add}{GEN P, GEN Q, GEN a4, GEN p} returns the sum $P+Q$
in the group $E(\F_p)$, where $E$ is defined by $E:y^2=x^3+a_4\*x+a_6$,
for any value of $a_6$ compatible with the points given.

\fun{GEN}{FpE_sub}{GEN P, GEN Q, GEN a4, GEN p} returns $P-Q$.

\fun{GEN}{FpE_dbl}{GEN P, GEN a4, GEN p} returns $2.P$.

\fun{GEN}{FpE_neg}{GEN P, GEN p} returns $-P$.

\fun{GEN}{FpE_mul}{GEN P, GEN n, GEN a4, GEN p} return $n.P$.

\fun{GEN}{FpE_changepoint}{GEN P, GEN m, GEN a4, GEN p} returns the image
$Q$ of the point $P$ on the curve $E:y^2=x^3+a_4\*x+a_6$ by the coordinate
change $m$ (which is a \kbd{FpV}).

\fun{GEN}{FpE_changepointinv}{GEN P, GEN m, GEN a4, GEN p} returns the image
$Q$ on the curve $E:y^2=x^3+a_4\*x+a_6$ of the point $P$ by the inverse of the
coordinate change $m$ (which is a \kbd{FpV}).

\fun{GEN}{random_FpE}{GEN a4, GEN a6, GEN p} returns a random point on
$E(\F_p)$, where $E$ is defined by $E:y^2=x^3+a_4\*x+a_6$.

\fun{GEN}{FpE_order}{GEN P, GEN o, GEN a4, GEN p} returns the order of $P$ in
the group $E(\F_p)$, where $o$ is a multiple of the order of $P$, or its
factorization.

\fun{GEN}{FpE_log}{GEN P, GEN G, GEN o, GEN a4, GEN p} Let \kbd{G} be a
point of order \kbd{o}, return $e$ such that $e.P=G$. If $e$ does not exists,
the result is currently undefined.

\fun{GEN}{FpE_tatepairing}{GEN P, GEN Q, GEN m, GEN a4, GEN p} returns the
Tate pairing of the point of $m$-torsion $P$ and the point $Q$.

\fun{GEN}{FpE_weilpairing}{GEN P, GEN Q, GEN m, GEN a4, GEN p} returns the
Weil pairing of the points of $m$-torsion $P$ and $Q$.

\fun{GEN}{FpE_to_mod}{GEN P, GEN p} returns $P$ as a vector of \typ{INTMOD}s.

\fun{GEN}{RgE_to_FpE}{GEN P, GEN p} returns the \kbd{FpE} obtained by applying
\kbd{Rg\_to\_Fp} coefficientwise.

\subsec{\kbd{Fle}}
Let $p$ be a prime \kbd{ulong}, and $E$ the elliptic curve given by the
equation $E:y^2=x^3+a_4\*x+a_6$, where $a_4$ and $a_6$ are \kbd{ulong}.
A \kbd{Fle} is either the point at infinity (\kbd{ellinf()}), or a \kbd{Flv}
with two components $[x,y]$.

\fun{long}{Fl_elltrace}{ulong a4, ulong a6, ulong p} returns the trace $t$ of
the Frobenius of $E(\F_p)$. The cardinality of $E(\F_p)$ is thus $p+1-t$,
which might not fit in an \kbd{ulong}.

\fun{long}{Fl_elltrace_CM}{long CM, ulong a4, ulong a6, ulong p} as
\tet{Fl_elltrace}. If \kbd{CM} is $0$, use the standard algorithm; otherwise
assume the curve has CM by a principal imaginary quadratic order of
discriminant \kbd{CM} and use a faster algorithm. Useful when the curve is
the reduction of $E/\Q$, which has CM by a principal order, and we need the
trace of Frobenius for many distinct $p$, see \tet{ellQ_get_CM}.

\fun{ulong}{Fl_elldisc}{ulong a4, ulong a6, ulong p}
returns the discriminant of the curve $E$.

\fun{ulong}{Fl_elldisc_pre}{ulong a4, ulong a6, ulong p, ulong pi}
returns the discriminant of the curve $E$, assuming $pi$ is the pseudo inverse
of $p$.

\fun{ulong}{Fl_ellj}{ulong a4, ulong a6, ulong p}
returns the $j$-invariant of the curve $E$.

\fun{ulong}{Fl_ellj_pre}{ulong a4, ulong a6, ulong p, ulong pi}
returns the $j$-invariant of the curve $E$, assuming $pi$ is the pseudo inverse
of $p$.

\fun{void}{Fl_ellj_to_a4a6}{ulong j, ulong p, ulong *pa4, ulong *pa6}
sets \kbd{*pa4} to $a_4$ and \kbd{*pa6} to $a_6$ where $a_4$ and $a_6$
define a fixed elliptic curve with $j$-invariant $j$.

\fun{void}{Fl_elltwist}{ulong a4, ulong a6, ulong p, ulong *pA4, ulong *pA6}
set \kbd{*pA4} to $A_4$ and \kbd{*pA6} to $A_6$ where $A_4$ and $A_6$
define the twist of $E$.

\fun{void}{Fl_elltwist_disc}{ulong a4, ulong a6, ulong D, ulong p, ulong *pA4,
ulong *pA6}
sets \kbd{*pA4} to $A_4$ and \kbd{*pA6} to $A_6$ where $A_4$ and $A_6$
define the twist of $E$ by the discriminant $D$.

\fun{GEN}{Fl_ellptors}{ulong l, ulong N, ulong a4, ulong a6, ulong p}
return a basis of the $l$-torsion subgroup of $E$.

\fun{GEN}{Fle_add}{GEN P, GEN Q, ulong a4, ulong p}

\fun{GEN}{Fle_dbl}{GEN P, ulong a4, ulong p}

\fun{GEN}{Fle_sub}{GEN P, GEN Q, ulong a4, ulong p}

\fun{GEN}{Fle_mul}{GEN P, GEN n, ulong a4, ulong p}

\fun{GEN}{Fle_mulu}{GEN P, ulong n, ulong a4, ulong p}

\fun{GEN}{Fle_order}{GEN P, GEN o, ulong a4, ulong p}

\fun{GEN}{Fle_log}{GEN P, GEN G, GEN o, ulong a4, ulong p}

\fun{GEN}{Fle_tatepairing}{GEN P, GEN Q, ulong m, ulong a4, ulong p}

\fun{GEN}{Fle_weilpairing}{GEN P, GEN Q, ulong m, ulong a4, ulong p}

\fun{GEN}{random_Fle}{ulong a4, ulong a6, ulong p}

\fun{GEN}{random_Fle_pre}{ulong a4, ulong a6, ulong p, ulong pi}

\fun{GEN}{Fle_changepoint}{GEN x, GEN ch, ulong p}, \kbd{ch} is assumed
to give the change of coordinates $[u,r,s,t]$ as a \typ{VECSMALL}.

\fun{GEN}{Fle_changepointinv}{GEN x, GEN ch, ulong p}, as \tet{Fle_changepoint}

\subsec{\kbd{FpJ}}

Let $p$ be a prime \typ{INT}, and $E$ the elliptic curve given by the
equation $E:y^2=x^3+a_4\*x+a_6$, where $a_4$ and $a_6$ are \typ{INT}.
A \kbd{FpJ} is a \kbd{FpV} with three components $[x,y,z]$, representing
the affine point $[x/z^2,y/z^3]$ in Jacobian coordinates, the point at
infinity being represented by $[1, 1, 0]$. The following must holds:
$y^2=x^3+a_4\*x\*z^4+a_6\*z^6$. For all nonzero $u$, the points
$[u^2\*x,u^3\*y,u\*z]$ and $[x,y,z]$ are representing the same affine point.

\fun{GEN}{FpJ_add}{GEN P, GEN Q, GEN a4, GEN p}

\fun{GEN}{FpJ_dbl}{GEN P, GEN a4, GEN p}

\fun{GEN}{FpJ_mul}{GEN P, GEN n, GEN a4, GEN p};

\fun{GEN}{FpJ_neg}{GEN P, GEN p} return $-P$.

\fun{GEN}{FpJ_to_FpE}{GEN P, GEN p} return the corresponding \kbd{FpE}.

\fun{GEN}{FpE_to_FpJ}{GEN P} return the corresponding \kbd{FpJ}.

\subsec{\kbd{Flj}}

Below, \kbd{pi} is assumed to be the precomputed inverse of $p$.

\fun{GEN}{Fle_to_Flj}{GEN P} convert a \kbd{Fle} to an equivalent \kbd{Flj}.

\fun{GEN}{Flj_to_Fle}{GEN P, ulong p} convert a \kbd{Flj} to the equivalent
\kbd{Fle}.

\fun{GEN}{Flj_to_Fle_pre}{GEN P, ulong p, ulong pi} convert a \kbd{Flj} to the
equivalent \kbd{Fle}.

\fun{GEN}{Flj_add_pre}{GEN P, GEN Q, ulong a4, ulong p, ulong pi}

\fun{GEN}{Flj_dbl_pre}{GEN P, ulong a4, ulong p, ulong pi}

\fun{GEN}{Flj_neg}{GEN P, ulong p} return $-P$.

\fun{GEN}{Flj_mulu_pre}{GEN P, ulong n, ulong a4, ulong p, ulong pi}

\fun{GEN}{random_Flj_pre}{ulong a4, ulong a6, ulong p, ulong pi}

\fun{GEN}{Flj_changepointinv_pre}{GEN P, GEN ch, ulong p, ulong pi}
where \kbd{ch} is the \kbd{Flv} $[u,r,s,t]$.

\fun{GEN}{FljV_factorback_pre}{GEN P, GEN L, ulong p, ulong pi}

\subsec{Elliptic curves over $\F_{2^n}$}
Let $T$ be an irreducible \kbd{F2x} and $E$ the
elliptic curve given by either the equation
$E:y^2+x*y=x^3+a_2\*x^2+a_6$, where $a_2, a_6$ are \kbd{F2x} in
$\F_2[X]/(T)$ (ordinary case) or $E:y^2+a_3*y=x^3+a_4\*x+a_6$, where
$a_3, a_4, a_6$ are \kbd{F2x} in $\F_2[X]/(T)$ (supersingular case).

A \kbd{F2xqE} is a point of $E(\F_2[X]/(T))$.  In the supersingular case, the
parameter \kbd{a2} is actually the \typ{VEC} $[a_3,a_4,a_3^{-1}]$.

\fun{GEN}{F2xq_ellcard}{GEN a2, GEN a6, GEN T}
Return the order of the group $E(\F_2[X]/(T))$.

\fun{GEN}{F2xq_ellgroup}{GEN a2, GEN a6, GEN N, GEN T, GEN *pm}
Return the group structure $D$ of the group $E(\F_2[X]/(T))$,
which is assumed to be of order $N$ and set \kbd{*pm} to $m$.

\fun{GEN}{F2xq_ellgens}{GEN a2, GEN a6, GEN ch, GEN D, GEN m, GEN T}
Returns generators of the group $E(\F_2[X]/(T))$ with the base change \kbd{ch}
(see \kbd{F2xqE\_changepoint}), where $D$ and $m$ are as returned by
\kbd{F2xq\_ellgroup}.

\fun{void}{F2xq_elltwist}{GEN a4, GEN a6, GEN T, GEN *a4t, GEN *a6t}
sets \kbd{*a4t} and \kbd{*a6t} to the parameters of the quadratic twist of $E$.

\subsec{\kbd{F2xqE}}

\fun{GEN}{F2xqE_changepoint}{GEN P, GEN m, GEN a2, GEN T} returns the image
$Q$ of the point $P$ on the curve $E:y^2+x*y=x^3+a_2\*x^2+a_6$ by the coordinate
change $m$ (which is a \kbd{F2xqV}).

\fun{GEN}{F2xqE_changepointinv}{GEN P, GEN m, GEN a2, GEN T} returns the image
$Q$ on the curve $E:y^2=x^3+a_4\*x+a_6$ of the point $P$ by the inverse of the
coordinate change $m$ (which is a \kbd{F2xqV}).

\fun{GEN}{F2xqE_add}{GEN P, GEN Q, GEN a2, GEN T}

\fun{GEN}{F2xqE_sub}{GEN P, GEN Q, GEN a2, GEN T}

\fun{GEN}{F2xqE_dbl}{GEN P, GEN a2, GEN T}

\fun{GEN}{F2xqE_neg}{GEN P, GEN a2, GEN T}

\fun{GEN}{F2xqE_mul}{GEN P, GEN n, GEN a2, GEN T}

\fun{GEN}{random_F2xqE}{GEN a2, GEN a6, GEN T}

\fun{GEN}{F2xqE_order}{GEN P, GEN o, GEN a2, GEN T} returns the order of $P$ in
the group $E(\F_2[X]/(T))$, where $o$ is a multiple of the order of $P$, or its
factorization.

\fun{GEN}{F2xqE_log}{GEN P, GEN G, GEN o, GEN a2, GEN T} Let \kbd{G} be a
point of order \kbd{o}, return $e$ such that $e.P=G$. If $e$ does not exists,
the result is currently undefined.

\fun{GEN}{F2xqE_tatepairing}{GEN P, GEN Q, GEN m, GEN a2, GEN T} returns the
Tate pairing of the point of $m$-torsion $P$ and the point $Q$.

\fun{GEN}{F2xqE_weilpairing}{GEN Q, GEN Q, GEN m, GEN a2, GEN T} returns the
Weil pairing of the points of $m$-torsion $P$ and $Q$.

\fun{GEN}{RgE_to_F2xqE}{GEN P, GEN T} returns the \kbd{F2xqE} obtained by
applying \kbd{Rg\_to\_F2xq} coefficientwise.

\subsec{Elliptic curves over $\F_q$, small characteristic $p>2$ }
Let $p > 2$ be a prime \kbd{ulong}, $T$ an irreducible \kbd{Flx} mod $p$, and
$E$ the elliptic curve given by the equation $E:y^2=x^3+a_4\*x+a_6$, where $a_4$
and $a_6$ are \kbd{Flx} in $\F_p[X]/(T)$.  A \kbd{FlxqE} is a point of
$E(\F_p[X]/(T))$.

In the special case $p = 3$, ordinary elliptic curves ($j(E)\neq 0$) cannot
be represented as above, but admit a model $E:y^2 = x^3+a_2\*x^2+a_6$ with
$a_2$ and $a_6$ being \kbd{Flx} in $\F_3[X]/(T)$. In that case, the parameter
\kbd{a2} is actually stored as a \typ{VEC}, $[a_2]$, to avoid ambiguities.

\fun{GEN}{Flxq_ellj}{GEN a4, GEN a6, GEN T, ulong p}
returns the $j$-invariant of the curve $E$.

\fun{void}{Flxq_ellj_to_a4a6}{GEN j, GEN T, ulong p, GEN *pa4, GEN *pa6}
sets \kbd{*pa4} to $a_4$ and \kbd{*pa6} to $a_6$ where $a_4$ and $a_6$
define a fixed elliptic curve with $j$-invariant $j$.

\fun{GEN}{Flxq_ellcard}{GEN a4, GEN a6, GEN T, ulong p}
returns the order of $E(\F_p[X]/(T))$.

\fun{GEN}{Flxq_ellgroup}{GEN a4, GEN a6, GEN N, GEN T, ulong p, GEN *pm}
returns the group structure $D$ of the group $E(\F_p[X]/(T))$,
which is assumed to be of order $N$ and sets \kbd{*pm} to $m$.

\fun{GEN}{Flxq_ellgens}{GEN a4, GEN a6, GEN ch, GEN D, GEN m, GEN T, ulong p}
returns generators of the group $E(\F_p[X]/(T))$ with the base change \kbd{ch}
(see \kbd{FlxqE\_changepoint}), where $D$ and $m$ are as returned by
\kbd{Flxq\_ellgroup}.

\fun{void}{Flxq_elltwist}{GEN a4, GEN a6, GEN T, ulong p, GEN *pA4, GEN *pA6}
sets \kbd{*pA4} and \kbd{*pA6} to the corresponding parameters for the
quadratic twist of $E$.

\subsec{\kbd{FlxqE}}

Let $p > 2$ be a prime number.

\fun{GEN}{FlxqE_changepoint}{GEN P, GEN m, GEN a4, GEN T, ulong p} returns
the image $Q$ of the point $P$ on the curve $E:y^2=x^3+a_4\*x+a_6$ by the
coordinate change $m$ (which is a \kbd{FlxqV}).

\fun{GEN}{FlxqE_changepointinv}{GEN P, GEN m, GEN a4, GEN T, ulong p} returns
the image $Q$ on the curve $E:y^2=x^3+a_4\*x+a_6$ of the point $P$ by the
inverse of the coordinate change $m$ (which is a \kbd{FlxqV}).

\fun{GEN}{FlxqE_add}{GEN P, GEN Q, GEN a4, GEN T, ulong p}

\fun{GEN}{FlxqE_sub}{GEN P, GEN Q, GEN a4, GEN T, ulong p}

\fun{GEN}{FlxqE_dbl}{GEN P, GEN a4, GEN T, ulong p}

\fun{GEN}{FlxqE_neg}{GEN P, GEN T, ulong p}

\fun{GEN}{FlxqE_mul}{GEN P, GEN n, GEN a4, GEN T, ulong p}

\fun{GEN}{random_FlxqE}{GEN a4, GEN a6, GEN T, ulong p}

\fun{GEN}{FlxqE_order}{GEN P, GEN o, GEN a4, GEN T, ulong p} returns the
order of $P$ in the group $E(\F_p[X]/(T))$, where $o$ is a multiple of the
order of $P$, or its factorization.

\fun{GEN}{FlxqE_log}{GEN P, GEN G, GEN o, GEN a4, GEN T, ulong p} Let \kbd{G}
be a point of order \kbd{o}, return $e$ such that $e.P=G$. If $e$ does not
exists, the result is currently undefined.

\fun{GEN}{FlxqE_tatepairing}{GEN P, GEN Q, GEN m, GEN a4, GEN T, ulong p}
returns the Tate pairing of the point of $m$-torsion $P$ and the point $Q$.

\fun{GEN}{FlxqE_weilpairing}{GEN P, GEN Q, GEN m, GEN a4, GEN T, ulong p}
returns the Weil pairing of the points of $m$-torsion $P$ and $Q$.

\fun{GEN}{RgE_to_FlxqE}{GEN P, GEN T, ulong p} returns the \kbd{FlxqE}
obtained by applying \kbd{Rg\_to\_Flxq} coefficientwise.

\subsec{Elliptic curves over $\F_q$, large characteristic }

Let $p > 3$ be a prime number, $T$ an irreducible polynomial mod $p$, and $E$
the elliptic curve given by the equation $E:y^2=x^3+a_4\*x+a_6$ with $a_4$ and
$a_6$ in $\F_p[X]/(T)$.  A \kbd{FpXQE} is a point of $E(\F_p[X]/(T))$.

\fun{GEN}{FpXQ_ellj}{GEN a4, GEN a6, GEN T, GEN p}
returns the $j$-invariant of the curve $E$.

\fun{int}{FpXQ_elljissupersingular}{GEN j, GEN T, GEN p} returns $1$ if $j$ is
the $j$-invariant of a supersingular curve over $\F_p[X]/(T)$, $0$ otherwise.

\fun{GEN}{FpXQ_ellcard}{GEN a4, GEN a6, GEN T, GEN p}
returns the order of $E(\F_p[X]/(T))$.

\fun{GEN}{Fq_ellcard_SEA}{GEN a4, GEN a6, GEN q, GEN T, GEN p, long s}
This function returns $\#E(\F_p[X]/(T))$, using the Schoof-Elkies-Atkin
algorithm.
Assume $p\neq 2,3$, and $q$ is the cardinality of $\F_p[X]/(T)$.
The parameter $s$ has the same meaning as in \kbd{Fp\_ellcard\_SEA}.
If the \kbd{seadata} package is installed, the function will be faster.

\fun{GEN}{FpXQ_ellgroup}{GEN a4, GEN a6, GEN N, GEN T, GEN p, GEN *pm}
Return the group structure $D$ of the group $E(\F_p[X]/(T))$,
which is assumed to be of order $N$ and set \kbd{*pm} to $m$.

\fun{GEN}{FpXQ_ellgens}{GEN a4, GEN a6, GEN ch, GEN D, GEN m, GEN T, GEN p}
Returns generators of the group $E(\F_p[X]/(T))$ with the base change \kbd{ch}
(see \kbd{FpXQE\_changepoint}), where $D$ and $m$ are as returned by
\kbd{FpXQ\_ellgroup}.

\fun{GEN}{FpXQ_elldivpol}{GEN a4, GEN a6, long n, GEN T, GEN p} returns the
$n$-division polynomial of the elliptic curve $E$.

\fun{GEN}{Fq_elldivpolmod}{GEN a4,GEN a6, long n, GEN h, GEN T, GEN p}
returns the $n$-division polynomial of the elliptic curve $E$ modulo the
polynomial $h$.

\fun{void}{FpXQ_elltwist}{GEN a4, GEN a6, GEN T, GEN p, GEN *pA4, GEN *pA6}
sets \kbd{*pA4} and \kbd{*pA6} to the corresponding parameters for the
quadratic twist of $E$.

\subsec{\kbd{FpXQE}}

\fun{GEN}{FpXQE_changepoint}{GEN P, GEN m, GEN a4, GEN T, GEN p} returns the
image $Q$ of the point $P$ on the curve $E:y^2=x^3+a_4\*x+a_6$ by the
coordinate change $m$ (which is a \kbd{FpXQV}).

\fun{GEN}{FpXQE_changepointinv}{GEN P, GEN m, GEN a4, GEN T, GEN p} returns
the image $Q$ on the curve $E:y^2=x^3+a_4\*x+a_6$ of the point $P$ by the
inverse of the coordinate change $m$ (which is a \kbd{FpXQV}).

\fun{GEN}{FpXQE_add}{GEN P, GEN Q, GEN a4, GEN T, GEN p}

\fun{GEN}{FpXQE_sub}{GEN P, GEN Q, GEN a4, GEN T, GEN p}

\fun{GEN}{FpXQE_dbl}{GEN P, GEN a4, GEN T, GEN p}

\fun{GEN}{FpXQE_neg}{GEN P, GEN T, GEN p}

\fun{GEN}{FpXQE_mul}{GEN P, GEN n, GEN a4, GEN T, GEN p}

\fun{GEN}{random_FpXQE}{GEN a4, GEN a6, GEN T, GEN p}

\fun{GEN}{FpXQE_log}{GEN P, GEN G, GEN o, GEN a4, GEN T, GEN p} Let \kbd{G} be a
point of order \kbd{o}, return $e$ such that $e.P=G$. If $e$ does not exists,
the result is currently undefined.

\fun{GEN}{FpXQE_order}{GEN P, GEN o, GEN a4, GEN T, GEN p} returns the order
of $P$ in the group $E(\F_p[X]/(T))$, where $o$ is a multiple of the order of
$P$, or its factorization.

\fun{GEN}{FpXQE_tatepairing}{GEN P,GEN Q, GEN m, GEN a4, GEN T, GEN p}
returns the Tate pairing of the point of $m$-torsion $P$ and the point $Q$.

\fun{GEN}{FpXQE_weilpairing}{GEN P,GEN Q, GEN m, GEN a4, GEN T, GEN p}
returns the Weil pairing of the points of $m$-torsion $P$ and $Q$.

\fun{GEN}{RgE_to_FpXQE}{GEN P, GEN T, GEN p} returns the \kbd{FpXQE} obtained
by applying \kbd{Rg\_to\_FpXQ} coefficientwise.

\section{Functions related to modular polynomials}

Variants of \tet{polmodular}, returning the modular polynomial of prime
level $L$ for the invariant coded by \kbd{inv} (0: $j$, 1: Weber-$f$, see
\tet{polclass} for the full list).

\fun{GEN}{polmodular_ZXX}{long L, long inv, long vx, long vy}
returns a bivariate polynomial in variables \kbd{vx} and \kbd{vy}.

\fun{GEN}{polmodular_ZM}{long L, long inv} returns a matrix of
(integral) coefficients.

\fun{GEN}{Fp_polmodular_evalx}{long L, long inv, GEN J, GEN p, long v,
int derivs} returns the modular polynomial evaluated
at $J$ modulo the prime $p$ in the variable $v$ (if \kbd{derivs} is nonzero,
returns a vector containing the modular polynomial and its first and second
derivatives, all evaluated at $J$ modulo~$p$).

\subsec{Functions related to modular invariants}

\fun{void}{check_modinv}{long inv} report an error if \kbd{inv} is not a
valid code for a mdular invariant.

\fun{int}{modinv_good_disc}{long inv, long D} test whether the
invariant \kbd{inv} is defined for the discriminant \kbd{D}.

\fun{int}{modinv_good_prime}{long inv, long D} test whether the
invariant \kbd{inv} is defined for the prime \kbd{p}.

\fun{long}{modinv_height_factor}{long inv} return the height factor
of the modular invariant \kbd{inv} with respect to the $j$-invariant.
This is an integer $n$ such that the $j$-invariant is asymptotically
of the order of the $n$-th power of the invariant \kbd{inv}.

\fun{long}{modinv_is_Weber}{long inv} test whether the invariant
\kbd{inv} is a power of Weber $f$.

\fun{long}{modinv_is_double_eta}{long inv} test whether the invariant
\kbd{inv} is a double $\eta$ quotient.

\fun{long}{disc_best_modinv}{long D} the integer $D$ being a negative discriminant,
return the modular invariant compatible with $D$ with the highest height
factor.

\fun{GEN}{Fp_modinv_to_j}{GEN x, long inv, GEN p} Let $\Phi$ the modular equation
between $j$ and the modular invariant \kbd{inv}, return $y$ such that
$\Phi(y,x)=0\pmod{p}$.

\section{Other curves}

The following functions deal with hyperelliptic curves in weighted projective
space $\P_{(1,d,1)}$, with coordinates $(x,y,z)$ and a model of the form
$ y^2 = T(x,z)$, where $T$ is homogeneous of degree $2d$, and squarefree.
Thus the curve is nonsingular of genus $d-1$.

\fun{long}{hyperell_locally_soluble}{GEN T, GEN p} assumes that $T\in\Z[X]$ is
integral. Returns $1$ if the curve is locally soluble over $\Q_p$, $0$
otherwise.

\fun{long}{nf_hyperell_locally_soluble}{GEN nf, GEN T, GEN pr} let $K$
be a number field, attached to \kbd{nf}, \kbd{pr} a \var{prid} attached
to some maximal ideal $\goth{p}$; assumes that $T\in\Z_K[X]$ is integral.
Returns $1$ if the curve is locally soluble over $K_{\goth{p}}$. The argument
\kbd{nf} is a true \var{nf} structure.

\newpage
